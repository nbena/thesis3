\section{Dettagli}

\subsection{Certificate Management}
Requisiti fondamentale per il funzionamento di OpenVPN è che il client e server dispongano
di certificati X509 firmati da una stessa CA.
A tale scopo, si dispone di una CA interna, la cui chiave pubblica e privata è deployata
assieme a questo servizio\footnote{In futuro si potrebbero maggiori di protezione per la
chiave privata, oltre alla protezione \textit{fisica} (es: regole di firewall) dell'accesso
al servizio.}. Attualmente il certificato è self-signed, in futuro potrebbe non essere così.

OpenVPN richiede, sia per client sia per server, che siano settati alcuni attributi
specifici nei certificati, da cui poi derivano le direttive
``\texttt{remote-cert-eku "TLS Web Server/Client
Authentication"}''. Ogni certificato ha un seriale univoco, il quale viene generato
in maniera pseudocasuale usando una funzione preposta dalla libreria che viene
utilizzata.

Il microservizio mantiene anche una CRL -- \textit{Certificate Revocation List}, la quale
contiene l'elenco dei seriali dei certificati revocati, la ragione (nel cui caso è ``\texttt{unspecified}''),
ed è firmata digitalmente dalla chiave privata della CA.\\
Ogni volta che si revoca un nuovo client, occorre creare una nuova CRL inserendo tutti i
precedenti certificati revocati aggiungendo quello nuovo. La CRL deve poi essere distribuita 
a tutti i server, ma questo è compito di un altro modulo.

I certificati sono generati utilizzando l'algoritmo \texttt{ECDSA}, con livello di sicurezza
128 bit per client e server. Secondo le raccomandazioni del NIST, un livello di sicurezza di
128 è sicuro fino al 2030 \ref{add fucking ref}. Si può passare senza
problemi ad un livello di sicurezza maggiore (192 o 256), semplicemente modificando i settaggi
del microservizio.

Per il certificato della CA, si utilizza invece una curva ellittica con livello di sicurezza
di 256.


Tutto il codice per la gestione di queste funzionalità è implementato nel modulo \texttt{openssl}
mediante il file \texttt{openssl.py}, di cui di seguito si mostra un estratto leggermente
modificato.
Si può vedere come si crea un certificato ed una CRL vuota (metodo
``\texttt{create\_empty\_crl}'', per la quale vi
si aggiunge un certificato fake il cui numero seriale vale 1.
Il tipo di certificato (client o server), dipende dai parametri con cui si chiama il metodo
``\texttt{create\_keys\_and\_cert}''.
Alcuni metodi sono omessi per brevità, tra essi il costruttore.
\begin{minted}{python}
import cryptography.hazmat.backends as backends
import cryptography.hazmat.primitives.asymmetric.ec as ec
import cryptography.hazmat.primitives.hashes as hashes
import cryptography.hazmat.primitives.serialization as serialization
import cryptography.x509 as x509


class OpenSSL(object):

    def __get_ca_private_key(self):
        if self.__root_ca_key_path is None:
            raise Exception('You should have provided the ca\'s key')

        ca_priv_key_file = open(self.__root_ca_key_path, 'rb')

        ca_private_key = serialization.load_pem_private_key(
            data=ca_priv_key_file.read(),
            backend=backends.default_backend(),
            password=None,
        )

        ca_priv_key_file.close()
        return ca_private_key


    def _get_revoked(self, serial):
        return x509.RevokedCertificateBuilder(
        ).serial_number(
            int(serial)
        ).revocation_date(
            datetime.datetime.today()
        ).add_extension(
            x509.CRLReason(x509.ReasonFlags.unspecified),
            critical=False
        ).build(backend=backends.default_backend())


    def _get_crl_builder(self, issuer_name, crl_number, last_update):
        return x509.CertificateRevocationListBuilder(
        ).issuer_name(
            issuer_name
        ).last_update(
            last_update
        ).next_update(
            last_update + self.NEXT_UPDATE
        ).add_extension(
            x509.CRLNumber(crl_number),
            critical=False
        )    


    def create_empty_crl(self):
        """
        Generates a certification revocation list (CRL) with
        a single revoked cert with serial = 1.

        :rtype  str the path to the CRL.
        """

        cert = self.__get_ca_cert()
        if self.__crl_path is None or self.__crl_path == '':
            self.__crl_path = os.path.abspath(os.path.join(
                self.__directory, 'crl.pem'))

        crl_builder = self._get_crl_builder(
            issuer_name=cert.issuer,
            crl_number=1,
            last_update=datetime.datetime.today()
        )

        revoked = self._get_revoked(1)
        crl_builder = crl_builder.add_revoked_certificate(revoked)

        crl = crl_builder.sign(
            private_key=self.__get_ca_private_key(),
            algorithm=hashes.SHA512(),
            backend=backends.default_backend()
        )

        self.__write_public_bytes(self.__crl_path, crl)
        return self.__crl_path

    def _basic_cert_setup(self, data, cert_type, serial=None):
        """
        Does a basic setup for the certificate.

        -   deciding which curve to use
        -   settings all of the attributes of the cert
        -   generate a serial if it's not given

        :rtype  tuple:
            -   tuple[0]    private_key
            -   tuple[1]    public_key
            -   tuple[2]    str serial
            -   tuple[3]    x509.CertificateBuilder the cert builder
            -   tuple[4]    datetime.datetime   issuing date
            -   tuple[4]    datetime.datetime   expiration date
        """
        common_name = data['CommonName']
        days = data.get('Days', 365)
        email = data['Email']
        organization_name = data['OrganizationName']

        self.__current_security_level =\
            self.__get_current_security_level(cert_type)

        curve, curve_str =\
            self.__get_curve(self.__current_security_level)

        self.__curve = curve_str

        private_key, public_key = self.__gen_keys_pair(curve)

        cert_attributes = x509.Name([
            x509.NameAttribute(x509.oid.NameOID.COUNTRY_NAME, 'IT'),
            x509.NameAttribute(x509.oid.NameOID.ORGANIZATION_NAME,
                               organization_name),
            x509.NameAttribute(x509.oid.NameOID.COMMON_NAME,
                               common_name),
            x509.NameAttribute(x509.oid.NameOID.EMAIL_ADDRESS,
                               email),
        ])

        if serial is None:
            serial = gen_random_serial()

        issuing_date = datetime.datetime.now(datetime.timezone.utc)
        expiration_date = issuing_date + datetime.timedelta(days=days)

        builder = x509.CertificateBuilder(
        ).subject_name(
            cert_attributes
        ).not_valid_before(
            issuing_date
        ).not_valid_after(
            expiration_date
        ).public_key(
            public_key
        ).serial_number(
            serial
        )

        return (private_key, public_key, serial, builder,
                issuing_date, expiration_date)        

    def __gen_keys_pair(self, curve):
        private_key = ec.generate_private_key(
            curve=curve,
            backend=backends.default_backend())
        public_key = private_key.public_key()

    def create_keys_and_cert(self, data, cert_type, serial=None):
        """
        Does the following:
        -   generates a key pair using ECDSA
        -   generates a certificate for that public key
        -   save to two files the certificate and the private key

        :param  data    dict, keys:
            -   `CommonName`   str the `CommonName` to insert into
                the cert
            -   `Email` str the `email` to insert into the cert
            -   `Days`  int the days for which the cert is valid
                starting from now
            -   `OrganizationName`  str the `OrganizationName` to
                insert into the cert
        :param  cert_type   enum of `CertType`
        :param  serial  str the optional serial to insert into the
                certificate.
                If `None`, a brand new one is generated using
                randomness.

        :note   `serial`, if is not `None`, should be pseudorandom.

        :rtype  tuple, as follows:
            -   tuple[0] = (path-to-cert, path-to-privkey-file)
            -   tuple[1] = (serial, issuing date, expiration date)

        :raises ValueError if `serial == 1`
        """

        if serial is not None and (serial == 1 or serial == '1'):
            raise ValueError('serial cant be = 1')

        truncated_name = utils.get_truncated_name(data['CommonName'])

        self.__key_file_name = os.path.abspath(os.path.join(
            self.__directory, truncated_name+'.key'))

        self.__cert_file_name = os.path.abspath(os.path.join(
            self.__directory, truncated_name+'.crt'))

        result = self._basic_cert_setup(data, cert_type, serial)

        private_key = result[0]
        # public_key = result[1]
        serial = result[2]
        cert_builder = result[3]
        issuing_date = result[4]
        expiration_date = result[5]

        extensions = None
        if cert_type == CertType.server:
            extensions = x509.ExtendedKeyUsage(
                [x509.oid.ExtendedKeyUsageOID.SERVER_AUTH])
        elif cert_type == CertType.client:
            extensions = x509.ExtendedKeyUsage(
                [x509.oid.ExtendedKeyUsageOID.CLIENT_AUTH])

        cert_builder = cert_builder.add_extension(extensions,
            critical=False)

        ca_private_key = self.__get_ca_private_key()
        ca_cert = self.__get_ca_cert()

        cert_builder = cert_builder.issuer_name(
            ca_cert.subject)

        cert = cert_builder.sign(private_key=ca_private_key,
                                 algorithm=hashes.SHA512(),
                                 backend=backends.default_backend())

        ca_private_key = None

        private_key_output = open(self.__key_file_name, 'wb')
        private_key_output.write(private_key.private_bytes(
            encoding=serialization.Encoding.PEM,
            format=serialization.PrivateFormat.TraditionalOpenSSL,
            encryption_algorithm=serialization.NoEncryption()
        ))
        private_key_output.close()
        private_key = None

        self.__write_public_bytes(self.__cert_file_name, cert)

        return (self.__cert_file_name,
                self.__key_file_name), (serial,
                                        issuing_date,
                                        expiration_date)
\end{minted}


\subsection{IP Mapping}

\subsection{Trasferimento file}
Vi sono diverse situazioni che richiedono il trasferimento di file
verso i VPN server di MoonCloud.
\begin{description}
	\item[Creazione di un nuovo server]Si trasferiscono sul server il file
	di configurazione principale di OpenVPN, la relativa chiave privata e certificato,
	la chiave pubblica della CA, la CRL.
	Vengono trasferiti anche due script:
	\begin{itemize}
		\item \texttt{move-files-server.sh}: dopo aver trasferito i file appena elencati
		      in una cartella sul server, viene lanciato questo script, il quale legge il file
		      di configurazione principale e da esso deduce la struttura di directory e file
		      richiesta, e quindi la crea.
		      Poiché le operazioni che lo script deve compiere non sono poche, si è
		      preferito scegliere questa strada anziché invocare direttamente $n$ commandi 
		      lungo la connessione SSH per ridurre l'I/O e quindi aumentare le prestazioni.
		\item \texttt{deleclient.py}. Quando si revoca un client, esso deve essere
		      \textit{totalmente eliminato}, ciò si traduce anche nell'eliminare ogni riferimento
		      ad esso dai file \texttt{client-up.sh} e \texttt{client-down.sh}, quindi eliminare
		      il corpo degli \texttt{if} all'interno dei due file. Questa operazione viene
		      svolta direttamente sui server, mediante l'invocazione di questo script.
		      Analogamente a quanto detto per lo script precedente, e a maggior ragione in questo
		      caso, eseguire \texttt{deleclient.py} direttamente sul server riduce di molto l'I/O:
		      se non si fosse scelta questa opzione sarebbe stato necessario trasferire i due file,
		      modificarli, e di nuovo caricarli sul server.
	\end{itemize}
	\item[Creazione di un nuovo client]E' necessario creare un file con lo stesso
	nome del \texttt{CommonName} presente nel certificato del client, il cui
	contenuto è la direttiva ``\texttt{iroute}''\footnote{Si rimanda al capitolo dedicato
	alle configurazioni di OpenVPN.}. Esso deve essere quindi trasferito in una cartella
	specifica nel VPN server. Oltre a tale file, è necessario modificare gli script
	\texttt{client-up.sh} e \texttt{client-down.sh}, aggiungendo un nuovo ramo
	\texttt{if} specifico per questo client.
	\item[Rinnovo certificato server]L'unico trasferimento da effettuare è il nuovo
	certificato e chiave pubblica, i quali devono essere spostati nella directory
	specificata, ancora una volta, nel file di configurazione di OpenVPN nel server.
	Rimane comunque la necessità di riavviare il server affinché questi cambiamenti abbiano
	effetto.
	\item[Rinnovo certificato client]E' necessario revocare il certificato precedente,
	quindi propagare la nuova CRL ad ogni server di MoonCloud.
	\item[Revoca certificato client]Similmenete al punto precedente, per ciò che concerne
	il trasferimento file, l'unica azione da effettuare è propagare la nuova
	CRL ai VPN server.
\end{description}

Il trasferimento dei file verso i server è una parte fondamentale del microservizio,
pertanto deve essere il più possibile robusto.
Il modo in cui viene effettuato è stato chiamato \textbf{\textit{asynchronous transferring}},
perché è svolto da un thread specifico a questo scopo, separato dagli altri thread di
esecuzione, implementato dalla classe \texttt{SSHBackgroundWorker}.
Il modulo \texttt{controllers} organizza i trasferimenti da effettuare e crea degli
oggetti (\textit{dizionari} Python) e li mette su una coda condivisa con il thread di
trasferimento.
La computazione \texttt{SSHBackgroundWorker} è composta da un loop infinito, dentro
il quale si estraggono elementi dalla coda: ciascun elemento contiene tutte le informazioni
che servono per trasferirlo sul server (o sui server) correttamente. I trasferimenti
quindi sono eseguiti serialmente, nell'ordine con cui sono stati inseriti
nella coda (\textit{FIFO}).
Ogni volta che si deposita un nuovo job sulla coda, si riceve un \texttt{WorkID}, il quale,
a secondo del tipo di job, può essere numerico, oppure alfanumerico. In quest'ultimo
caso, l'ID è generato in maniera pseudocasuale mediante del materiale
prelevato dal generatore pseudocasuale del sistema operativo.
Parallelamente alla coda infatti, \texttt{SSHBackgroundWorker} mantiene anche un \textit{dizionario}
(detto \texttt{shared\_dict})
nel quale si mantengono a loro volta dei dizionari, ciascuno con le seguenti informazioni:
\begin{itemize}
    \item ID del job
    \item tipo (es: \textit{trasferimento server}, \textit{rinnovo certificato}, ecc\ldots)
    \item stato, gli stati possibili sono: \texttt{Completato},\texttt{Pendente},\texttt{In corso},
    \texttt{Concluso con errore}
    \item ulteriori informazioni, come una descrizione in caso di errore
\end{itemize}
Ogni volta che si inserisce un nuovo lavoro nella coda, si aggiunge un corrispondente oggetto
nello \texttt{shared\_dict}. L'accesso ad esso è protetto da lock.

Una volta che il job è stato depositato nella coda, si ritorna immediatamente una risposta HTTP
al chiamante. Il body della risposta conterrà, tra le altre cose, anche il \texttt{WorkID}.
Il chiamante può quindi fare del \textit{polling} per sapere lo stato di un certo lavoro: dietro
le quindi si accede allo \texttt{shared\_dict}.

La ragione per cui il trasferimento segue questo paradigma non è prestazionale, bensì è per
garantire l'integrità dei file sui server.
Si pensi ad un semplice esempio, per cui si fanno due richieste a questo microservizio nello stesso
momento, ed entrambe richiedono di aggiungere un nuovo client allo stesso server.
Si supponga che vi siano due thread, ciascuno che gestisce una delle due risposte. Essi quindi
accedono al file \texttt{client-down.sh} sul server, lo modificano aggiungendo informazioni
relative solo al proprio client, quindi il primo thread trasferisce il file, immediatamente dopo
il secondo trasferisce la propria versione, ed ecco che la modifica fatta dal primo viene
sovrascritta.
Un'esecuzione seriale mette al sicuro da questi problemi. Anche con quando il numero di server
si farà potenzialmente alto, un thread è stato ritenuto sufficiente perché i file trasferiti
sono sempre testuali e relativamente brevi. L'utilizzo di certificati su curva ellittica
ha l'effetto di avere anche tali file di dimensioni contenute.