\section{Requisiti}
Il microservizio implementa delle API REST per le seguenti funzionalità:
\begin{description}
  \item[Configurazione VPN server]Creazione di tutti i file di configurazione
  necessari ad OpenVPN server per funzionare.
  \item[Configurazione device client]Creazione di tutti i file richiesti al
  device VPN client, compresi quelli necessari ad OpenVPN ed il file per nftables.
  \item[Gestione dei certificati]Client e server hanno bisogno di certificati
  X509 per poter stabilire un collegamento VPN. E' richiesto di creare tali
  certificati, sia per client sia per server, e di gestirne il loro ciclo di vita,
  compreso di rinnovo ed eventuale revoca.
  \item[Trasferimento file]Una volta che file di configurazione e certificati
  sono stati creati, devono essere trasferiti sul VPN server in maniera \textit{sicura}.
  Allo stesso modo, ogni volta che si crea un nuovo client, occorre aggiornare
  la configurazione del server a cui si connetterà (si pensi ai file \texttt{client-up.sh}
  e \texttt{client-down.sh}). Questo trasferimento deve inoltre preservare l'integrità
  della configurazione del server.
  \item[Visualizzare informazioni]Devono essere disponbili delle API che ritornino informazioni
  su specifici client e server, incluse anche informazioni sui certificati.
  \item[Gestione dell'IP mapping]Questa attività si scompone in:
  \begin{itemize}
    \item ogni volta che si crea un nuovo client VPN, allocare ad esso $n$ reti mappate
    univoche per il server al quale esso si collegherà
    \item dato un indirizzo IP originale di un certo client, ritornare l'IP mappato
    \item \textit{Blacklisting}: deve essere possibile specificare che alcuni indirizzi
    IP (indirizzi di host o indirizzi di rete) o nomi di dominio\footnote{Si intende
    l'indirizzo IP corrispondente a tale nome di dominio.} non devono essere mai assegnati
    a reti mappate.
  \end{itemize}
  \item[Revocare client]Per vari ragioni, deve essere necessario poter revocare un
  client. Questa attività si traduce nell'invalidare il certificato ad esso relativo
  ed alla cancellazione del mappaggio dal database.
  La revoca deve essere propagata ad ogni server in modo che esso non riesca per nessun
  motivo a connettersi nuovamente alla VPN di MoonCloud.
\end{description}

Attualmente, l'intero deployment del mio microservizio si compone di un container
su cui è in esecuzione Django, cioè il server web che espone le API REST, e
di un secondo container su cui è in esecuzione PostgreSQL.
Non è escluso che in futuro questo servizio venga separato in microservizi
ancora più piccoli, ad esempio si può pensare di creare un microservizio
che gestisca solo le funzionalità legate ai certificati. Realizzarlo non sarebbe
difficile poiché ho cercato di mantenere il più possibile separate ed independenti
tra loro le singole componenti (\textit{Low Coupling}).

\section{Architettura}
Il microservizio è composto da diversi moduli, ciascuno dei quali ben
specializzato per una certa funzionalità, ad esempio: creazione file di configurazione
server, creazione file per nftables, trasferimento file, ecc\ldots
Il modulo \texttt{controllers.py}  definisce
una serie di funzioni che, tramite \texttt{views.py} risponde direttamente alle
chiamate alle API del servizio. Si cura di chiamare i metodi delle classi di tutti
gli altri moduli.
La figura \ref{fig:microservice-archi} mostra in maniera grafica i diversi di cui
si compone il microservizio, i quali sono descritti in maggior dettaglio di seguito.

\begin{description}
  \item[\texttt{client}]Questo modulo si occupa delle configurazioni relative ai
  client VPN. In particolare, contiene classi responsabili di creare il file
  di configurazione principale per un VPN client, creare i file client-specific
  da trasferire sul server, creare il file di script per nftables dato il mapping
  da usare per quel client.
  \item[\texttt{dnmcp}]Si occupa in maniera specifica di creare il mapping date
  le reti originali. Una volta che è stato definito, è responsabile di salvarlo
  nel database.
  \item[\texttt{dns}]Un modulo specifico dedicato alla risoluzione dei nomi
  DNS relative al blacklisting. Oltre a risolvere nuovi nomi ricevuti in input,
  è anche responsabile di aggiornare tutti quelli presenti nel database.
  La risoluzione viene fatta in maniera asincrona, ovvero si fanno tutte le query
  necessarie, una dopo l'altra senza aspettare risposta, e solo quando l'ultima
  è stata completata si aspettano i risultati complessivi. Questo consente
  di aumentare le performance mediante I/O non bloccante.
  \item[\texttt{openssl}]Una classe preposta alla creazione, revoca ed aggiornamento
  dei certificati di client e server, e gestione della CRL. Utilizza una
  libreria Python che a sua volta utilizza OpenSSL come backend.
  \item[\texttt{server}]L'unico compito dell'unica classe di questo modulo è quello
  di creare il file di configurazione principale per un VPN server, dati una serie
  di parametri in input.
  \item[\texttt{trans}]Il nome del modulo è l'abbreviazione di ``transfer''. Si
  compone di una serie di classi responsabili di trasferire file sui server mediante
  \texttt{scp} (trasferimento file su un canale sicuro SSH) e di creare
  la struttura di directory così come assunta nel file di configurazione di OpenVPN
  (esempio: creare le cartelle \texttt{/etc/openvpn/\ldots/ccd/} e \texttt{/etc/openvpn/\ldots/certs}).
  \item[\texttt{workers}]Contiene la classe \texttt{SSHBackgroundWorker} la quale, mediante
  un thread separato, si occupa del trasferimento mediante il modulo \texttt{trans}.
  Maggiori dettagli su questa funzionalità verranno fornite in seguito durante
  il capitolo.
\end{description}

Infine si descrivono i seguenti moduli:
\begin{description}
  \item[\texttt{controllers}]Organizzato in dversi file, questo modulo si occupa di
  \textit{unire} le funzionalità messe a disposizione da tutti gli altri moduli,
  oltre a creare direttamente i cosidetti ``\texttt{Model}'' nella terminologia Django,
  ovvero i dati presenti nel database\footnote{\url{https://docs.djangoproject.com/en/2.1/topics/db/models/}}.
  \item[\texttt{views}]Definisce le API a cui il microservizio risponde, deserializza e serializza
  gli input e gli output, fà le chiamate a \texttt{controllers}, ne intercetta eventuali errori.
\end{description}

%\subsection{Singole componenti}

\section{Dettagli}

\subsection{Creazione dei certificati}

\subsection{Gestione del mapping}

\subsubsection{Blacklisting}

\subsection{Creazione dei file -- client}

\subsection{Creazione dei file -- server}

\subsection{Trasferimento file ai server}

\section{API}
Questa sezione provvede a dare una rapida overview delle principali API REST
esposte. Per questioni di spazio, vengono riportate solo le API più importanti,
e di ciascuna si dà una breve descrizione (questo documento \textit{non} è
la documetazione per \texttt{MoonCloud\_VPN}).
