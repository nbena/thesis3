\section{Requisiti}
Il mio microservizio implementa delle API REST per le seguenti funzionalità:
\begin{itemize}
  \item creazione di tutti i file di configurazione e strutture di directory
  necessarie per creare un nuovo OpenVPN server
  \item creazione di tutti i file di configurazione necessari per un nuovo client VPN
  \item creazione di coppie di chiavi e certificati sia per server sia per client VPN
  \item gestione dell'IP mapping:
  \begin{itemize}
    \item ogni volta che si crea un nuovo client VPN, allocare ad esso $n$ reti mappate
    univoche
    \item implementazione di una API che dato un indirizzo IP originale restituisca
    l'indirizzo IP mappato
    \item gestione di blacklist di indirizzi IP, reti, nomi di dominio che non
    possono essere assegnati a reti mappate
  \end{itemize}
  \item trasferimento dei file necessari ai server VPN
  \item revoca dei client, essa riguarda:
  \begin{itemize}
    \item cancellazione del mapping, se si richiede l'IP mappato per tale client
    l'API dovrà rispondere che non vi sono mappaggi disponibili
    \item revoca del certificato propagata a tutti i VPN server: se tale client
    provasse a connettersi in VPN i server rifiuterebbero la connessione
  \end{itemize}
\end{itemize}

Attualmente, l'intero deployment del mio microservizio si compone di un container
su cui è in esecuzione Django, cioè il server web che espone le API REST, e
di un secondo container su cui è in esecuzione PostgreSQL.
Non è escluso che in futuro questo servizio venga separato in microservizi
ancora più piccoli, ad esempio si può pensare di creare un microservizio
che gestisca solo le funzionalità legate ai certificati. Realizzarlo non sarebbe
difficile poiché ho cercato di mantenere il più possibile separate ed independenti
tra loro le singole componenti (\textit{Low Coupling}).

\section{Architettura}

%\subsection{Singole componenti}

\section{Dettagli}

\subsection{Creazione dei certificati}

\subsection{Gestione del mapping}

\subsubsection{Blacklisting}

\subsection{Creazione dei file -- client}

\subsection{Creazione dei file -- server}

\subsection{Trasferimento file ai server}

\section{API}
Questa sezione provvede a dare una rapida overview delle principali API REST
esposte. Per questioni di spazio, vengono riportate solo le API più importanti,
e di ciascuna si dà una breve descrizione (questo documento \textit{non} è
la documetazione per \texttt{MoonCloud\_VPN}).
