\section{Requisiti}
Il microservizio implementa delle API REST per le seguenti funzionalità:
\begin{description}
  \item[Configurazione VPN server]Creazione di tutti i file di configurazione
  necessari ad OpenVPN server per funzionare.
  \item[Configurazione device client]Creazione di tutti i file richiesti al
  device VPN client, compresi quelli necessari ad OpenVPN ed il file per nftables.
  \item[Gestione dei certificati]Client e server hanno bisogno di certificati
  X509 per poter stabilire un collegamento VPN. E' richiesto di creare tali
  certificati, sia per client sia per server, e di gestirne il loro ciclo di vita,
  compreso di rinnovo ed eventuale revoca.
  \item[Trasferimento file]Una volta che file di configurazione e certificati
  sono stati creati, devono essere trasferiti sul VPN server in maniera \textit{sicura}.
  Allo stesso modo, ogni volta che si crea un nuovo client, occorre aggiornare
  la configurazione del server a cui si connetterà (si pensi ai file \texttt{client-up.sh}
  e \texttt{client-down.sh}). Questo trasferimento deve inoltre preservare l'integrità
  della configurazione del server.
  \item[Visualizzare informazioni]Devono essere disponbili delle API che ritornino informazioni
  su specifici client e server, incluse anche informazioni sui certificati.
  \item[Gestione dell'IP mapping]Questa attività si scompone in:
  \begin{itemize}
    \item ogni volta che si crea un nuovo client VPN, allocare ad esso $n$ reti mappate
    univoche per il server al quale esso si collegherà
    \item dato un indirizzo IP originale di un certo client, ritornare l'IP mappato
    \item \textit{Blacklisting}: deve essere possibile specificare che alcuni indirizzi
    IP (indirizzi di host o indirizzi di rete) o nomi di dominio\footnote{Si intende
    l'indirizzo IP corrispondente a tale nome di dominio.} non devono essere mai assegnati
    a reti mappate.
  \end{itemize}
  \item[Revocare client]Per vari ragioni, deve essere necessario poter revocare un
  client. Questa attività si traduce nell'invalidare il certificato ad esso relativo
  ed alla cancellazione del mappaggio dal database.
  La revoca deve essere propagata ad ogni server in modo che esso non riesca per nessun
  motivo a connettersi nuovamente alla VPN di MoonCloud.
\end{description}


