\section{Architettura}
Il microservizio è composto da diversi moduli, ciascuno dei quali ben
specializzato per una certa funzionalità, ad esempio: creazione file di configurazione
server, creazione file per nftables, trasferimento file, ecc\ldots
Il modulo \texttt{controllers.py}  definisce
una serie di funzioni che, tramite \texttt{views.py} risponde direttamente alle
chiamate alle API del servizio. Si cura di chiamare i metodi delle classi di tutti
gli altri moduli.
La figura \ref{fig:microservice-archi} mostra in maniera grafica i diversi di cui
si compone il microservizio, i quali sono descritti in maggior dettaglio di seguito.

\begin{description}
  \item[\texttt{client}]Questo modulo si occupa delle configurazioni relative ai
  client VPN. In particolare, contiene classi responsabili di creare il file
  di configurazione principale per un VPN client, creare i file client-specific
  da trasferire sul server, creare il file di script per nftables dato il mapping
  da usare per quel client.
  \item[\texttt{dnmcp}]Si occupa in maniera specifica di creare il mapping date
  le reti originali. Una volta che è stato definito, è responsabile di salvarlo
  nel database.
  \item[\texttt{dns}]Un modulo specifico dedicato alla risoluzione dei nomi
  DNS relative al blacklisting. Oltre a risolvere nuovi nomi ricevuti in input,
  è anche responsabile di aggiornare tutti quelli presenti nel database.
  La risoluzione viene fatta in maniera asincrona, ovvero si fanno tutte le query
  necessarie, una dopo l'altra senza aspettare risposta, e solo quando l'ultima
  è stata completata si aspettano i risultati complessivi. Questo consente
  di aumentare le performance mediante I/O non bloccante.
  \item[\texttt{openssl}]Una classe preposta alla creazione, revoca ed aggiornamento
  dei certificati di client e server, e gestione della CRL. Utilizza una
  libreria Python che a sua volta utilizza OpenSSL come backend.
  \item[\texttt{server}]L'unico compito dell'unica classe di questo modulo è quello
  di creare il file di configurazione principale per un VPN server, dati una serie
  di parametri in input.
  \item[\texttt{trans}]Il nome del modulo è l'abbreviazione di ``transfer''. Si
  compone di una serie di classi responsabili di trasferire file sui server mediante
  \texttt{scp} (trasferimento file su un canale sicuro SSH) e di creare
  la struttura di directory così come assunta nel file di configurazione di OpenVPN
  (esempio: creare le cartelle \texttt{/etc/openvpn/\ldots/ccd/} e \texttt{/etc/openvpn/\ldots/certs}).
  \item[\texttt{workers}]Contiene la classe \texttt{SSHBackgroundWorker} la quale, mediante
  un thread separato, si occupa del trasferimento mediante il modulo \texttt{trans}.
  Maggiori dettagli su questa funzionalità verranno fornite in seguito durante
  il capitolo.
\end{description}

Infine si descrivono i seguenti moduli:
\begin{description}
  \item[\texttt{controllers}]Organizzato in dversi file, questo modulo si occupa di
  \textit{unire} le funzionalità messe a disposizione da tutti gli altri moduli,
  oltre a creare direttamente i cosidetti ``\texttt{Model}'' nella terminologia Django,
  ovvero i dati presenti nel database\footnote{\url{https://docs.djangoproject.com/en/2.1/topics/db/models/}}.
  \item[\texttt{views}]Definisce le API a cui il microservizio risponde, deserializza e serializza
  gli input e gli output, fà le chiamate a \texttt{controllers}, ne intercetta eventuali errori.
\end{description}