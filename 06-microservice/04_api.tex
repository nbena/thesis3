\section{API}
In questa sezione si descrivono le API REST messe a disposizione da \texttt{MoonCloud\_VPN}.
Input ed output sono descritti senza specificare l'esatta sintassi, poiché è fuori dallo scopo
di questo documenti fornire una documentazione delle API. Tutti gli input e gli
output sono in formato JSON.

\begin{description}
    \item[Creazione server]\texttt{POST /server}
    \begin{itemize}
        \item Descrizione: mediante questa API si creano tutte le configurazioni necessarie per
        un nuovo VPN server. Tutti i file che servono vengono trasferiti sul server mediante
        SSH.
        \item Input: un oggetto con i seguenti campi:
        \begin{itemize}
            \item nome DNS da inserire nel certificato ed ulteriori informazioni (es: per il campo \texttt{Email})
            \item porta di ascolto OpenVPN
            \item NET ID e subnet mask del tunnel VPN
            \item credenziali SSH, compresa la porta di ascolto 
        \end{itemize}
        \item Output: l'ID del server assegnato dal database ed il \texttt{WorkID} del lavoro di
        trasferimento.
    \end{itemize}

    \item[Stato creazione server]\texttt{GET /server/creation/<work-id>}
    \begin{itemize}
        \item Descrizione: viene utilizzata per controllare se il trasferimento dei file
        per la creazione di un nuovo server è stato completato.
        \item Input: nessuno
        \item Output: un oggetto JSON che riporta lo stato del trasferimento, il tipo di job,
        in questo caso di tipo \texttt{TransferServer}, il \texttt{WorkID}.
    \end{itemize}

    \item[Dettagli server]\texttt{GET /server/<server-id>}
    \begin{itemize}
        \item Descrizione: ritorna tutte le informazioni memorizzate relative ad un certo
        server.
        \item Input: nessuno
        \item Output: oggetto JSON che riporta le seguenti informazioni:
        \begin{itemize}
            \item ID
            \item nome DNS e porta OpenVPN di ascolto
            \item indirizzo IP interno mediante il quale è raggiungibile
            \item alcuni path della configurazione di OpenVPN
            \item NET ID e subnet mask del tunnel VPN
            \item informazioni sul certificato, tra cui data di firma, data di scadenza,
            numero seriale ed algoritmo di firma.
        \end{itemize}
    \end{itemize}

    \item[Rinnovo certificato server]\texttt{PUT /server/<server-id>/renew}
    \begin{itemize}
        \item Descrizione: i certificati hanno una scandenza, ed una volta scaduti
        non sono più validi e quindi non è più possibile effettuare il collegamento VPN,
        pertanto è necessario rinnovarli periodicamente.
        \item Input: nessuno
        \item Output: il \texttt{WorkID} relativo al trasferimento del certificato e
        della chiave privata sul server in questione.
    \end{itemize}

    \item[Stato rinnovo certificato server]\texttt{GET /server/renew/<work-id>}
    \begin{itemize}
        \item Descrizione: utilizzata per verificare lo stato del trasferimento verso
        il server del nuovo certificato e chiave privata.
        \item Input: nessuno
        \item Output: stato del trasferimento, tipo, \texttt{WorkID}.
    \end{itemize}

    \item[Creazione client]\texttt{POST /client}
    \begin{itemize}
        \item Descrizione: si chiama questa API per creare un nuovo client. Il microservizio
        crea tutti i file di configurazione, registra un mapping, trasferisce sul
        server designato i file necessari per poter stabilire un collegamento con esso.
        \item Input: un oggetto JSON che deve contenere i seguenti campi:
        \begin{itemize}
            \item ID server a cui il client si connetterà
            \item nome dell'azienda cliente a cui il device andrà
            \item nome mnemonico con cui identificare il device (es: \texttt{office1});
            assieme al nome dell'azienda andrà a comporre il \texttt{CommonName} del
            certificato
            \item email da inserire nel certificato
            \item lista di reti raggiungibil dal device
        \end{itemize}
        \item Output: un oggetto JSON costituito da tre campi, l'ID del device client
        all'interno del databse, il \texttt{WorkID}, uno zip file in formato base64 contenente tutti
        i file che devono essere poi trasferiti sul dispositivo, unitamente ad
        uno script che automatizza la creazione della struttura di directory assunta
        da OpenVPN.
    \end{itemize}

    \item[Stato creazione client]\texttt{GET /client/creation/<work-id>}
    \begin{itemize}
        \item Descrizione: analogamente ad ogni altra API relativa ad ottenere
        lo stato di un trasferimento, questa viene utilizzata per verificare
        quando i file relativi ad un nuovo client sono stati spostati sul server.
        \item Input: nessuno a parte la URI in sè; si noti che solo in questo caso il
        \texttt{WorkID} equivale all'ID del client ritornato al momento della creazione.
        \item Output: informazioni sullo stato del trasferimento.
    \end{itemize}

    \item[Dettagli client]\texttt{GET /client/<client-id>}
    \begin{itemize}
        \item Descrizione: mediante questa API si ottengono informazioni relative ad
        un certo VPN client.
        \item Input: nessuno
        \item Output: un oggetto JSON con vari campi tra cui:
        \begin{itemize}
            \item ID
            \item nome azienda e nome mnemonico
            \item informazioni relative al certificato del client, tra cui data di firma,
            data di scadenza, algoritmo, livello di sicurezza.
        \end{itemize}
    \end{itemize}

    \item[Rinnovo certificato client]\texttt{PUT /client/<client-id>/renew}
    \begin{itemize}
        \item Descrizione: come per il certificato del server, anche quello
        del client deve essere sempre valido, altrimenti il server rifiuterà
        la connessione VPN. Questa API serve a creare un nuovo certificato valido
        e contemporaneamente a revocare il precedente, distribuendo la nuova CRL
        a tutti i server.
        \item Input: nessuno
        \item Output: lo \texttt{WorkID} del trasferimento ed uno zip file
        in base64 contenente la nuova chiave privata e certificato.
    \end{itemize}

    \item[Stato rinnovo certificato client]\texttt{GET /client/renew/<work-id>}
    \begin{itemize}
        \item Descrizione: viene utilizzata per verificare lo stato del
        trasferimento della CRL verso tutti i server in seguito al rinnovo
        di un certificato client.
        \item Input: nessuno.
        \item Output: informazioni sullo stato del trasferimento.
    \end{itemize}

    \item[Revoca client]\texttt{DELETE /client/<client-id>/}
    \begin{itemize}
        \item Descrizione: per numerose ragioni, tra cui il semplice fatto
        di non servire più, è necessario dover cancellare un VPN client.
        Tra le azioni coinvolte, oltre a cancellare vari record dal database,
        occorre revocare il certificato usato da tale client, e propagare la
        nuova CRL a tutti i server.
        \item Input: nessuno
        \item Output: lo \texttt{WorkID} relativo alla propagazione della CRL.
    \end{itemize}

    \item[Stato revoca client]\texttt{GET /revocation/<work-id>}
    \begin{itemize}
        \item Descrizione: si usa questa API per verificare lo stato
        del trasferimento della CRL verso i VPN server in seguito alla
        cancellazione di un client.
        \item Input: nessuno
        \item Output: informazioni sullo stato del trasferimento.
    \end{itemize}

    \item[Ottenimento IP mappato]\texttt{GET /client/<client-id>/mapping}
    \begin{itemize}
        \item Descrizione: questa API implementa una delle funzionalità
        più importanti, ovvero fornire la versione mappata di un indirizzo
        IP originale per un certo VPN client.
        \item Input: il body della richiesta prevede un oggetto JSON contenente
        come unico campo l'IP richiesto.
        \item Output: un oggetto JSON composto dall'IP originale e dall'IP mappato.
    \end{itemize}

    \item[Aggiunta di un'esclusione]\texttt{POST /blacklist/add}
    \begin{itemize}
        \item Descrizione: è necessario poter definire delle reti o
        indirizzi IP (o nomi di dominio tradotti poi in IP) che non
        devono essere mai utilizzati come reti mappate. Mediante questa
        API è possibile aggiungere nuove esclusioni alla blacklist.
        \item Input: un oggetto JSON con diverse combinazioni,
        tra cui un singolo indirizzo IP, una rete (NET ID e subnet mask),
        un nome di dominio. Vi è anche un campo \texttt{isInternal} per specificare
        se quell'esclusione è relativa a MoonCloud (ad esempio, si vuole escludere
        l'indirizzo IP che contiene la repository Docker).
        \item Output: un oggetto JSON con gli stessi campi dell'input ed in
        più l'ID assegnato dal database a quella particolare esclusione.
    \end{itemize}

    \item[Elenco blacklist]\texttt{GET /blacklist/all}
    \begin{itemize}
        \item Descrizione: si utilizza questa API per visualizzare l'intera
        blacklist.
        \item Input: nessuno
        \item Output: si ritorna un array di oggetti JSON in cui ciascun oggetto
        rappresenta un'esclusione, tra i vari campi si mostra anche l'ID.
    \end{itemize}

    \item[Aggiornare blacklist]\texttt{PUT /blacklist/update}
    \begin{itemize}
        \item Descrizione: i nomi di dominio salvati nella blacklist devono
        essere aggiornati per riflettere i cambiamenti nell'associazione
        indirizzo IP -- nome DNS. Si noti che questo aggiornamento è fatto
        anche ogni volta che si crea un nuovo client, per cui si può presumere
        che quest'API non sarà chiamata spesso.
        \item Input: nessuno
        \item Output: un oggetto JSON con i seguenti campi:
        \begin{itemize}
            \item una lista di dei nomi DNS aggiornati
            \item una lista dei nomi DNS che non sonos stati aggiornati a causa
            di errori
            \item una descrizione degli errori incontrati durante la risoluzione.
        \end{itemize}
    \end{itemize}

    \item[Vedere specifiche esclusioni]\texttt{GET /blacklist/dns/internal|external|all}
    \begin{itemize}
        \item Descrizione: questa API è stata pensata appositamente per visualizzare solo
        le esclusioni che hanno un nome DNS registrato. Le si possono ritornare
        tutte (``\texttt{all}''), solo quelle esterne (``\texttt{external}''),
        oppure solo quelle interne (``\texttt{internal}'').
        \item Input: nessuno
        \item Output: una lista di oggetti JSON, per ciascuno si mostra il nome DNS,
        se interna o no, e l'ID.
    \end{itemize}

    \item[Dettagli esclusione]\texttt{GET /blacklist/<blacklist-id>}
    \begin{itemize}
        \item Descrizione: utilizzata per visualizzare
        una particolare esclusione dalla blacklist, dato il suo ID.
        \item Input: nessuno
        \item Output: un oggetto JSON con tutte le informazioni relative all'esclusione.
    \end{itemize}

    \item[Cancellare esclusione]\texttt{DELETE /blacklist/<blacklist-id>}
    \begin{itemize}
        \item Descrizione: mediante quest'API si cancella una particolare esclusione. 
        \item Input: nessuno
        \item Output: nessuno
    \end{itemize}

    \item[Specificare il supporto a ChaCha20]\texttt{PUT /chachasupport/true|false}
    \begin{itemize}
        \item Descrizione: l'algoritmo di cifratura \texttt{ChaCha20} è
        supportato solo in alcune versioni di OpenVPN a seconda di come sia stato compilato.
        Questa API presuppone che, ad esempio, da un certo punto in poi tale algoritmo sia
        sempre supportato sui nuovi client e server, e che si possa specificare tale
        supporto mediante questa API senza dover riavviare il microservizio (poiché
        il valore di default è specificato nei settaggi).
        \item Input: nessuno
        \item Output: nessuno
    \end{itemize}

    \item[Refresh CRL]\texttt{PUT /crl}
    \begin{itemize}
        \item Descrizione: tra i vari campi che si inseriscono in una CRL vi
        è ``\texttt{NextUpdate}'', ovvero per quando ci si aspetta il prossimo
        aggiornamento. OpenVPN, nel controllo della CRL, verifica anche il valore di
        questo campo, e se la CRL non è aggiornata il server rifiuterà nuove connessioni,
        pertanto è importante mantenerla aggiornata. In questo caso si genera una
        nuova CRL con gli stessi certificati revocati della precedente, ma con data
        di \texttt{NextUpdate} posticipata di un anno. Dopo aver generato una
        nuova CRL, viene distribuita a tutti i server.
        \item Input: nessuno
        \item Output: lo \texttt{WorkID} relativo alla distribuzione della CRL
        a tutti i VPN server di MoonCloud.
    \end{itemize}

    \item[Stato refresh CRL]\texttt{GET /crl/refresh/<work-id>}
    \begin{itemize}
        \item Descrizione: si usa l'API in questione per vedere lo stato
        del trasferimento della CRL ai server in seguito ad un refresh.
        \item Input: nessuno
        \item Output: nessuno
    \end{itemize}

    \item[Ultimi aggiornamenti alla CRL]\texttt{GET /crl}
    \begin{itemize}
        \item Descrizione: questa API ritorna una lista degli ultimi dieci aggiornamenti
        fatti alla CRL. I dati vengono prelevati da un'apposita tabella
        dal database, la quale è aggiornata solo dopo che la CRL modificata è stata distribuita
        a tutti i server.
        \item Input: nessuno 
        \item Output: un array JSON in cui ciascun oggetto rappresenta un aggiornamento.
        Per ciascun item dell'array i campi sono la data di aggiornamento  e la ragione
        (``\texttt{refresh}'' o ``\texttt{revocation}'' di un client). 
    \end{itemize}

    \item[Far ripartire un job]\texttt{PUT /<work-id>}
    \begin{itemize}
        \item Descrizione: vi sono molte ragioni per cui un work di trasferimento può
        terminare con un errore, tra cui un problema di connettività. Per situazioni
        come queste, in cui per problemi che poi sono stati risolti non è stato
        possibile completare un trasferimento, si è pensata questa API.
        Dopo aver verificato che un trasferimento non è andato a buon fine, e dopo
        aver risolto l'errore (se possibile), si può far ripartire il job. Esso
        sarà rimesso sulla coda come lavoro più recente.
        \item Input: nessuno
        \item Output: nessuno
    \end{itemize}

    \item[Elenco job]\texttt{GET /all-works}
    \begin{itemize}
        \item Descrizione: oltre a fare del polling per ogni singolo job, può essere
        più veloce utilizzare questa API per vedere l'elenco di tutti i job in corso,
        compresi quelli terminati di cui non si è ancora verificato lo stato,
        quelli in corso, quelli conclusi con errore.
        \item Input: nessuno
        \item Output: un array JSON contenente oggetti ciascuno nell'esatto
        formato ritornato dalle API relative a vedere lo stato di un job.
        Tutti i job completati e ritornati sono rimossi, pertanto non sarà
        più possibile vedere il loro stato in un secondo momento.
    \end{itemize}
\end{description}