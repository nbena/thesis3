\section{Conclusioni}
In conclusione del capitolo si fornisce un riassunto di pro e contro delle principali
tecnologie.
Nonostante WireGuard non sia stato analizzato
in dettaglio, viene comunque inserito in questa lista grazie ai suoi vantaggi.
\begin{description}
  \item[\textbf{OpenVPN}]
  \begin{itemize}
    \item ampio supporto e documentazione
    \item sicurezza basata su TLS e protocollo custom assimilabile ad esso
    \item vari tipi di autenticazione
    \item possibile connessione anche al layer 2
    \item topologie consigliate:
    \begin{enumerate}
      \item \textit{LS} con client in \textit{NAT al contrario}
    \end{enumerate}
    \item Contro:
    \begin{itemize}
      \item utilizzo di TLS su protocollo custom, quindi il traffico non riesce
      ad essere \textit{spacciato} completamente per HTTPS
      \item necessaria gestione dei certificati
    \end{itemize}
  \end{itemize}
    \item[\textbf{SoftEther}]
    \begin{itemize}
      \item multi-protocol e multi-platform
      \item incapsulamento in HTTPS
      \item possibile connessione anche al layer 2;
      \begin{itemize}
        \item bypassa la maggior parte dei firewall
      \end{itemize}
      \item NAT Traversal del server
      \item Dynamic DNS
      \item VPN over ICMP e over DNS
      \item topologia consigliate:
      \begin{enumerate}
        \item \textit{RS} senza bisogno di abilitare port forwarding sfruttando
        NAT Traversal
        \item \textit{LS}
      \end{enumerate}
      \item Contro:
      \begin{itemize}
        \item necessità di una seconda NIC sul server in modalità promiscua
        \item documentazione buona per Windows, meno per Linux
      \end{itemize}
    \end{itemize}
    \item[\textbf{IPsec}]
    \begin{itemize}
      \item IKEv2 dà ampia scelta di configurazioni
      \item sicurezza mediante ESP (cifratura payload pacchetto IP)
      \item NAT Traversal
      \item topologie consigliate:
      \begin{enumerate}
        \item \textit{LS}
        \item \textit{RS}
      \end{enumerate}
      \item Contro:
      \begin{itemize}
        \item setup non facile
        \item necessaria gestione dei certificati
        \item non passa tutti i firewall poiché si tratta di un nuovo protocollo
        che deve essere abiltato
        \item voci di indebolimento da parte dell'NSA
      \end{itemize}
    \end{itemize}
    \item[\textbf{WireGuard}]
    \begin{itemize}
      \item molto più performante di ogni altra soluzione
      \item altamente sicuro
      \item keep-alive per funzionare con NAT
      \item Contro:
      \begin{itemize}
        \item il sito stesso non lo definisce \textit{production ready};
        \item necessità di scambiare chiavi a priori
        \item traffico UDP potrebbe essere bloccato
      \end{itemize}
    \end{itemize}
\end{description}
La soluzione SoftEther con HTTPS è sembrata essere quasi perfetta, se il NAT Traversal
funzionasse come dichiarato si sarebbe potuta creare una topologia \textit{RS}.
Per firewall molto stringenti sono sembrate molto interessanti le possibilità di incapsulamento
in ICMP  e DNS, sebbene dichiarate come non completamente affidabili.

L'opzione IPsec con IKEv2 è altrettanto valida, il principale svantaggio è che i firewall più stringenti
potrebbero bloccare il traffico \textit{ESP}. In tal caso si può incapsulare \textit{ESP} in \textit{UDP},
ma comunque si richiede che il traffico UDP sia permesso (porta 500).

Come prima soluzione si è scelta SoftEther, ma infine quella utilizzata è OpenVPN. Il
prossimo capitolo ne affronta la configurazione nel dettaglio.
