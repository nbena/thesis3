\section{Introduzione}
\textit{VPN} è l'acronimo di \textit{Virtual Private Network}, ed indica diverse
tecnologie e protocolli utilizzati per, sostanzialmente, connettere tra loro
diversi reti locali geograficamente non contigue, oppure per connettere singoli host
ad una rete,
od anche singoli host tra loro. Questa \textit{connessione} avviene a diversi livelli dello stack
ISO/OSI, principalmente al livello 2 od al livello 3.\\
Si possono realizzare diverse topologie:
\begin{description}
  \item[Remote Access]Si realizza questa configurazione quando si vuole un connettere
  uno o più host ad una rete. Un esempio classico è quello di dipendenti che connettono
  il proprio dispositivo (PC, tablet, smartphone, etc \ldots) alla rete aziendale
  quando non si trovano in sede, potendo così accedere alle risorse che normalmente
  avrebbero disponibili solo se si trovassero fisicamente nella sede fisica dell'azienda.
  \item[LAN-to-LAN]A volte ci si riferisce a questa topologia anche come \textit{Intranet
  VPN}. In questo caso, si utilizza una VPN per connettere tra di loro due o più
  reti che sono geograficamente dislocate in posti diverse. E' molto utile per aziende
  che hanno più sedi sul territorio (anche in diversi stati), e che desiderano connettere
  tra di loro le reti di ciascuna sede.
  \item[Extranet VPN]Si tratta di un caso speciale di \textit{Intranet VPN}, e si realizza
  quando quando l'accesso ad una certa rete viene ristretto, ovvero è possibile accedere
  solo ad alcune sue parti mediante VPN.
\end{description}
Ma cosa significa davvero \textit{connettere tra di loro più reti}? La domanda sorge
spontanea, poiché una volta che due reti sono in Internet, in qualche modo
esse sono (indirettamente) connesse tra loro. In Internet, una normale rete privata
raggiunge solo le risorse che altre reti hanno rese disponbili su Internet,
siano esse dei server web o dei server mail.
Semplificando, si potrebbe dire che le varie reti accedono alle varie risorse delle
altre \textit{a livello di protocolli applicativi}.\\
Nel caso delle VPN, \textit{connettere tra di loro più reti a diversi livelli dello
stack ISO/OSI (o TCP/IP)} significa realizzare un collegamento \textit{ad un livello
più basso di quello applicativo}. La maggior parte delle VPN realizza un collegamento
al livello 2 od al 3, e questo ha i seguenti effetti pratici sulle $n$ reti connesse
tra di loro in VPN:
\begin{itemize}
  \item livello 2: le reti si trovano all'interno di un unico spazio di indirizzi IP
  (con stesso indirizzo di rete), ed è \textit{come se fossero separate da uno switch}.
  \item Livello 3: le diverse reti sono ciascune in uno spazio di indirizzamento separato,
  ed è \textit{come se fossero separate da un router}.
\end{itemize}
Supponendo che vi siano due reti, $A$ e $B$, connesse tramite una VPN al livello 3,
un host della rete $A$ può accedere a \textit{tutte} le risorse della rete $B$,
non solo quelle che $B$ espone verso la rete Internet.


Si distinguono tre tipologie di VPN:
\begin{description}
  \item[Secure VPN]Una VPN realizzata utilizzando crittografia dei dati in transito
  tra i diversi host/reti coinvolte, \textit{secure} poiché l'utilizzo di algoritmi
  crittografici (ammesso che siano utilizzati correttamente) garantiscono che nessun
  attaccante che intercetti tale traffico sia in grado di decifrarlo o alterarlo in
  qualsiasi modo.
  \item[Trusted VPN]Sono fornite dagli ISP, i quali garantiscono che nessun'altro
  cliente sia sullo stesso circuito VPN, \textit{trusted} poiché ci si fida di
  questa garanzia, inoltre, non vengono fornite particolari proprietà di sicurezza
  mediante protocolli crittografici.
  \item[Hybrid VPN]Termine che indica una VPN composta da tratti \textit{Secure}
  e da tratti \textit{Trusted}.
\end{description}
Di seguito, ci si concentrerà esclsuivamente sulla prima tipologia, poiché essa è
quella di interesse per MoonCloud.


\section{Anatomia}
Nell'ambito delle VPN del primo tipo, un pricipio base su cui si fonda il loro
funzionamento è quello dell'\textit{incapsulamento}. Si tratta di un termine
ben noto nell'ambito dei protocolli di rete, che può essere sintetizzato come
di seguito. Siano $A, B$ due protocolli di rete, con $A$ che si trova ad un livello
più basso rispetto ad $B$ (quindi più vicino al \textit{fondo} della pila
protocollare). Entrambi i protocolli hanno un \textit{header} ed un
\textit{payload}.\\
Si dice che \textit{B incapsula A} quando il payload di $B$ è costituito dall'intero
pacchetto di $A$, come mostrato in figura \ref{fig:encaps}.
E' importante notare che l'incapsulamento può essere ulteriormente generalizzato,
non è richiesto che $A$ sia di un livello più basso rispetto a $B$. Nello pila
di protocolli di rete, sia essa ISO/OSI o TCP/IP, l'incapsulamento è il modo di
procedere normale, per cui ad esempio un segmento TCP ha nel proprio payload
un intero pacchetto IP, il cui a sua volta ha come payload un intero frame Ethernet.


Di seguito si procede a dare una descrizione del funzionamento della maggior parte
delle VPN secure, iniziando con il definire con \textit{protocollo VPN}
il procollo utilizzato nel collegamento tra le reti/host.
Generalmente si distingue tra VPN client e VPN server, sebbene
una volta stabilito il collegamento, il protocollo VPN sia spesso peer-to-peer (un
pò come nel protocollo TLS, nella fase di \textit{handshake} vi è una chiara distinzione
tra client e server, ma una volta stabilita una connessione le due parti sono di fatto
paritetiche); in ogni caso vi è sempre un host che inizia la connessione.\\
Il VPN server può ricevere più connessioni dai VPN client, i quali in quanto tali
iniziano la connessione verso il server. A seconda della topologia/tecnologia,
i client possono essere responsabili di connettere alla rete del server la rete
a cui essi appartengono; viceversa, il server \textit{può} rendere visibile ai client
la rete cui appartiene. Altrettanto opzionalmente,
il server può connettere tra di loro i diversi client e le loro reti (nel senso che
i client possono comunicare tra loro anziché solo con il server/con la rete del server).\\
Per poter effettivamente collegare tra loro più \textit{reti}, il VPN \textit{peer}
deve avere i seguenti \textit{punti di contatto}:
\begin{itemize}
  \item collegamento con l'altro peer, raggiungibile tramite la rete Internet;
  \item punto di ricezione da cui riceve pacchetti provenienti dagli host della
  rete in cui esso si trova e diretti alla rete dell'altro peer;
  \item punto di invio al quale l'host invia i pacchetti provenienti dalla rete dietro
  la VPN
  e diretti ad host della propria.
\end{itemize}
Il collegamento di cui al primo punto si realizza spesso (ma non sempre) con un socket,
mentre gli ultimi due punti si realizzano con una \textit{scheda di rete virtuale}.\\
Nel collegamento lungo il socket, la tecnologia VPN definisce un protocollo (che ovviamente
garantisca proprità di sicurezza), può essere un protocollo già esistente come TLS,
oppure uno sviluppato ad hoc.\\
Una scheda di rete virtuale (o virtual NIC -- Network Interface Card) è una
scheda di rete che esiste nel kernel del sistema
operativo ma che non ha un corrispettivo fisico. Se ne può creare una al livello 2
od al livello 3 (utilizzando \texttt{TUN}, modulo del
kernel implementato in diversi sistemi operativi). Una scheda di rete
al livello 3 è dotata di un indirizzo IP proprio come ogni NIC reale.\\
Una scheda di rete virtuale è associata ad un software che compie operazioni su essa,
e, come per ogni scheda di rete, tali operazioni sono quelle di invio e ricezione.
\begin{description}
  \item[Inviare]Il software associato invia
  o funzione equivalente) un pacchetto sulla NIC, l'effetto è che tale
  pacchetto viene \textit{ricevuto} dal kernel del sistema operativo e processato
  come un qualsiasi altro pacchetto, pertanto l'OS deciderà
  dove e come inviarlo, e se applicare ulteriori trasformazioni.
  \item[Ricevere]Il sistema operativo riceve un pacchetto che ha per indirizzo destinazione
  quello delle scheda di rete virtuale, l'OS quindi inoltra il pacchetto alla NIC
  virtuale in questione, l'effetto è che il software associato riceve i dati
  che il sistema operativo ha inoltrato. Il pacchetto che viene inoltrato alla NIC
  può essere benissimo un pacchetto inviato da un altro host, e che quindi è arrivato
  al sistema operativo mediante una scheda di rete reale (oppure un'altra virtuale).
\end{description}
A livello di processo associato alla NIC, \textit{inviare un dato alla NIC} significa
chiamare la funzione \texttt{write()}, mentre \textit{ricevere} significa
utilizzare \texttt{read()} (o funzioni equivalenti).


L'idea di base è che ciò che il software VPN legge dalla scheda di rete virtuale
è ciò che è destinato ad un altro membro della VPN, e quindi venga incapsulato secondo
il protocollo VPN usato. Allo stesso modo, ciò che si riceve dal socket è destinato
alla proprio rete (salvo il caso si tratti di un server che connette più client), pertanto
viene scritto sulla NIC, per essere dato in gestione al proprio sistema operativo in
modo che possa inoltrarlo al destinatario.\\
Due note fondamentali che occorre sempre tenere presenti:
\begin{itemize}
  \item se si realizza una VPN al livello 3, tutte le reti partecipanti devono
  avere degli spazi di indirizzamento diversi.
  \item In una VPN al livello 2, tutte le reti partecipanti devono stare
  nella stessa rete IP.
\end{itemize}


Per capire meglio quanto spiegato, si procede con un esempio.
Si supponga ora di voler configurare una certa \textit{VPN X}, e che si voglia realizzare una
topologia LAN-to-LAN tra due rete $A, B$, in cui nella prima si trova il server VPN $X$,
nella seconda naturalmente si trova il client.  Lo scenario è il seguente:
\begin{itemize}
  \item rete $A$: indirizzo di rete: \texttt{192.168.1.0/24}
  \item rete $B$: indirizzo di rete: \texttt{192.168.10.0/24}
  \item indirizzo interno del server VPN in $A$: \texttt{192.168.1.200}
  \item indirizzo pubblico del server VPN: \texttt{2.7.200.70}
  \item indirizzo interno del client VPN: \texttt{192.168.10.20}
\end{itemize}
Per brevità, ci si riferirà al server con $s$, ed al client con $c$. Si definiscono
infine gli host \texttt{192.168.1.5} come $a_5$, e \texttt{192.168.10.5} in $b_5$, ed infine
si suppone che gli indirizzi IP
dei default gateway delle due reti finiscano in \texttt{.254}.\\
Per il momento non ci si concentra troppo sulla configurazione delle rotte, supponendo che i
pacchetti arrivino
agli host corretti. Si anticipa soltanto che in tutte le reti partecipanti alla VPN
(in questo caso $A$ e $B$), occorre configurare \textit{almeno una rotta}; nel capitolo in cui si descrivono le configurazioni di OpenVPN,
questo aspetto viene affrontato nel dettaglio.\\
Si vede quindi cosa succede quando $b_1$ vuole comunicare con $a_1$, posto che il
collegamento VPN tra $c$ ed $s$ sia già stato stabilito con successo. Si precisa che
si utilizza il termine generico \textit{pacchetto} per indicare un qualsiasi messaggio
di un qualsiasi protocollo di rete.
\begin{itemize}
  \item Il pacchetto da $b_1$ viene inviato, e quindi ricevuto da $c$.
  \item Il sistema operativo di $c$ invia il pacchetto alla scheda di rete virtuale
  della VPN.
  \item Il pacchetto originale viene incapsulato da $c$ in un nuovo pacchetto secondo
  il protocollo VPN, quindi inviato ad $s$ e da $s$ ricevuto.
  \item $s$ decifra (ed effettua verifiche di autenticità, integrità, ecc\ldots) il
  pacchetto, a questo punto il risultato della decifratura è un pacchetto esattamente
  uguale a quello generato al punto 1.
  \item $s$ scrive il pacchetto sulla propria scheda di rete virtuale.
  \item L'OS di $s$ riceve il pacchetto, lo invia a $a_1$.
\end{itemize}


Il protocollo di trasporto preferito è UDP in quanto introduce meno overhead rispetto al TCP. A causa
dei particolari requisiti di MoonCloud, ci si è concentrati invece su soluzioni che
supportassero il TCP, questo perché è possibile che nelle reti target UDP sia bloccato,
mentre è infattibili ritenere che TCP stesso sia bloccato. Probabilmente vi saranno
delle restrizioni, e nel caso in cui proprio la VPN non riesca a funzionare si può sempre
chiedere di aprire una porta sul firewall, l'importante è non essere troppo invasivi.


\section{Topologie}
In questa sezione si esaminano le diverse topologie realizzabili per connettere una rete
target a MoonCloud. Non tutte le tecnologie consentono di realizzare le topologie qui
dettagliate.
\begin{description}
  \item[\textit{LS - Local Server}]In questa configurazione si prevede di installare i
  VPN server in MoonCloud, mentre nelle reti target si porta (o si installa) il device
  VPN client, il quale è incaricato di connettersi al VPN server. Un server può essere
  potenzialmente responsabile per più reti client diverse. I container che fanno analisi
  (o l'host su cui sono in esecuzione)
  sono configurati per inoltrare i pacchetti al VPN server, il quale li invia alla rete
  target mediante il collegamento VPN.
  \item[\textit{RSMC - Remote Server Multi Client}]Opposta alla precedente, in \textit{RSMC}
  si installa un singolo VPN server nelle reti target. In una configurazione di tipo
  \textit{Multi Client}, ciascun Docker host è connesso direttamente in VPN con il server, il
  quale, una volta ricevuti i pacchetti dalla VPN li invia agli host target.
  \item[\textit{RSSC - Remote Server Single Client}]Simile ad \textit{RSMC}, il VPN
  server è presente nella rete target, tuttava in MoonCloud si realizza, per ciascuna rete
  target, un unico VPN client, a cui i Docker host responsabili per una certa rete target inviano
  i pacchetti per tale rete. Il client invia quindi in pacchetti lungo la VPN al server,
  che quindi provvede ad inoltrarli agli host nella rete target.
\end{description}
Nelle configurazioni di tipo \textit{Remote Server}, è necessario che il server sia
direttamente raggiungibile da MoonCloud, e quindi deve disporre di un indirizzo IP pubblico.
Si è supposto ragionevolmente che la rete target disponga di un collegamento ad Internet, e che
sia dietro router/firewall che esegua NAT. Come tale, la rete target è raggiungibile
mediante un indirizzo IP pubblico dinamico. Poiché vi è NAT, è fondamentale che il server disponga
di un qualche meccanismo di NAT Traversal, poiché si vuole evitare di dover configurare
port forwarding sul router del cliente.\\
Vi sono naturalmente casi in cui non c'è NAT, ma sono una minoranza.


Già in questa fase iniziale di studio avevo previsto l'utilizzo del \textit{NAT al contrario},
un termine coniato qui per indicare un particolare utilizzo del NAT. Si supponga di
essere nella situazione delle due reti elencate precedentemente, ma in cui non vi sia
possibilità di intervenire sulle rotte nella rete $A$ (non si sono ancora viste quali rotte,
ma si è detto che sono necessarie delle rotte introdotte sull'intera rete) perché essa è
la rete target, e quindi si vogliono limitare gli interventi in essa.
I pacchetti inoltrati da $s$ (nella rete $A$) verso la rete e provenienti da $B$, hanno
come indirizzo IP sorgente un indirizzo IP in $B$. Una volta che tali pacchetti
sono stati ricevuti da un host di $A$, esso non ha modo di sapere che le risposte devono
tornare ad $s$ (proprio perché non vi sono rotte configurate), e quindi invierebbe il
pacchetto al proprio default gateway, che quindi lo dropperebbe (trattandosi di una
destinazione con indirizzo IP privato).\\
Il \textit{NAT al contrario} consiste nell'applicare NAT sui pacchetti
da $s$ (o da $c$ a seconda se sia client o server nella rete target) provenienti dalla
VPN e destinati alla propria rete: in questo modo raggiungono l'host target con l'indirizzo IP
di $s$, che si trova nella stessa rete del target, e per tale ragione il target può
inviare ad $s$ le risposte senza passare dal default gateway.\\
Questo meccanismo è stato descritto in maniere molto sintetica, poiché è una soluzione che è stata
davvero applicata, viene analizzato molto più nel dettaglio nel capitolo dedicato alle configurazioni
di OpenVPN.\\
Tra queste topologie, si anticipa che quella scelta è \textit{LS}.


\section{Motivazioni}
TODO.

\section{Introduzione alle tecnologie}
La tabella \ref{tbl:vpn-comparison} riassume le principali tecnologie VPN. Nella
penultima colonna, con \textit{firewall stringenti} ci si riferisce a firewall che
lavorano a livello applicativo.
\begin{table}\label{tbl:vpn-comparison}
  \begin{tabular}{|p{3.3cm}|p{2.7cm}|p{3.1cm}|p{1.7cm}|p{3cm}|}
    \hline
    Technology & Procollo di trasporto & Protocollo incapsulato & Passa fw. stringenti & NAT Traversal\\
    \hline
    OpenVPN & TCP/UDP & Ethernet/IP & No & Sì, keepalive-based\\
    \hline
    IPsec (IKEv2) & IPsec & IP & No & Sì\\
    \hline
    SoftEther & HTTPS (anche ICMP, DNS) & Ethernet/IP & Sì & Sì, parzialmeente\\
    \hline
    L2TP/IPsec (IKEv1) & IPsec & IP & No & Sì (con SoftEther)\\
    \hline
    \hline
    SSTP & HTTPS & PPP (IP) & Sì & Sì (con SoftEther)\\
    \hline
    OpenConnect (Cisco Any Connect)& DTLS e HTTPS & IP & Sì con HTTPS & ?\\
    \hline
    OpenSSH & SSH & Ethernet/IP/TCP & No & ?\\
    \hline
    WireGuard & WireGuard & IP & No & Sì, keepalive-based \\
    \hline
    L2TPv3/IPsec & IPsec & Ethernet (e altri layer 2) & No & Sì (con IKEv2 o SoftEther)\\
    \hline
    EtherIP/IPsec & IPsec & Ethernet & No & Sì (con IKEv2 o SoftEther)\\
    \hline
    PPTP & GRE & PPP (IP and others) & No & No\\
    \hline
  \end{tabular}
  \caption[Tecnolgie per VPN]{Principali tecnologie per VPN.}
\end{table}

Prima di proseguire, occorre subito chiarire che il problema principale dell'usare una VPN
è che \textbf{spesso è necessario configurare almeno un firewall} o default gateway o router di
confine. Per questo motivo
si cercherà, nei limiti del possibile, di valutare quali opzioni sono più \textit{soft}
da questo punto di vista.\\
Con le VPN si possono realizzare due topologie:
\begin{itemize}
  \item \textbf{Remote Access}: in questa configurazione si connette un host ad una LAN.
  Sulla LAN sarà necessario installare il componente server, sul PC il componente client
  dell'infrastruttura VPN\footnote{Nella maggior parte delle tecnologie VPN vi è una distinzione
  tra client e server, anche nei casi \textit{peer-to-peer} come IPsec.}.
  \item \textbf{LAN-to-LAN}: si realizza una connessione tra due reti locali, in modo che gli
  host di ciascuna rete possano connettersi agli host dell'altra, e viceversa (a livello 2
  o a livello 3). Per fare ciò, su un host della rete si installa il server, su un host
  dell'altra il client oppure un altro componente specifico. Attuando le opportune configurazioni, i due host sono responsabili
  di instradare tutto il traffico della propria rete verso l'altro. Il problema, come si vedrà
  più avanti, è che per far sì che tutto funzioni occorre aggiungere delle rotte in entrambe le
  reti.
\end{itemize}
Esiste anche \textbf{extranet}, che corrisponde a Remote Access o LAN-to-LAN con accesso ristretto alle
risorse della rete remota. Visto lo scopo di questo documento, non ci si sofferma su questa differenza.
Da ora in poi si userà il termine \textit{rete target} per indicare la rete che MoonCloud analizza.\\
Le due topologie si traducono nel seguente modo per l'architettuta MoonCloud:
\begin{itemize}
  \item Remote Access: in questo caso i container diventano client del server VPN installato nella rete
  target\footnote{Si tenga presente che è
  possibile \textit{invertire} questo paradigma, ed avere il server in MoonCloud.}.
  \item LAN-to-LAN: i container che fanno analisi e la rete target sono connessi in un'unica rete.
  Sono possibili soluzioni per cui il server si installa su MoonCloud oppure nella rete target.
\end{itemize}
Ciascuna di queste due strade ha vantaggi e svantaggi, che verranno dettagliate tecnologia per tecnologia.\\
Ora si passa a valutare le singole tecnologie VPN. Per ciascuna di esse si fornisce una overview
generale, si elencano i passi necessari per raggiungere una certa configurazione
(in modo da avere un'idea della complessità), si valutano le diverse topolgie posssibili. Si conclude
infine suggerendo quali sono le opzioni migliori.


Un'ultima nota di proseguire: durante tutto questo capitolo ed il prossimo, si utilizzerà
il termine \textit{pacchetto} per indicare una generica sequenza di byte di un generico
protocollo, presente in un certo momento in un host dopo essere stata ricevuta o prima
di essere inviata, oppure in viaggio sulla rete. Sono perfettamente consapevole che per
ogni livello dello stack ISO/OSI vi sia un termine specifico per indicare tale
sequenza, ad esempio \textit{frame} per il livello 2 con Ethernet, oppure \textit{datagrammi}
per il protocollo UDP, e così via. Tuttavia, come si è visto dalla tabella \ref{tbl:vpn-comparison},
le VPN possono funzionare su diversi protocolli di trasporto (non solo al livello 4),
e possono altresì incapsulare
pacchetti di livelli divrersi. Per questa ragione ho scelto di utilizzare
\textit{indiscriminatamente} il termine \textit{pacchetto} per riferirmi ad essi.


% TODO check if I already said this
Tra le soluzioni indicate di seguito, la prima scelta è stata SoftEther. Per motivi che
verranno spiegati in seguito, la soluzione definitiva è poi OpenVPN.
