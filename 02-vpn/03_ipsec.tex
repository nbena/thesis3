\section{IPsec}
\subsection{Overview}
IPsec è una suite di protocolli usata per rendere sicuro il livello 3 dello stack di
rete. Mediante crittografia si realizza una connessione che garantisce
numerose proprietà di sicurezza. Due protocolli disponibili:
\begin{itemize}
  \item \textit{AH - Authentication Header}: integrità ed autenticazione (no IP spoofing) del pacchetto con HMAC.
  E' fondamentale comprendere che questa soluzione non è compatibile con NAT.
  \item \textit{ESP - Encapsulating Payload}: confidenzialità del pacchetto mediante cifratura simmetrica.
  Opzionalmente in \textit{ESP} si può abilitare direttamente l'autenticazione e l'integrità,
  si consiglia questa strada.
\end{itemize}
Si può abilitare anche la protezione da replay attacks.\\

Il terzo membro della suite IPsec è \textit{IKE -- Internet Key Exchange} che viene
utilizzato per negoziare i parametri per i due protocolli precedentemente citati.

IPsec (sia \textit{AH} o \textit{ESP}) può funzionare in due modalità:
\begin{itemize}
  \item \textit{Transport Mode}: aggiunta di un nuovo header (AH o ESP) tra l'header IP
  ed il payload.
  Il suo uso tipico è quello in comunicazioni \textit{dirette} tra due host, che non si affidano a
  due gateway di confine per eseguire IPsec.
  \item \textit{Tunnel Mode}: si crea un nuovo pacchetto. Esso è costituito da
  un nuovo header IP, dall'header IPsec (AH o ESP), e quindi dal payload di IPsec, costituito dal
  vecchio header IP e dal suo payload.
\end{itemize}

Per ciò che concerne IKE, è un protocollo abbastanza complesso. Brevemente, la negoziazione
dei parametri avviene in due fasi:
\begin{itemize}
  \item nella \textit{phase 1} viene negoziata una \textit{Security Association} (set di parametri)
  bidirezionale, utilizzata nella
  \item \textit{phase 2}; in questa fase si utilizza la SA precedente per negoziare i
  parametri per la sessione IPsec, tra essi vi sono le chiavi di sessione (stabilite con
  uno scambio Diffie-Hellman).
\end{itemize}
Attualmente, è disponibile la versione 2 di IKE. Si tratta di un upgrade importante sotto molteplici punti
di vista, ed è fortemente consigliata una implementazione di IPsec che supporti IKEv2.
La porta standard utilizzata è la 500 UDP.


L2TP (Layer 2 Forwarding Protocol) è un protocollo standard il cui scopo è quello
di realizzare una connessione punto-punto, per cui non include vere funzionalità
di sicurezza. L'RFC 3193 standardizza l'uso combinato di IPsec per creare una
connessione sicura, sulla quale si negoziano
i parametri L2TP \cite{RFC3193}.


Per realizzare VPN con IPsec tipicamente si utilizzano due configurazioni:
\begin{itemize}
  \item L2TP on IPsec (negoziazione di L2TP su un canale IPsec sicuro--\textit{ESP Transport Mode}
  \item IPsec with IKEv2.
\end{itemize}


Lo stack tipico di IPsec su Linux è il seguente:
\begin{itemize}
  \item ESP ed AH implementati nel kernel, per cui molto veloci
  \item demone IKE in user-space, negozia le policy per la sessione e le
  passa al kernel.
\end{itemize}
L'implementazione IKE di riferimento per Linux, sebbene sia disponibile anche
cross-platform, è \texttt{strongSwan} \cite{strongswan}. Maggiori dettagli su essa sono descritti
tra qualche riga.


E' possibile realizzare configurazioni Remote Access e LAN-to-LAN. IPsec è uno stack
complesso, e configurarlo nella maniera corretta non è facile, tuttavia non è nemmeno
impossibile, il sito di strongSwan è ricco di esempi e di documentazione \cite{strongswan-example}.
Nella terminologia IPsec con strongSwan si definiscono i seguenti termini:
\begin{itemize}
  \item \textit{roadwarrior} è il client, cioè il dispositivo che si connette da remoto alla rete.
  \item \textit{VPN Gateway} è il dispositivo server che riceve le connessioni.
\end{itemize}
Il client è l'\textit{initiator},
il server è il \textit{responder}.

Da subito ci si è concentrati su un setup basato su IKEv2, poiché si tratta della
versione più recente del protocollo. Non si è considerato IPsec con L2TP
visto che utilizza tipicamente IKEv1, ed è considerato \textit{legacy} \cite{nordvpn}.

Utilizzando strongSwan sono possibili moltissime forme di autenticazione, distinguendo tra:
\begin{itemize}
  \item autenticazione degli host, può essere fatta con PSK (fortemente sconsigliato) o certificati X509
  \item autenticazione degli utenti, in questo caso è possibile usare anche un
  server RADIUS.
\end{itemize}
Si tenga presente che l'autenticazione degli utenti è fatta \textit{on top} dell'autenticazione
degli host.
Per ciò che concerne un'autenticazione basata su certificati, occorre seguite gli stessi
fatti già descritti per OpenVPN, ovvero creare chiavi/certificato per la CA, quindi generare
certificati con essa, ecc\ldots


Come per la maggior parte dei software Linux, la configurazione di strongSwan viene
fatta mediante file di configurazione.
E' di particolare rilevanza configurare correttamente \textit{quali} indirizzi IP vengono
incapsulati nel tunnel IPsec, perché le policy presenti nel kernel descrivono anche questo,
e se un pacchetto ha sorgente o destinazione IP non corretti sarà droppato dal kernel.

strongSwan è una soluzione molto interessante e completa, alcune caratteristiche
che meritano una menzione sono relative ad alcuni algoritmi crittografici supportati:
\begin{itemize}
  \item supporto a certificati \texttt{Ed25519} (istanza dell'algoritmo di firma digitale
  \textit{EdDSA} basato su \textit{Twisted Edward Curves},
  particolarmente interessante per numerosi motivi tra cui le
  performance e la facilità di implementazione)
  \item supporto ad alcuni algoritmi a chiave pubblica post-quantistici (basati su
  problemi matematici per i quali i computer quantistici non portano alcun vantaggio
  nella loro soluzione) come \texttt{NTRU} (\textit{lattice-based cryptography})
\end{itemize}
strongSwan offre NAT Traversal basato su UDP.


\subsection{IPsec e MoonCloud}
In IPsec, una volta
che le policy sono state stabilite, non vi è davvero distinzione tra chi sia client e server.
Si distingue solamente nelle topologie che si realizzano, ed in base a dove si posiziona il client cioè
l'\textit{initiator}.
Sono quindi possibili sia configurazioni \textit{RS} ed \textit{LS}, un server nella rete target
potrebbe sfruttare il NAT Traversal che strongSwan offre, occorrerebbe valutarne l'efficacia.
La topologia \textit{LS} è sempre possibile.


\subsection{Conclusioni}
Si è consigliata una topologia \textit{RS}, posto che il NAT Traversal consente
la raggiungibilità dall'esterno, in caso contrario \textit{LS}.
Vi sono diverse voci fondate dalle dichiarazioni di Snowden, secondo cui l'\textit{NSA avrebbe volontariamente
indebolito il protocollo}\ldots\footnote{La bibliografia in questo è il corso di Sicurezza delle Reti.}
Altre voci invece riportano di come l'NSA sia in grado di rompere sessioni IPsec
configurate con PSK (\cite{ipsec-nsa}).
Si rilevano infine due problemi principali:
\begin{itemize}
  \item complessità: IPsec è costituito da numerosi protocolli e non è generalmente considerato
  facile da configurare, in particolar modo occorre prestare attenzione alle policy che specificano
  quali indirizzi IP possono essere incapsulati
  \item supporto ad UDP: è indispensabile che il traffico UDP sia abilitato perché IKE funziona
  su tale protocollo.
\end{itemize}

