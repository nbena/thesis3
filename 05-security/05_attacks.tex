\section{Scenari di attacco}\label{sec:attacks}

In questa sezione si discutono alcuni scenari di attacco
che sfruttano la VPN.
Per ciascuno di essi, si utilizza la seguente struttura:
\begin{description}
    \item[Scenario]Una breve descrizione dello scenario.
    \item[Livello di difficoltà]Lo sforzo richiesto all'attaccante
    per portare a termine l'attacco, di cosa deve disporre per avere
    successo.
    \item[In caso di successo]Che cosa otterrebbe l'attaccante
    in caso di successo.
    \item[Criticità]Quanto è grave l'attacco.
    \item[Prevenzione]Le principali 
    per mitigare la possibilità che l'attacco si verifichi.
    \item[Ulteriori misure preventive]Ulteriori misure
    di prevenzione secondaria attuate e attuabili.
    \item[Cosa fare in caso di attacco]Nel caso in cui
    l'attacco si verifichi, quali contromisure occorre
    adottare. Oltre a quelle indicate, è sempre necessario
    verificare \textit{come mai} l'attacco si sia verificato
    (ovvero: \textit{come ha fatto l'attaccante a riuscirci?}).
\end{description}
Poiché allo stato delle cose non è ancora stato approntato come
sarà effettivamente realizzato il VPN client (es: quale versione
di Linux? Quali utenti? Quali interfacce? Ecc\ldots), è prematuro
analizzare le contromisure da attuare sui client.
Inoltre, è comunque possibile che l'attaccante usi un proprio
dispositivo client, su cui ovviamente MoonCloud non ha controllo.

E' molto importante notare che la protezione data dalle regole di
firewalling è molto efficace, anche nei casi di attacchi con livello
di criticità alto.


% \begin{description}
%     \item[Scenario]
%     \item[Livello di difficoltà]
%     \item[In caso di successo]
%     \item[Criticità]
%     \item[Prevenzione]
%     \item[Ulteriori misure preventive]
%     \item[Cosa fare in caso di attacco]
% \end{description}


\subsection{Scenario 1}
\begin{description}
    \item[Scenario]Inviare pacchetti malevoli alla rete MoonCloud
    mediante un VPN client.
    \item[Livello di difficoltà]Medio. Si suppone che l'attaccante abbia
    ha disposizione un VPN client legittimo, ovvero con un certificato
    valido. E' possibile che l'attaccante sia riuscito ad ottenre un
    VPN client \textit{ufficiale}, o che sia riuscito ad ottenere
    da esso
    un certificato valido.
    \item[In caso di successo]L'attaccante riuscirebbe a connettersi
    ai server VPN, ma le sue possibilità si fermerebbero lì grazie
    alle contromisure.
    \item[Criticità]Media, la rete MoonCloud è intatta e la vulnerabilità
    è eventualmente sul client VPN.
    \item[Prevenzione]Grazie alle regole di firewalling sui server,
    se anche l'attaccante ottenesse dei certificati validi, non
    potrebbe \textit{in ogni caso} inviare pacchetti a MoonCloud, poiché
    sono consentite solo risposte a richieste dalle Docker machine. 
    \item[Ulteriori misure preventive]E' possibile revocare il certificato
    usato dall'attaccante, limitando del tutto la sua possibilità di azione.
    \item[Cosa fare in caso di attacco]Il servizio  \texttt{MoonCloud\_VPN}
    offre una API dedicata alla revoca dei certificati, questo è il primo
    passo da seguire.
\end{description}

\subsection{Scenario 2}
\begin{description}
    \item[Attaccante]ciao
\end{description}

\subsection{Scenario 3}
L'attaccante in questo caso può disporre di un VPN client valido,
cioè dispone di certificato valido che sarà accettato dai server
VPN.
L'attaccante riuscirebbe effettivamente a connettersi al server,
tuttavia non potrebbe fare altro: infatti, grazie alle regole
di filtraggio sui server, \textit{passano solo le risposte
alle richieste di MoonCloud}.
Ulteriori azioni mitigatorie: revocare il certificato.
% \begin{description}
%     \item[Ulteriori requisiti]L'attaccante deve anche sapere:
%     \begin{itemize}
%         \item 
%     \end{itemize}
% \end{description}

\subsection{Scenario 4}
\begin{description}
    \item[Scenario]Compromissione della chiave private della CA
    \item[Livello di difficoltà]Alto: la chiave privata della CA 
    non è mai distribuita al di fuori della rete MoonCloud, tantomeno
    viene distribuita ai VPN server.
    \item[In caso di successo]Un attaccente potrebbe generare dei nuovi
    certificati validi per client VPN. Egli potrebbe quindi connettersi
    senza alcuna difficoltà ai VPN server.
    \item[Criticità]Molto alta, l'attaccante è per forza di cose riuscito 
    ad introdursi nella rete MoonCloud ed aver ottenuto la chiave privata. 
    \item[Prevenzione]La chiave privata è utilizzata \textit{esclusivamente}
    dal microservizio \texttt{MoonCloud\_VPN}, e sono in approntamento
    misure di protezione basate su firewall per far sì che
    solo richieste legittime arrivano al servizio.
    \item[Ulteriori misure preventive]Anche se l'avversario riuscisse a connettersi,
    non potrebbe
    comunque modificare le regole di firewalling che consentono
    alle sole risposte di transitare dai VPN server verso la rete MoonCloud.
    Per limitare ulteriormente lo spazio di manovra dell'attaccante
    si può usare l'opzione \texttt{tls-verify} per una sorta di \textit{abilitazione
    esplicita}.
    \item[Cosa fare in caso di attacco]E' chiaro che la compromissione della
    chiave privata sia uno scenario estremamente grave, pertanto vi sono
    numerose contromisure attuabili per limitare i danni. Tuttavia, in
    caso in cui questo si verifichi, occorre spegnare i server VPN, e
    chiaramente generare delle nuove chiavi.
\end{description}