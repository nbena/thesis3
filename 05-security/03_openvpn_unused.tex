\subsection{Opzioni non ancora utilizzate}
In questa sottosezione si mostrano alcune configurazioni che OpenVPN offre che
non sono state utilizzate. Per ciascuna di esse vale la seguente considerazione:
in futuro potranno essere adottate.

\subsubsection{\texttt{tls-crypt} e \texttt{tls-auth}} Si tratta di ulteriori
misure di sicurezza:
\begin{itemize}
  \item \texttt{tls-crypt}: il \texttt{Control Channel} viene cifrato ulteriormente
  con una chiave precondivisa nota a tutti i partecipanti.
  \item \texttt{tls-auth}: il \texttt{Control Channel} viene protetto da un ulteriore
  layer di MAC usando una chiave statica che deve essere nota a tutti i partecipanti a priori.
\end{itemize}
Lo scopo principale è quello di mitigare attacchi DoS: se il server riceve un pacchetto con HMAC non valido,
  o che non riesce a decifrare, immediatamente droppa la connessione, anziché iniziare
  il TLS handshake (che eventualmente fallirebbe nel caso l'attaccante non avesse
  nessun certificato valido).
Se questa chiave precondivisa fosse compromessa rimarrebbero comunque i certificati
come forma effettiva di protezione, infatti i certificati sono e rimangono
l'effettiva forma di autenticazione.

Si è scelto di non adottare questa soluzione per ora a causa della necessità di dover gestire
in modo sicura la chiave precondivisa: essa deve essere nota a tutti i partecipanti
alla VPN. Nel caso in cui essa fosse compromessa occorrerebbe cambiare \textit{tutte}
le chiavi, poiché sebbene rimanga la protezione dei certificati, una chiave
compromessa non garantirebbe nessun livello di protezione, per cui appunto, bisognerebbe
generarne una nuova.
In futuro si parla di usare
chiavi diversi per ogni coppia client-server, rimuovendo quindi la necessità di sostituire
\textit{tutte} le chiavi in caso di compromissione.
Non si esclude che a quel punto tale soluzione verrà adottata.

A puro scopo di documentazione, il comando per generare una chiave valida per
\texttt{tls-auth} o \texttt{tls-crypt} e scriverla in un file chiamato \texttt{tls\_crypt.key}
è il seguente:
\begin{minted}{bash}
openvpn --genkey --secret tls_crypt.key
\end{minted}
Il contenuto di tale file è simile al seguente:
\begin{verbatim}
  #
  # 2048 bit OpenVPN static key
  #
  -----BEGIN OpenVPN Static key V1-----
  54d84d5baa8534de608864311958bc82
  26d630157c0f7c03306d83f52eda7b6e
  bd5b95cfb9f107e1e968fb14d72ba515
  81f9fdb74473eea7e68682483c026170
  532994275e0c81a07f23231aeefa816a
  837dbd8ccba4fbbac1fd281dc986e403
  cb6486b57066b96790bae3b1d102757d
  3c09b96c0f21f8e5d5836301b41aab11
  d55c4167110952971fd698fc89c03c5f
  897c487adb18149d892668c0c54f1fe2
  6a65c112acbbd4f00977993d073e34ae
  140bc7159da6a9a7c5075b6baf6dcf26
  545cfe36478ed731cb53e5a3a36bdcb0
  3f92ecec02d8060ff3afb3b58c560ae5
  48f0a13a443cebe31431f8234586c0f9
  20c5f523330dc5b0ff5e4c9478bd2357
  -----END OpenVPN Static key V1-----
\end{verbatim}


\subsubsection{\texttt{tls-verify}}
\texttt{tls-verify} è un hook che scatta quando un client ha appena concluso il TLS
handshake ed ha passato tutte le verifiche, tranne, eventualmente, quella della CRL.
Il lanciato con questo hook ha accesso ad un certo numero di variabili d'ambiente
(minori rispetto a \texttt{client-connect} perché non è stato ancora completamente
\textit{accettato}). 

Si può usare questo hook per una sorta di \textit{abilitazione esplicita} dei client,
per la quale dei dati relativi al nuovo client (come ad esempio il seriale), devono
essere \textit{trasmessi} al server per poter accettarlo.
Si noti che comunque
affinché un client possa completare il TLS handshake deve avere un certificato valido
creato dalla CA in questione.

A puro titolo esemplificativo, si presente uno script in Python che
verifica che il seriale settato al lancio da OpenVPN sia in una lista di seriali noti,
in caso positivo ritorna \texttt{0}, sennò \texttt{-1}.
\begin{minted}{python}
#!/usr/bin/env python3

import os

# lista dei seriali ammessi
clients = ['3','4']
  
returned_value = -1

# accesso alla variabile d'ambiente
serial = os.getenv('tls_serial_0', None)

# se la variabile d'ambiente non è definita ritorna 0,
# perché vi sono casi 'leciti' in cui
# tale variabile non sia settata, cioè quando si stanno
# verificano tutti i certificati superiori nella
# gerarchia a quello del client in questione.
if serial is None or (serial is not None and serial in clients):
  returned_value = 0
  
exit(returned_value)   
\end{minted}
Per abilitare l'uso dello script:
\begin{minted}{bash}
  tls-verify /etc/openvpn/server/cs/tls_verify.py
\end{minted}

