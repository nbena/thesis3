\section{Sicurezza di OpenVPN}
\subsection{Protocollo}
OpenVPN crea due canali di comunicazione tra client e server:
\begin{itemize}
  \item \texttt{Control Channel} protetto con TLS ed usato per scambiarsi informazioni
  \textit{di servizo}
  \item \texttt{Data Channel} utilizzato per l'effettivo scambio dei dati
\end{itemize}
Entrambi i due canali sono multiplexati su una singola connessione TCP o UDP.
Poiché il protocollo TLS è stato progettato per funzionare su un livello di trasporto affidabile,
questa affidabilità viene offerta da OpenVPN quando il protocollo di trasporto è UDP.

La sicurezza di OpenVPN può essere riassunto dalla seguente frase:
\say{Use the IPSec ESP protocol for tunnel packet security,
but then drop IKE in favor of SSL/TLS for session authentication}\cite{openvpn-security-faq}.\\
Di fatto, IPsec ESP qui corrisponde al \texttt{Data Channel}.\\
% TLS è stato analizzato e studiato in dettaglio in tutti questi anni ed io non intendo
% aggiungere niente di nuovo rispetto alle analisi già fatte. La versione attuale
% di TLS è la 1.3, tuttavia quest'ultima versione non è completamente supportata
% nelle varie libreria SSL/TLS, in particolare non è ancora supportata dalla versione
% di \texttt{OpenSSL} utilizzata in OpenVPN. Le altre librerie supportate sono:
% \begin{itemize}
%   \item \texttt{LibreSSL} (supportata non in maniera ufficiale)
%   \item \texttt{mbedTLS}
% \end{itemize}
TLS utilizza uno scambio di chiavi Diffie-Hellman (è possibile anche una scambio
basato su RSA), che se usato propriamente garantisce la proprietà di \textit{Perfect Forward Secrecy}, ciò
significa che se anche la chiave privata contenuta nel certificato fosse compromessa
ed un avversario avesse registrato tutto il traffico comunque non riuscirebbe a decifrarlo,
poiché le chiavi utilizzate per Diffie-Hellman vengono generate \textit{al volo} ad
ogni connessione, e quindi anche se l'avversario potesse vedere le chiavi pubbliche
Diffie-Hellman, comunque dovrebbe essere in grado di rompere tale algoritmo, il che
non è assolutamente fattibile.\\
Le chiavi negoziate sono unidirezionali.\\
TLS offre diverse \textit{cipher-suite}, in particolare quelle considerate migliori
sono basate su uno scambio di chiavi Diffie-Hellman effimero su curva ellittica
(\textit{ECDHE}) ed un cifrario \textit{AEAD - Authenticated Encryption with Associated Data}.
Diffie-Hellman effimero garantisce la proprietà di \textit{Perfect Forward Secrecy}.\\
\textit{Authenticated Encryption with Associated Data} è una proprietà di sicurezza
che garantisce confidenzialità, integrità ed autenticazione dei dati, oltre
a autenticazione ed integrità di ulteriori dati \textit{associati}, come ad esempio
metadati relativi al protocollo che non devono essere anche essere segreti.
Esistono algoritmi di cifratura che utilizzati nella maniera opportuna, tipicamente in combinazione
con una funzione di MAC, garantiscono questa proprietà con un'unica chiave simmetrica
(anziché con due chiavi, una per la cifratura ed una per il MAC).
% Tutti gli algoritmi
% utilizzati in TLS 1.3 garantiscono questa proprietà, essi sono:
% possibili, gli unici algoritmi ammessi in TLS 1.3 sono:
% \begin{itemize}
%   \item \texttt{AES-CCM}, ovvero AES in modalità \texttt{CTR} combinato con CBC-MAC
%   \item \texttt{AES-GCM} cioè AES in modalità \texttt{CTR} combinata con una funzione di hash nota come
%   \texttt{GHASH}
%   \item \texttt{ChaCha20-Poly1305} che è la combinazione del cifrario a stream \textit{ChaCha20}
%   con l'algoritmo di MAC \textit{Poly1305}
% \end{itemize}


Una volta completato il TLS handshake, i due partecipanti alla VPN
generano del materiale pseudo casuale e lo trasmettono sul \texttt{Control Channel}: tale
materiale è utilizzato per le chiavi nel \texttt{Data Channel}. Il \texttt{Data Channel},
proprio come TLS, garantisce \textit{Authenticated Encryption}.


OpenVPN non ha pubblicato una specifica formale del proprio protocollo, tuttavia
esistono in letteratura diverse analisi che hanno provato a ricostruire, con buon
successo, il suo funzionamento mediante \textit{protocol state fuzzing} (si veda
ad esempio \cite{openvpn-fuzz-1} e \cite{openvpn-fuzz-2}). Il risultato è stato
che non sono state trovate particolari vulnerabilità nel protocollo, sebbene
altri test abbiano scovato vulnerabilità
legate ad esempio ad una gestione scorretta della memoria (OpenVPN è scritto in C) \cite{openvpn-bug}.
