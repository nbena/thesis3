\section{Protezione di MoonCloud}
L'utilizzo di OpenVPN per collegare la rete MoonCloud alle reti
target presenta un semplice problema di sicurezza: la VPN \textit{apre}
un nuovo punto di ingresso in MoonCloud, e come tale deve essere protetto
il più possibile.\\
A tale scopo sono state predisposte delle regole iptables da applicare
ai server OpenVPN, essendo essi i \textit{punti di esposizione}.


Sebbene il \textit{NAT lato server} di fatto agisca come una sorta
di firewall consentendo solo al traffico proveniente dalla rete
di MoonCloud ed alle sue risposte di arrivare ed essere accettato
dal server, esso non è sufficiente. Lo scenario dettagliato di seguito
ne è infatti la prova.\\
Se qualcuno, ad esempio un attaccante,
effettuasse delle connessioni da un VPN client legittimo senza utilizzare il mappaggio
degli indirizzi IP (passando per la VPN) esse arriverebbero senza problemi
al VPN server ed anche oltre. Questo perché OpenVPN server si limita
a decifrare i pacchetti provenienti dal client ed a scriverli sulla scheda di rete
virtuale. Allo stesso modo iptables applica il NAT in uscita verso la VPN solo per
i pacchetti destinati ad una rete mappata, e relativa traduzione inversa. Per pacchetti
che non non provengono da una rete mappata iptables non effettua la traduzione
inversa (perché tali pacchetti non matchano le regole), per cui semplicemente
i pacchetti passano.
Un attaccante riesce quindi con successo (è stato testato) ad inviare pacchetti
al server in questo modo.
Gli si presentano però alcuni problemi:
\begin{itemize}
	\item se volesse inviare pacchetti ad una qualsiasi \textit{Docker machine}
	      passando per il VPN server, dovrebbe conoscerne l'indirizzo IP
	\item non avrebbe modo di vedere le risposte alle proprie richieste, poiché
	      esse sono fatte con l'IP sorgente reale e non mappato, e le \textit{Docker machine}
	      non hanno nessuna rotta configurata che dica che le risposte verso tali indirizzi
	      IP \textit{reali} debbano passare per il VPN server. Alla peggio, l'indirizzo IP
	      sorgente originale della rete target usato dall'attaccante corrisponde
	      ad un'altra rete rimappata, a cui eventualmente tornerebbero risposte.
\end{itemize}
Sebbene esistano questi due ostacoli, rimangono comunque due possibilità che
l'attaccante può sfruttare:
\begin{itemize}
	\item mandare pacchetti al server
	\item mandare pacchetti alle \textit{Dokcer machine}, assumendo di conoscerne
	      l'indirizzo IP, senza curarsi delle risposte
\end{itemize}

 
Per contrastare questa possibile minaccia, come anticipato, si è utilizzato iptables.
L'idea delle regole applicate si può riassumare così:
\begin{itemize}
	\item usare una politica \texttt{DEFAULT DENY} sul VPN server per cui \textit{solo
		il traffico da MoonCloud verso la rete target e le risposte ad esso} possono passare sul server
		\item ogni volta che un nuovo client si connette, si \textit{concede alle reti
			mappate di tale client} l'accesso.
\end{itemize}
		
Naturalmente, queste regole devono riguardare solo il traffico proveniente/destinato
alla VPN, pertanto un ulteriore matching applicato alle regole riguarda la scheda di
rete virtuale destinazione/sorgente dei pacchetti, che appunto deve essere quella
di OpenVPN. Poiché tali schede seguono tutte la stessa nomenclatura, ovvero iniziano con
``\texttt{ovpn}'', è stato facile applicare le restrizioni in questione\footnote{Si ricorda che
è possibile specificare nel file di configurazione di OpenVPN quale nome dare alle schede di rete
virtuali.}.

La protezione è stata affidata ad iptables inizialmente ed a nftables poi. Concettualmente,
le regole applicate sono le stesse. Si mostrano prima le regole iptables.
		
\subsection{Politica di default}
		
\begin{minted}{bash}
iptables -t filter -I INPUT -i ovpn+ -j DROP
iptables -t filter -I FORWARD -i ovpn+ -j DROP
\end{minted}
		
		Le due regole che si vedono applicano la politica di \texttt{DEFAULT DENY}:
		\begin{itemize}
			\item \texttt{iptables -t filter -I INPUT -i ovpn+ -j DROP} dice di droppare
			      il traffico ricevuto (\texttt{INPUT})
			      da tutte le schede di rete che iniziano con ``\texttt{ovpn}''
			      (il carattere ``\texttt{+}'' fa da \textit{jolly}) e destinato al server locale
			\item \texttt{iptables -t filter -I FORWARD -i ovpn+ -j DROP} invece dice
			      di droppare il traffico in transito sul server locale (\texttt{FORWARD}) (quindi
			      destinato a MoonCloud)
			      che è stato ricevuto dalla schede di rete OpenVPN
		\end{itemize}
		
		Si tratta di una politica di \texttt{DEFAULT} poiché si usa il flag ``\texttt{-I}''
		che indica di aggiungere la regola in questione all'inizio della chain, e per tale
		ragione sarà la prima valutata da iptables. Il flag ``\texttt{-i} <NIC-name>'' è un match
		per pacchetti che sono stati ricevuti da ``\texttt{<NIC-name>}''.
		
		
		Queste regole vengono inserite sui server OpenVPN appena vengono creati, e
		devono essere sempre reinserite ad ogni riavvio, pocihé sono la \textit{base}
		di tutto il meccanismo di protezione di MoonCloud. E' fuori da questa sede la discussione
		di come garantire la persistenza delle regole iptables tra un riavvio e l'altro.
		
		
		\subsection{Traffico consentito}
		Lo step successivo la definizione delle politiche di default è definire le regole che
		consentano al traffico legittimo di passare.
		
		Queste regole si attivano ogni volta che un client si connette al server, e per
		questo motivo si sfrutta ancora una volta l'hook \texttt{client-connect} di OpenVPN.
		Nello script che viene eseguito dal server quando un client si è connesso vi sono
		anche delle regole iptables per consentire \textit{solo le richieste
		dalle Docker machine alle reti mappate di tale client ed alle sue risposte}
		di transitare.
		Supponendo che la rete rimappata di un client dal \texttt{CommonName = client100} sia
		\texttt{192.168.100.0/24}, le regole per farlo sono:
\begin{minted}[breaklines, tabsize=4]{bash}
iptables -t filter -I OUTPUT -d 192.168.100.0/24 -o ovpn+ -m state --state NEW,RELATED,ESTABLISHED -j ACCEPT
iptables -t filter -I INPUT -s 192.168.100.0/24 -i ovpn+ -m state --state RELATED,ESTABLISHED -j ACCEPT
iptables -t filter -I FORWARD -s 192.168.100.0/24 -i ovpn+ -m state --state RELATED,ESTABLISHED -j ACCEPT
\end{minted}
		
		Innanzitutto, si utilizzano regole stateful, per cui iptables mantiene
		una tabella di stato: per ogni nuova connessione, se consentita, si inserisce
		una nuova entry in tale tabella. Tutti i successivi pacchetti in arrivo sono consentiti
		senza ulteriori verifiche
		se appartengono ad una connessione già iniziata.
 		Ciò rende più difficile
		attacchi di \textit{TCP Session Hijacking}, dove un attaccante tenta di inserirsi in
		una connessione TCP.
		
		\begin{itemize}
			\item \texttt{iptables -t filter -I OUTPUT -d 192.168.100.0/24 -o ovpn+ \ldots -j ACCEPT}
			      è la regola che consente al traffico generato localmente e destinato ad OpenVPN
			      di passare. Non c'è un corrispettivo nelle regole di default poiché si tratta di
			      \textit{traffico generato dal server}, non da qualcuno di esterno, pertanto
			      non vi è motivo di negare di default questo tipo di traffico.\
			      L'unico traffico consentito rimane comunque quello verso la rete mappata
			      \texttt{192.168.100.0/24}.
			\item \texttt{iptables -t filter -I INPUT -s 192.168.100.0/24 -i ovpn+ \ldots -j ACCEPT}
				  invece è la regola inserita (``\texttt{-I}'') in una posizione più specifica rispetto
				  alla regola di default
			      che nega il traffico ricevuto da OpenVPN e destinato localmente, di passare.
			      Con questa regola si consente solo al traffico proveniente da \texttt{192.168.100.0/24}
			      destinato al server di transitare. In particolare, si consente \textit{alle sole risposte}
			      (\texttt{-m state --state RELATED,ESTABLISHED})
			      provenienti da \texttt{192.168.100.0/24}
			      di passare. E' molto importante la mancanza di ``\texttt{--state NEW}'',
			      poiché è ciò che abilita, appunto, le sole risposte.
			\item \texttt{iptables -t filter -I FORWARD -s 192.168.100.0/24 -i ovpn+ \ldots -j ACCEPT}
			      al pari della precedente, è più specifica rispetto alla regola che di default vieta il traffico
			      dal server OpenVPN verso MoonCloud.
			      Eseguendola, si consente \textit{alle sole risposte}
			      provenienti da \texttt{192.168.100.0/24}
			      di raggiungere le \textit{Docker machine}. Come nella precedente, si noti che manca
			      ``\texttt{--state NEW}''.
		\end{itemize}
		
		
		\paragraph{Rimozione delle regole}
		Poiché le regole appena elencate devono essere aggiunte al server OpenVPN solo quando
		un client si connette, è chiaro che devono essere altresì rimosse quando esso si disconnette.
		Per farlo ci si affida all'hook \texttt{client-disconnect}, riscrivendo
		le stesse regole ma con l'opzione ``\texttt{-D}'', che indica di rimuovere la regola così
		composta.
		
		Supponendo sempre di aver a che fare con \texttt{client100}, per rimuovere le precedenti regole:
\begin{minted}[breaklines, tabsize=4]{bash}
iptables -t filter -D OUTPUT -d 192.168.100.0/24 -o ovpn+ -m state --state NEW,RELATED,ESTABLISHED -j ACCEPT
iptables -t filter -D INPUT -s 192.168.100.0/24 -i ovpn+ -m state --state RELATED,ESTABLISHED -j ACCEPT
iptables -t filter -D FORWARD -s 192.168.100.0/24 -i ovpn+ -m state --state RELATED,ESTABLISHED -j ACCEPT   
\end{minted}
		Le regole combaciano alla perfezione con quelle usate in \texttt{client-connect}, in questo modo
        iptables può rimuoverle con successo.
        

    \subsection{Configurazione applicata}
    Sebbene sia già stato mostrato nel capitolo precedente (a partire da pagina \pageref{sec:ending}),
    di seguito si riporta come si presentano i file \texttt{client-up.sh} e \texttt{client-down.sh},
    assumendo, come in precedenza, che la rete mappata di \texttt{client100} sia \texttt{192.168.100.0/24}.

\begin{minted}[breaklines, tabsize=4]{bash}

# /etc/openvpn/ovpn1/cs/client-up.sh

#!/bin/bash
                
if [ "$common_name" == "client100" ]; then
	# aggiunta rotta
	ip route add 192.168.100.0/24 via 10.7.0.1

	# aggiunta nat lato server
	iptables -t nat -A POSTROUTING -d 192.168.100.0/24 -j MASQUERADE
        
	# aggiunta traffico consentito
	iptables -t filter -I OUTPUT -d 192.168.100.0/24 -o ovpn+ -m state --state NEW,RELATED,ESTABLISHED -j ACCEPT
	iptables -t filter -I INPUT -s 192.168.100.0/24 -i ovpn+ -m state --state RELATED,ESTABLISHED -j ACCEPT
	iptables -t filter -I FORWARD -s 192.168.100.0/24 -i ovpn+ -m state --state RELATED,ESTABLISHED -j ACCEPT
fi
\end{minted}
        
\begin{minted}[breaklines, tabsize=4]{bash}

# /etc/openvpn/ovpn1/cs/client-down.sh
        
#!/bin/bash
             
if [ "$common_name" == "client100" ]; then
	# rimozione rotta
	ip route add 192.168.100.0/24 via 10.7.0.1

	# rimozione nat lato server
	iptables -t nat -D POSTROUTING -d 192.168.100.0/24 -j MASQUERADE
        
	# rimozione regole di filtraggio
	iptables -t filter -D OUTPUT -d 192.168.100.0/24 -o ovpn+ -m state --state NEW,RELATED,ESTABLISHED -j ACCEPT
	iptables -t filter -D INPUT -s 192.168.100.0/24 -i ovpn+ -m state --state RELATED,ESTABLISHED -j ACCEPT
	iptables -t filter -D FORWARD -s 192.168.100.0/24 -i ovpn+ -m state --state RELATED,ESTABLISHED -j ACCEPT
fi
\end{minted}

\subsection{Protezione con nftables}
Si sfrutta appieno la struttura dati \textit{Set}: l'idea è che si abbiano
delle regole di default analoghe alle precedenti, e che si usi un insieme
in cui si inseriscano le reti mappate. In questo modo, quando un client si
connette si devono solo aggiungere elementi a tale insieme; viceversa quando
si disconnette (``\texttt{client-disconnect}'') è sufficiente rimuovere
elementi da esso.
In questo modo, si mantengono un certo numero di regole fisse (poche),
aumentando le prestazioni del firewall.

Le regole sono raccolte in un file \texttt{init.nft} che deve essere
eseguito sui VPN server al loro avvio:
\begin{minted}[tabsize=4]{squidconf}
#!/usr/sbin/nft -f

table ip ovpn_management {
	
	# 'interval' perché si inseriranno
	# dei NET ID
	set allowed_clients {
		type ipv4_addr;
		flags interval;
	}
	
	# traffico generato dal server verso i client
	chain output {
		type filter hook output priority 0; policy accept;
		ip daddr @allowed_clients meta oifname "ovpn*"
			ct state new,related,established accept
	}
	
	# traffico dai client destinato al server
	chain input {
		type filter hook input priority 0; policy accept;
		# consentire traffico dai client registrati
		ip saddr @allowed_clients meta iifname "ovpn*"
			ct state related,established accept
		# tutto il resto del traffico che viene dalla VPN
		# viene droppato
		meta iifname "ovpn*" drop
	}
	
	# traffico dai client alle Docker machines
	chain forward {
		type filter hook forward priority 0; policy accept;
		# consentire traffico dai client registrati
		ip saddr @allowed_clients meta iifname "ovpn*"
			ct state related,established accept
		# il resto del traffico che viene dalla VPN viene
		# droppato
		meta iifname "ovpn*" drop
	}
	
	# masquerading verso la VPN
	chain masquerading {
		type nat hook postrouting priority 0; policy accept;
		ip daddr @allowed_clients meta oifname "ovpn*"
			masquerade
	}
}
\end{minted}

Infine, ecco come si presentano i nuovi file per gli hook \texttt{client-connect}
e \texttt{client-disconnect}.

\begin{minted}[breaklines, tabsize=4]{bash}
# /etc/openvpn/ovpn1/cs/client-up.sh

#!/bin/bash

if [ "$common_name" == "client100" ]; then
	ip route add 192.168.100.0/24 via 10.7.0.1
	nft add element ovpn_management allowed_clients { 192.168.100.0/24 }
fi	
\end{minted}

\begin{minted}[breaklines, tabsize=4]{bash}
# /etc/openvpn/ovpn1/cs/client-down.sh
	
#!/bin/bash
	
if [ "$common_name" == "client100" ]; then
	ip route del 192.168.10.0/24 via 10.7.0.1
	nft delete element ovpn_management allowed_clients { 192.168.100.0/24 }
fi	
\end{minted}

    % TODO REMEMBER TO ADD THE IMAGES THAT SHOWS
    % THAT ATTACKER CAN NO LONGER VIEW SOMETHING EXPOSES ON THE
	% SERVER.
		
		
		
