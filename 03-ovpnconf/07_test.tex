\section{Testing}
Si passa a descrivere
i due test che sono stati effettuati per verificare che il funzionamento di OpenVPN.

\subsection{Test 1}
Il primo test è stato eseguito sulla configurazione spiegata all'inizio del capitolo:
\begin{itemize}
	\item OpenVPN server ospitato su AWS, raggiungibile da nome DNS, sistema
	      operativo \texttt{Ubuntu Server}, indirizzo IP privato \texttt{172.31.40.221}\footnote{\url{https://www.ubuntu.com/}}
	\item \textit{Docker host} ospitato su AWS, nella stessa rete privata del
	      server, sistema
	      operativo \texttt{Ubuntu Server}, indirizzo IP privato \texttt{172.31.43.141}
	\item due reti virtuali, create con VirtualBox:
	      \begin{itemize}
	      	\item rete 1, detta \texttt{net50}, costituita da:
	      	      \begin{itemize}
	      	      	\item firewall stateful, NAT verso l'esterno e traffico consentito solo
	      	      	      sulle porte 80, 443 TCP e 53 UDP, indirizzo IP \texttt{192.168.100.254},
	      	      	      sistema operativo \texttt{Alpine Linux}\footnote{\url{https://www.alpinelinux.org/}}
	      	      	\item client VPN, sistema operativo \texttt{Ubuntu desktop}, indirizzo
	      	      	      IP \texttt{192.168.100.20}, default gateway \texttt{192.168.100.254}
	      	      	\item due host \texttt{Alpine Linux}, indirizzi IP rispettivamente
	      	      	      \texttt{192.168.100.30} e \texttt{192.168.100.40}, entrambi con un server
	      	      	      web in ascolto sulla porta 80
	      	      \end{itemize}
	      	\item rete 2, detta \texttt{net200}, costituita da:
	      	      \begin{itemize}
	      	      	\item firewall stringente, configurato analogamente a quello dell'altra rete
	      	      	\item client VPN, sistema operativo \texttt{Ubuntu desktop}, indirizzo
	      	      	      IP \texttt{192.168.100.20}, default gateway \texttt{192.168.100.254}
	      	      	\item un host \texttt{Alpine Linux}, con indirizzo IP
	      	      	      \texttt{192.168.100.30} ed un server
	      	      	      web in ascolto sulla porta 80\footnote{Non si è creato un secondo host anche
	      	      	      	in questa rete perché il carico di lavoro sul PC utilizzato iniziava ad
	      	      	      essere elevato.}
	      	      \end{itemize}
	      \end{itemize}
\end{itemize}
Le due reti sono state appositamente scelte con NET ID uguali per testare il funzionamento
dell'\textit{IP Mapping}. In particolare:
\begin{itemize}
	\item \texttt{net50} è stata mappata su \texttt{192.168.50.0/24}
	      \begin{itemize}
	      	\item \texttt{192.168.100.30} $\leftrightarrow$ \texttt{192.168.50.30}
	      	\item \texttt{192.168.100.40} $\leftrightarrow$ \texttt{192.168.50.40}
	      \end{itemize}
	\item \texttt{net200} è stata mappata su \texttt{192.168.200.0/24}
	      \begin{itemize}
	      	\item \texttt{192.168.100.30} $\leftrightarrow$ \texttt{192.168.200.30}
	      	      % \item \texttt{192.168.100.40} $\leftrightarrow$ \texttt{192.168.200.40}
	      \end{itemize}
\end{itemize}
La figure \ref{fig:openvpn-test1} dà una rappresentazione grafica della topologia
realizzata.\\
I due firewall sono stati configurati volutamente in modo restrittivo, al fine
di verificare se OpenVPN fosse stato in grado di passarli.
% Si riportano i comandi
% \texttt{iptables} utilizzati (uguali per entrambi i firewall):
% \begin{minted}[breaklines]{bash}
% #!/bin/ash
% # ash è la shell di Alpine

% # natting
% iptables -t nat -A POSTROUTING -o eth0 -j MASQUERADE

% # dns
% iptables -t filter -A FORWARD -p udp --dport 53 -s 192.168.100.0/24 -j ACCEPT
% iptables -t filter -A FORWARD -p udp --sport 53 -d 192.168.100.0/24 -j ACCEPT

% # http stateful
% iptables -t filter -A FORWARD -p tcp --dport 80 -s 192.168.100.0/24 -m state --state NEW,RELATED,ESTABLISHED -j ACCEPT
% iptables -t filter -A FORWARD -p tcp --sport 80 -d 192.168.100.0/24 -m state --state RELATED,ESTABLISHED -j ACCEPT

% # https stateful
% iptables -t filter -A FORWARD -p tcp --dport 443 -s 192.168.100.0/24 -m state --state NEW,RELATED,ESTABLISHED -j ACCEPT
% iptables -t filter -A FORWARD -p tcp --sport 443 -d 192.168.100.0/24 -m state --state RELATED,ESTABLISHED -j ACCEPT
% \end{minted}
% Sia i client sia il server sono stati configurati come indicato nel capitolo
% precedente. Qui di seguito sono mostrati i file di configurazione, a cominciare
% dal server, poi client in \texttt{net50}, infine quello in \texttt{net200}.
% \begin{minted}{squidconf}
% # server remoto

% mode server
% tls-server

% proto tcp

% dev-type tun
% dev ovpn-server

% persist-tun
% persist-key

% script-security 2

% port 443

% # VPN subnet
% server 10.7.0.0 255.255.255.0

% topology subnet

% client-config-dir /etc/openvpn/server-single/ccd

% # TLS configuration
% remote-cert-eku "TLS Web Client Authentication"
% tls-version-min 1.2
% tls-cipher TLS-ECDHE-RSA-WITH-AES-256-GCM-SHA384
% cipher AES-256-GCM

% # chiavi e certificati
% ca /etc/openvpn/certs/ca.crt
% cert /etc/openvpn/server-single/certs/server-single.crt
% key /etc/openvpn/server-single/certs/server-single.key

% dh none

% reneg-sec 1200

% # log configuration
% log /var/log/openvpn/server-single/openvpn.log
% verb 4
% status /var/log/openvpn/server-single/openvpn-status.log

% script-security-level 2

% client-connect /etc/openvpn/server-single/cs/client-up.sh
% client-disconnect /etc/openvpn/server-single/cs/client-down.sh

% keepalive 10 60

% \end{minted}
% Il contenuto dello script eseguito quando un client si connette:
% \begin{minted}[breaklines]{bash}
% #!/bin/bash

% if [ "$common_name" == "client50" ]; then
% 	ip route add 192.168.50.0/24 via 10.7.0.1
% 	iptables -t nat -A POSTROUTING -d 192.168.50.0/24 -j MASQUERADE
% fi

% if [ "$common_name" == "client200" ]; then
% 	ip route add 192.168.200.0/24 via 10.7.0.1
% 	iptables -t nat -A POSTROUTING -d 192.168.200.0/24 -j MASQUERADE
% fi

% \end{minted}
% Mentre quello eseguito quando i client si disconnettono:
% \begin{minted}[breaklines]{bash}
% #!/bin/bash

% if [ "$common_name" == "client50" ]; then
% 	ip route del 192.168.50.0/24 via 10.7.0.1
% 	iptables -t nat -D POSTROUTING -d 192.168.50.0/24 -j MASQUERADE
% fi

% if [ "$common_name" == "client200" ]; then
% 	ip routedel 192.168.200.0/24 via 10.7.0.1
% 	iptables -t nat -D POSTROUTING -d 192.168.200.0/24 -j MASQUERADE
% fi

% \end{minted}
% Il file \path{/etc/openvpn/server-single/ccd/client50} contiene la seguente
% linea:
% \begin{minted}{squidconf}
% iroute 192.168.50.0 255.255.255.0
% \end{minted}
% Parallelamente, ecco come si presenta
% \path{/etc/openvpn/server-single/ccd/client200}:
% \begin{minted}{squidconf}
% iroute 192.168.200.0 255.255.255.0
% \end{minted}
% Di seguito invece file di configurazione del client VPN in \texttt{net50}, il cui
% \texttt{Common Name} è \texttt{client50}.
% \begin{minted}{squidconf}
% # /etc/openvpn/client50.conf

% client

% proto tcp

% remote ec2-52-15-172-187.us-east-2.compute.amazonaws.com 443
% verify-x509-name server-single name

% dev-type tun
% dev ovpn-client50

% persist-tun
% persist-key

% group nogroup
% user nobody

% # TLS configuration
% remote-cert-eku "TLS Web Server Authentication"
% tls-version-min 1.2
% tls-cipher TLS-ECDHE-RSA-WITH-AES-256-GCM-SHA384
% cipher AES-256-GCM

% # chiavi e certificati
% ca /etc/openvpn/certs/ca.crt
% cert /etc/openvpn/client50/certs/client50.crt
% key /etc/openvpn/client50/certs/client50.key

% # log configuration
% log /var/log/openvpn/client50/openvpn.log
% verb 4
% status /var/log/openvpn/client50/openvpn-status.log
% \end{minted}
% Infine, il file del client in \texttt{net200}, dal \texttt{Common Name}:
% \texttt{client200}.
% \begin{minted}{squidconf}
% # /etc/openvpn/client200.conf

% client

% proto tcp

% remote ec2-52-15-172-187.us-east-2.compute.amazonaws.com 443
% verify-x509-name server-single name

% dev-type tun
% dev ovpn-client200

% persist-tun
% persist-key

% group nogroup
% user nobody

% # TLS configuration
% remote-cert-eku "TLS Web Server Authentication"
% tls-version-min 1.2
% tls-cipher TLS-ECDHE-RSA-WITH-AES-256-GCM-SHA384
% cipher AES-256-GCM

% # chiavi e certificati
% ca /etc/openvpn/certs/ca.crt
% cert /etc/openvpn/client200/certs/client200.crt
% key /etc/openvpn/client200/certs/client200.key

% # log configuration
% log /var/log/openvpn/client200/openvpn.log
% verb 4
% status /var/log/openvpn/client200/openvpn-status.log
% \end{minted}
Una volta messa a punto la connessione, l'obiettivo del test era riuscire a visualizzare
le pagine web offerte dai target dal \textit{Docker host}.
% Come detto prima, i tre host che si vogliono raggiungere sono
% mappati su:
% \begin{itemize}
%   \item \texttt{192.168.50.30} (\texttt{net50})
%   \item \texttt{192.168.50.40} (\texttt{net50})
%   \item \texttt{192.168.200.30} (\texttt{net200})
% \end{itemize}
Per far sì che il \textit{Docker host} possa raggiungere i target, è necessario
configurare la routing table del suo kernel per instradare i pacchetti destinati
ai target al VPN server. L'indirizzo IP di tale server è \texttt{172.31.40.221},
per cui si sono eseguiti i seguenti due comandi sul \textit{Docker host}:
\begin{minted}{bash}
ip route add 192.168.50.0/24 via 172.31.40.221
ip route add 192.168.200.0/24 via 172.31.40.221
\end{minted}

Mentre con SoftEther non si era nemmeno riusciti a completare la connessione, la
figura \ref{fig:test1-result} mostra che la soluzione funziona. Si tratta della
shell dell'autore connessa in SSH al \textit{Docker host} sul quale è in esecuzione un container
\texttt{Alpine}.

\begin{figure}
	\includegraphics[scale=0.6]{img/openvpn_test1}
	\caption{Test 1 per OpenVPN}
	\label{fig:openvpn-test1}
\end{figure}

\begin{figure}[h]
	\includegraphics[scale=0.35]{img/test1-result}
	\caption{Il risultato dell'esecuzione del test}
	\label{fig:test1-result}
\end{figure}
% A titolo di documetazione, si mostra anche l'output del comando \texttt{curl} eseguito
% direttamente sui tre host target.
% \begin{figure}
%   \includegraphics[scale=0.4]{img/alpine-internal-client1-net50}
%   \caption[Output di \texttt{curl} su target 1 in \texttt{net50}]
%   {L'output di \texttt{curl} su \texttt{192.168.50.30} (mappato)}
%   \label{fig:alpine-internal-client1-net50}
% \end{figure}
% \begin{figure}
%   \includegraphics[scale=0.4]{img/alpine-internal-client2-net50}
%   \caption[Output di \texttt{cur} su target 2 in \texttt{net50}]
%   {L'output di \texttt{curl} su \texttt{192.168.50.40} (mappato)}
%   \label{fig:alpine-internal-client2-net50}
% \end{figure}
% \begin{figure}
%   \includegraphics[scale=0.4]{img/alpine-internal-client1-net50}
%   \caption[Output di \texttt{curl} su target 3 in \texttt{net200}]
%   {L'output di \texttt{curl} su \texttt{192.168.200.30} (mappato)}
%   \label{fig:alpine-internal-client-net200}
% \end{figure}

\subsection{Test 2}
Il secondo test è stato eseguito interamente su macchine virtuali messe
a disposizione dal SESAR Lab. Per brevità, non si riporta di nuovo la
configurazione dettagliata del test, poiché a parte gli indirizzi IP
che in sè sono diversi, il setup è stato di fatto lo stesso. L'unica
differenza è che si è usata una sola rete target anziché due.
L'obiettivo era testare una connessione SSH dalla \textit{Docker machine}
collegata al VPN server verso un host nella rete target (passando
ovviamente per la VPN).
Il test ha dato esito positivo, lo schema è mostrato figura \ref{fig:openvpn-test2}.
\begin{figure}
	\includegraphics[scale=0.55]{img/openvpn_test2}
	\caption{Test 2 per OpenVPN}
	\label{fig:openvpn-test2}
\end{figure}