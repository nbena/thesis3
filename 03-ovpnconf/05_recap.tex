\section{Recap}
Prima di passare ad analizzare come tutte le soluzioni appena indicate si combinano
tra loro, è utile fare un breve riassunto dei problemi affrontati
e delle soluzioni attuate.
\begin{description}
  \item[\textit{NAT al contrario}]I pacchetti provenienti dalla rete MoonCloud
  hanno come IP sorgente un IP appartente a tale rete, affinché le risposte possano
  tornare indietro, occorre che in ciascuna rete target sia configurata la
  seguente rotta: \textit{rete: rete-MoonCloud via: IP-client-VPN}. Essa può
  essere inserita nel default gateway di ciascuna rete oppure in ogni host di tali reti.
  Tuttavia, non è possibile intervenire sui dispositivi appena elencati, pertanto
  la soluzione è quella di applicare NAT ai pacchetti provenienti da MoonCloud e
  diretti alla rete target, quindi ai pacchetti in uscita dal device VPN client.
  In questo modo, i pacchetti hanno come indirizzo IP sorgente quello del device
  client, il quale si trova nella stessa rete degli host target, che quindi inviano
  ad esso le proprie risposte, senza bisogno di configurare alcuna rotta
  aggiuntiva,
  \item[\textit{Bypassing della direttiva ``\texttt{route \ldots}''}]Lo
  scopo di tale direttiva, sia essa posta nel client o nel server, è quella di
  indicare quali rotte sono raggiungibili mediante l'\textit{altro}. Tuttavia, poiché
  OpenVPN legge il file di configurazione solo all'avvio, e non si sa in principio
  quali client saranno connessi al server, diventa un problema inserirla nel file
  del server. L'effetto di questa opzione è di aggiungere una nuova rotta nel kernel
  del sistema operativo così fatta: \texttt{<net> via <virtual-NIC's-IP>}. Per
  ovviare a questo problema si aggiunge direttamente tale rotta con il relativo comando di
  sistema al momento in cui un client si connette, e la si cancella quando si
  disconnette; si sfruttano gli hook \texttt{client-connect} e \texttt{client-disconnect}
  che consentono di specificare uno script da eseguire in uno dei due eventi
  appena descritti.
  \item[\textit{NAT lato server}]In OpenVPN
  client i pacchetti ricevuti sulla scheda di rete virtuale sono risposte provenienti
  dagli host nella rete target, ed hanno come IP di destinazione un IP appartenente
  alla rete in cui si trovano i \textit{Docker host}. Affinché OpenVPN sappia che
  tali IP sianno raggiungibili tramite la VPN, è necessario che il server
  pubblicizzi tale rotte, e questo viene normalmente fatto mediante la direttiva
  \texttt{push "route <route-to-push>"}, il cui effetto è che il client aggiunga
  tale rotta alla routing table del kernel. Tuttavia, non si può sapere a priori
  quali reti saranno dietro il server, a tale scopo l'OS del server
  effettua del NAT sui pacchetti diretti alla scheda di rete virtuale OpenVPN,
  settando l'IP sorgente come l'IP della scheda di rete virtuale. Spostandosi
  nel client, si ha che i pacchetti di risposta a MoonCloud hanno come IP destinazione
  quello virtuale del server, il quale si trova nello spazio di indirizzamente della
  VPN, quindi il sistema operativo sa come instradarlo. Infatti, l'indirizzo IP
  della scheda di rete virtuale del client, si trova nella stessa subnet di quello
  del server, il cui effetto è quello di avere una rotta automaticamente aggiunta
  dal sistema operativo: \texttt{<VPN-subnet> via <virtual-NIC's IP>}.
  \item[\textit{IP mapping}]Per poter connettere tra di loro $n$ reti mediante
  una VPN è necessario che ciascuna rete partecipante abbia un NET ID diverso. Poiché
  si vuole utilizzare un server per connettere diverse reti target di diversi clienti,
  è ragionevole presumere che vi sia prima o poi un conflitto, e che due reti abbiano
  lo stesso NET ID. Per superare questo problema è stato introdotto il nuovo concetto
  di \textit{mappaggio delle reti/degli indirizzi IP}, per il quale MoonCloud assegna
  ad ogni rete un nuovo NET ID, e mappa gli IP originali nel nuovo NET ID. Questo
  mappaggio è garantito univoco. Lato client, si utilizza \texttt{iptables} per
  modificare gli header IP e rendere ai clienti questa complessa operazione del tutto
  trasparente.
\end{description}
