\section{Configurazione finale}\label{sec:ending}
Dopo aver visto tutti i problemi affrontati e come sono stati risolti, è il
momento di vedere come sono configurati \textit{davvero} i server ed i client
VPN in MoonCloud. Come per il mappaggio degli indirizzi IP, il microservizio
\texttt{MoonCloud\_VPN}
si occupa di automatizzare e di generare tutto ciò che
serve.\\
Durante tutte le sottosezioni successive, si supporrà la seguente configurazione
di esempio:
\begin{itemize}
  \item rete target 1: \texttt{192.168.100.0/24} (cliente $A$)
  \item rete target 1 mappata: \texttt{192.168.1.0/24}
  \item rete MoonCloud: \texttt{192.168.200.0/24}
  \item indirizzo IP interno VPN server: \texttt{192.168.200.20}
  %\item indirizzo IP pubblico VPN server: \texttt{100.4.78.5}
  \item \texttt{CommonName} e nome DNS del VPN server: \texttt{ovpn1.moon-cloud.eu}
  \item indirizzo IP del \textit{Docker host} per client $A$: \texttt{192.168.200.2}
  \item indirizzo IP VPN client $A$: \texttt{192.168.100.50}
  \item subnet VPN: \texttt{10.7.0.0/24}
  \item indirizzo IP scheda di rete virtuale VPN server: \texttt{10.7.0.1}
  \item indirizzo IP scheda di rete virtuale VPN client $A$: \texttt{10.7.0.2}
  \item \texttt{CommonName} del VPN client $A$: \texttt{office1.company.com}
\end{itemize}
Si tenga presente che ci si concentra sulla configurazione della VPN ed anche l'
\textit{IP mapping} viene affrontato solo dal punto di vista della VPN. Per avere ulteriori
informazioni sulla generazione dei certificati, sull'algoritmo che mappa una rete o che
mappa un IP, il capitolo \ref{ch:microservice} è dedicato ad esso.
Nei file \texttt{client-up.sh} e \texttt{client-down.sh} si noteranno alcune regole iptables
che non state ancora spiegate: esse sono funzionali ad incrementare la sicurezza della VPN,
e sono descritte nel prossimo capitolo.

\subsection{Configurazione del server}
Al momento della creazione di un nuovo server, viene generato un file di configurazione
simile a quello mostrato di seguito.
\begin{minted}[breaklines]{squidconf}
# /etc/openvpn/ovpn1.conf

# ruolo
mode server
tls-server

# protocollo L4
proto tcp

# configurazione NIC
# virtuale
dev-type tun
dev ovpn-1

persist-tun
persist-key

script-security 2

# porta di ascolto 443
port 443

# VPN subnet
server 10.7.0.0 255.255.255.0

topology subnet

# path per client-specific file
client-config-dir /etc/openvpn/ovpn1/ccd

# TLS configuration
remote-cert-eku "TLS Web Client Authentication"
tls-version-min 1.2

tls-cipher TLS-ECDHE-ECDSA-WITH-AES-256-GCM-SHA384
cipher AES-256-GCM

# keys and certs
ca /etc/openvpn/certs/ca.crt
key /etc/openvpn/ovpn1/certs/ovpn1.key
cert /etc/openvpn/ovpn1/certs/ovpn1.crt

# per ECDH
dh none

# nuove chiavi ogni 20 minuti
reneg-sec 1200

# log configuration
log /var/log/openvpn/ovpn1/openvpn.log
verb 4
status /var/log/openvpn/ovpn1/openvpn-status.log

# livello di sicurezza per comandi user-defined
script-security-level 2

# scripts
client-connect /etc/openvpn/ovpn1/cs/client-up.sh
client-disconnect /etc/openvpn/ovpn1/cs/client-down.sh

# opzioni 'ping'
keepalive 10 60
\end{minted}
Il file mostrato sopra verrà posizionato in \path{/etc/openvpn/ovpn1.conf}, ed
ovviamente tutti i percorsi indicati in questo file devono esistere.\\
I file \path{/etc/openvpn/ovpn1/cs/client-up.sh} e \\
\path{/etc/openvpn/ovpn1/cs/client-down.sh}:
\begin{minted}[breaklines]{bash}
# /etc/openvpn/ovpn1/cs/client-up.sh

#!/bin/bash

if [ "$common_name" == "office1.company.org" ]; then
  ip route add 192.168.1.0/24 via 10.7.0.1
  iptables -t nat -A POSTROUTING -d 192.168.1.0/24 -j MASQUERADE

  # non fondamentali, verranno spigete in seguito.
  iptables -t filter -I OUTPUT -d 192.168.1.0/24 -o ovpn+ -m state --state NEW,RELATED,ESTABLISHED -j ACCEPT
  iptables -t filter -I INPUT -s 192.168.1.0/24 -i ovpn+ -m state --state RELATED,ESTABLISHED -j ACCEPT
  iptables -t filter -I FORWARD -s 192.168.1.0/24 -i ovpn+ -m state --state RELATED,ESTABLISHED -j ACCEPT
fi
\end{minted}

\begin{minted}[breaklines]{bash}
# /etc/openvpn/ovpn1/cs/client-down.sh

#!/bin/bash

if [ "$common_name" == "office1.company.org" ]; then
  ip route add 192.168.1.0/24 via 10.7.0.1
  iptables -t nat -D POSTROUTING -d 192.168.1.0/24 -j MASQUERADE

  # verranno spiegate in seguito
  iptables -t filter -D OUTPUT -d 192.168.1.0/24 -o ovpn+ -m state --state NEW,RELATED,ESTABLISHED -j ACCEPT
  iptables -t filter -D INPUT -s 192.168.1.0/24 -i ovpn+ -m state --state RELATED,ESTABLISHED -j ACCEPT
  iptables -t filter -D FORWARD -s 192.168.1.0/24 -i ovpn+ -m state --state RELATED,ESTABLISHED -j ACCEPT
fi
\end{minted}
Il file \path{/etc/openvpn/ovpn1/ccd/office1.company.org}:
\begin{minted}{squidconf}
iroute 192.168.1.0 255.255.255.0
\end{minted}

\subsection{Configurazione del client}
Il file di configurazione del client si presenta come di seguito.
\begin{minted}{squidconf}
# /etc/openvpn/office1.conf

client

# protocollo L4
proto tcp

# server address and port
remote ovpn1.moon-cloud.eu 443

verify-x509-name ovpn1.moon-cloud.eu name

# virtual NIC configuration
dev-type tun
dev ovpn-office1

persist-tun
persist-key

group nogroup
user nobody

# TLS configuration
remote-cert-eku "TLS Web Server Authentication"
tls-version-min 1.2
tls-cipher TLS-ECDHE-ECDSA-WITH-AES-256-GCM-SHA384
cipher AES-256-GCM

# chiavi e certificati
ca /etc/openvpn/certs/ca.crt
key /etc/openvpn/office1/certs/office1.key
cert /etc/openvpn/office1/certs/office1.crt

# log configuration
log /var/log/openvpn/office1/openvpn.log
verb 4
status /var/log/openvpn/office1/openvpn-status.log
\end{minted}
Si possono osservare le seuguenti differenze rispetto al server:
\begin{itemize}
  \item mancanza della direttiva ``\texttt{topology subnet}'' poiché è
  riservata al server
  \item mancanza della direttiva ``\texttt{client-config-dir}'' per la stessa
  ragione
  \item mancanza delle direttive ``\texttt{client-connect}'' e
  ``\texttt{client-disconnect}'' per la stessa ragione
  \item aggiunta della direttiva ``\texttt{remote}'' che indica come raggiungere
  il server
  \item aggiunta delle direttive ``\texttt{user nobody}'' e ``\texttt{group nogroup}''
  che consentono di fare il downgrade dei privilegi. Non è possibile farlo sul
  server poiché è necessario che abbia privilegi amministrativi per eseguire i
  comandi contenuti negli script di connessione e disconnessione dei client.
\end{itemize}
Si è supposto di mappare \texttt{192.168.100.0/24} in \texttt{192.168.1.0/24},
ed ogni IP della rete originale viene mappato nella nuova rete nel seguente
modo: $\texttt{192.168.100.1} \leftrightarrow \texttt{192.168.1.1}$, e così via
fino a $\texttt{192.168.100.254} \leftrightarrow \texttt{192.168.1.254}$.
Si produce quindi uno script di comandi per \texttt{iptables} così composto:
\begin{minted}[breaklines]{bash}
#!/bin/bash

# ip mapping, modifica destinazione
# (done for request packets)
iptables -t nat -A PREROUTING -d 192.168.1.1 -s 10.7.0.0/24 -j DNAT --to-destination 192.168.100.1
iptables -t nat -A PREROUTING -d 192.168.1.2 -s 10.7.0.0/24 -j DNAT --to-destination 192.168.100.2
iptables -t nat -A PREROUTING -d 192.168.1.3 -s 10.7.0.0/24 -j DNAT --to-destination 192.168.100.3
# ...
iptables -t nat -A PREROUTING -d 192.168.1.254 -s 10.7.0.0/24 -j DNAT --to-destination 192.168.100.254

# 'reverse NAT'
iptables -t nat -A POSTROUTING -d 192.168.1.0/24 -s 10.7.0.0/24 -j MASQUERADE

# ip mapping, modifica sorgente
# (done for response packets)
iptables -t nat -A POSTROUTING -s 192.168.100.1 -d 10.7.0.0/24 -j SNAT --to-source 192.168.1.1
iptables -t nat -A POSTROUTING -s 192.168.100.2 -d 10.7.0.0/24 -j SNAT --to-source 192.168.1.2
iptables -t nat -A POSTROUTING -s 192.168.100.3 -d 10.7.0.0/24 -j SNAT --to-source 192.168.1.3
# ...
iptables -t nat -A POSTROUTING -s 192.168.100.254 -d 10.7.0.0/24 -j SNAT --to-source 192.168.1.254
\end{minted}
