\section{SoftEther}
SoftEther è stata la prima scelta una volta terminato lo studio di cui al
capitolo precedente. La ragione di ciò
sta principalmente nella funzionalità di NAT traversal e di VPN-over-HTTPS; la
configurazione che si era pensata era di tipo \textit{Remote Server}.\\
Il passo successivo è stato quello di testare tale soluzione in un ambiente virtuale,
ecco come si componeva la configurazione nel test
iniziale:
\begin{itemize}
  \item su Amazon EC2: 1 host che fungeva da client VPN
  \item sul laptop dell'autore, mediante VirtualBox, due reti locali separate (come
  se fossero due reti target di due clienti diversi), ciascuna composta da:
  \begin{itemize}
    \item 1 router/firewall di confine che eseguisse il NAT verso l'esterno
    \item 1 server VPN
    \item due host ``normali'' con un server web in esecuzione
  \end{itemize}
\end{itemize}
Tutti i PC coinvolti utilizzavano una distribuzione Linux, in particolare \texttt{Ubuntu}
(in versione server su Amazon), mentre i firewall utilizzavano \texttt{Alpine Linux}.\\
L'obiettivo di questo test era, innazitutto riuscire a connetere client e server in
VPN, e quindi poter visualizzare dal client VPN le pagine web offerte dagli host
nelle due reti.\\
I firewall avevano una configurazione di tipo \textit{stateful}, e l'unico traffico
ammesso era sulle porte 80 e 443 TCP. Si voleva infatti la capacità di SoftEther di passare
tramite un firewall stringente.


Sul server VPN si è installato il componente ``\textit{SoftEther VPN Server}'', sul client si è
invece installato ``\textit{SoftEther VPN Bridge}''. Realizzare questo primo passo
non si è dimostrato facile, a causa della scarsa documentazione disponbile per Linux,
ed anche a causa di \texttt{vpncmd}, la CLI per gestire SoftEther.\\
Una volta completata l'installazione, con una certa difficoltà si è scoperto che la funzionalità
di NAT traversal tanto importante era disponibile solo su UDP, per forza di
cose si è scartata la soluzione \textit{Remote Server} (poiché non c'era modo di contattare
il server dietro ad un NAT).\\
Il passo successivo è stato quindi quello di installare il \textit{SoftEther VPN Server}
sull'host in esecuzione nel cloud di Amazon ed il \textit{Bridge} sulla VM che precedentemente
ospitava il server. Dopo aver proceduto alla configurazione dei due software, si è tentato
di connetterli. Si è notato che client e server riuscivano a comunicare, ma non a portare
a termine la connessione. Da una disamina dei log, è apparso un misterioso \texttt{Error Code 33}.
Un'attenta ricerca sul web non ha portato a risultati, pertanto si è deciso di abbandonare
questa tecnologia.
