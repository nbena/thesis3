\section{SoftEther}
Dopo aver concluso lo studio sulle principali tecnologie VPN attualmente disponibili,
è stata scelta, assieme al prof. Anisetti, SoftEther. La ragione di tale scelta
sta principalmente nella funzionalità di NAT traversal e di VPN-over-HTTPS; la
configurazione che si era pensata era di tipo \textit{Remote Server}.\\
Il passo successivo è stato quello di testare tale soluzione in un ambiente virtuale,
mi sono quindi iscritto ad Amazon AWS in modo che disponessi di un host in Internet
su cui effettuare le mie prove. Ecco come si componeva la configurazione nel test
iniziale:
\begin{itemize}
  \item su Amazon EC2: 1 host che fungeva da client VPN
  \item sul mio PC ho creato, mediante VirtualBox, due reti locali separate (come
  se fossero due reti target di due client diversi), ciascuna composta da:
  \begin{itemize}
    \item 1 router/firewall di confine che eseguisse il NAT verso l'esterno
    (utilizzando iptables)
    \item 1 server VPN
    \item due host ``normali'' con un server web in esecuzione
  \end{itemize}
\end{itemize}
Tutti i PC coinvolti utilizzavano una distribuzione Linux, in particolare \texttt{Ubuntu}
(in versione server su Amazon), mentre i firewall utilizzavano \texttt{Alpine Linux}.\\
L'obiettivo di questo test era, innazitutto riuscire a connetere client e server in
VPN, e quindi poter visualizzare dal client VPN le pagine web offerte dagli host
nelle due reti.\\
I firewall avevano una configurazione di tipo \textit{stateful}, e l'unico traffico
ammesso era sulle porte 80 e 443 TCP.\\
Questo test non ha avuto successo, infatti non sono riuscito nemmeno a raggiungere il
punto 1, non c'è stato modo di far connettere client e server. Nel prossimo paragrafo
descriverò più nel dettaglio la configurazione di SoftEther che ho testato.


Sul server VPN ho installato il componente ``\textit{SoftEther VPN Server}'', sul client ho
invece installato ``\textit{SoftEther VPN Bridge}''. Già solo realizzare questo primo passo
non si è dimostrato facile, a causa della scarsa documentazione disponbile per Linux,
ed anche a causa dell'\textit{operosità} dell'installazione e della configurazione:
infatti per gestire tali software è disponibile un programma linea di comando chiamato
``\texttt{vpncmd}'' con il quale l'interazione è abbastanza \textit{tediosa}.\\
Una volta che l'installazione è terminata, mi sono scontrato con il primo vero problema:
il tanto sbandierato NAT traversal offerto dal server funziona su UDP (scoprirlo non
è stato affatto facile), pertanto la soluzione \textit{Remote Server} è stata scartata.
Tra i requisiti per la VPN vi era infatti la necessità di funzionare su TCP, e senza
la possibilità per il server installato nella rete cliente di porter essere raggiunto
dall'esterno (trovandosi ragionevolmente dietro un NAT) questa configurazione è stata
abbandonata.\\
Il passo successivo è stato quindi quello di installare il \textit{SoftEther VPN Server}
sull'host in esecuzione nel cloud di Amazon ed il \textit{Bridge} sulla VM che precedentemente
ospitava il server. Ho quindi configurato propriamente i due software, ed ho quindi
tentato di collegarli, tuttavia il collegamento VPN non si instaurava. Ho quindi
esaminato i file di log, ed ho scoperto che i componenti riuscivano ad instaurare
una connessione, ma vi era un misterioso \texttt{Error Code 33}. Ho cercato a lungo
cosa significasse tale errore, ma dopo due giorni mi sono arreso.\\

Sia chiaro che non metto in dubbio la bontà di SoftEther, esso infatti è largamente usato,
ad esempio anche nel progetto \texttt{VPN Gate} (\url{http://www.vpngate.net}); tuttavia
SoftEther non ha funzionato nello scenario in questione.
