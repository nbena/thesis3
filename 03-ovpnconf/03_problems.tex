\section{Problematiche di configurazione}
Nella precedente sezione si è descritto OpenVPN e si è mostrato una configurazione
tipica. A questo punto, prima di dettagliare come OpenVPN è stata configurato
per MoonCloud, è necessario soffermarsi su quali siano le problematiche che i particolari
requisiti di VPN per MoonCloud introducono. Per
ciascun problema si analizza anche la relativa soluzione adottata. Durante
questa sezione si potrà notare l'elevata versatilità di OpenVPN, grazie alle numerosissime
opzioni di configurazione possibili.\\
Prima di procedere, si tenga presente che la configurazione \textit{definitiva}
è di tipo \textit{Local Server - LS}, con i server OpenVPN installati
in MoonCloud, e nelle reti target vi sono i client.

\subsection{Impossibilità di aggiungere configurazioni alla rete target}\label{subsec:client-nat}
\subsubsection{Il problema}
A questo problema era già stata pensata una soluzione nella fase iniziale di valutazione
delle diverse VPN. Il quesito è il seguente: \textit{per realizzare una topologia
LAN-to-LAN è necessario configurare delle rotte sui router di confine, ma non è
pensabile di operare sul router della rete target. Come fare?}
Per cercare di comprendere meglio il problema, si analizza il flusso seguito
da un ipotetico pacchetto inviato da un \textit{probe} dalla rete MoonCloud alla rete target,
posto che il collegamento sia stato instaurato con successo. Naturalmente, non vi sono
problemi nel configurare rotte lato MoonCloud. Si supponga quindi di trovarsi nel
seguente scenario:
\begin{itemize}
  \item rete target: \texttt{192.168.100.0/24}
  \item rete MoonCloud: \texttt{192.168.200.0/24}
  \item indirizzo IP interno VPN server: \texttt{192.168.200.20}
  \item indirizzo IP pubblico VPN server: \texttt{100.4.78.5}
  \item indirizzo IP del \textit{Docker host}: \texttt{192.168.200.2}
  \item indirizzo IP VPN client: \texttt{192.168.100.50}
  \item subnet VPN: \texttt{10.7.0.0/24}
  \item indirizzo IP scheda di rete virtuale VPN server: \texttt{10.7.0.1}
  \item indirizzo IP scheda di rete virtuale VPN client: \texttt{10.7.0.2}
  \item \texttt{CommonName} del VPN client: \texttt{client100}
\end{itemize}
Sul \textit{Docker host} di MoonCloud è configurata le seguente rotta:
\texttt{192.168.100.0/24 via 192.168.200.20} che specifica che per contattare la rete target
è necessario passare per il VPN server. Oltre ad essa, naturalmente, il server ed il
client VPN hanno configurato le rotte come si è visto nella sezione precdente
(per cui si dice ``\textit{rete dietro l'altro endpoint via virtual NIC's IP}'').\\
Si supponga quindi che un \textit{probe}  generi un pacchetto di qualsiasi tipo destinato
ad un host della rete target, ad esempio \texttt{192.168.100.254}.
Il flusso seguito dal pacchetto è quello descritto di seguito.
\begin{enumerate}
  \item Il \textit{probe} genera il pacchetto, la sorgente di tale pacchetto è
  \texttt{192.168.200.2}, la destinazione \texttt{192.168.100.254}\footnote{Volendo essere
  precisi, al momento della generazione
  del pacchetto in un container Docker, la sua sorgente è l'indirizzo IP di tale
  container, che poi viene modificato da Docker nell'indirizzo IP dell'host in cui
  il container è esecuzione, ai fini dell'esempio questo non è importante.}
  \item il sistema operativo del \textit{Docker host} consulta la propria routing table
  ed instrada il pacchetto verso \texttt{192.168.200.20}
  \item il sistema operativo del VPN server riceve il pacchetto, consulta la
  routing table e trova l'entry \texttt{192.168.100.0/24 via 10.7.0.1}, quindi inoltra
  alla scheda di rete virtuale di OpenVPN
  \item OpenVPN riceve il pacchetto, consulta la propria routing table interna e
  trova l'entry che specifica che il client ``\texttt{client100}'' è responsabile per
  la rete \texttt{192.168.100.0/24}, provvede quindi ad inviarlo a tale client
  \item il VPN client riceve il pacchetto dal socket, e lo scrive sulla scheda di rete
  virtuale
  \item il pacchetto destinato a \texttt{192.168.100.254} viene ricevuto dal sistema
  operativo, dopo aver compiuto le dovute operazioni di routing viene quindi inviato
  sulla scheda di rete fisica dell'host
  \item \texttt{192.168.100.254} riceve il pacchetto, lo elabora, produce una risposta,
  la destinazione di tale risposta è \texttt{192.168.200.2}
  \item il sistema operativo di \texttt{192.168.100.254} invia la risposta al proprio
  default gateway come avviene in tutti i casi in cui un host non abbia una rotta
  ben determinata per un pacchetto
  \item il default gateway della rete \texttt{192.168.100.0/24} riceve il pacchetto
  destinato a \texttt{192.168.200.2}. Esso può dropparlo in quanto indirizzo privato, o
  comunque inviarlo su Internet. In ogni caso non raggiungerà la destinazione corretta.
\end{enumerate}
In conclusione quindi, il problema sta nel fatto che nella rete target manca una rotta
che dica che per raggiungere \texttt{192.168.200.0/24} occorre contattare il
VPN client. Di nuovo, non è pensabile di dover configurare ogni rete target in questa
maniera, poiché l'utilizzo di MoonCloud deve essere il più possibile \textit{leggero};
la necessità di dover configurare al router potrebbe infatti scoraggiare un potenziale
cliente dall'usare MoonCloud.

\subsection{La soluzione}
La soluzione a ciò è semplice, ovvero far sì che tutti i pacchetti inviati dal
VPN client provenienti dalla VPN e destinati alla rete target abbiano come
indirizzo IP sorgente l'indirizzo IP del device stesso. In questo modo, quando si
produce la risposta, la destinazione di tale pacchetto è un indirizzo IP appartenente
alla stessa rete in cui si è, ovvero la rete target. L'OS quindi invia il
pacchetto al device client, il quale provvede quindi a modificare
l'IP destinazione nell'IP di destinazione \textit{vero}, ovvero quello del Docker host.
Ciò che avviene quindi è che il device client fa del NAT, come se la rete target
fosse la rete \textit{esterna} verso la quale il NAT è tipicamente eseguito. Questa
configurazione di NAT viene qui chiamata ``\textit{NAT al contrario}'', ed è eseguita con
\texttt{iptables}\footnote{\url{https://netfilter.org/projects/iptables/index.html}}.\\
In questo caso, un singolo comando ad iptables risolve tutto questo problema:
\begin{minted}[breaklines]{bash}
iptables -t nat -A POSTROUTING -d 192.168.100.0/24 -s 192.168.200.0 -j MASQUERADE
\end{minted}
Utilizzare il NAT introduce appartenente un'altra problematica, ovvero il fatto che
solo risposte a richieste provenienti da \texttt{192.168.200.0/24} possono ``passare''
il NAT: in questo caso però non vi è alcun problema, proprio perché i pacchetti che
vengono inviati da \texttt{192.168.100.0} \textit{sono} risposte a
\texttt{192.168.200.0/24}.\\
Si analizza ora nuovamente il flusso indicato sopra alla luce del \textit{NAT al contrario}.
\begin{enumerate}
  \item Il \textit{probe} genera il pacchetto, la sorgente di tale pacchetto è
  \texttt{192.168.200.2}, la destinazione \texttt{192.168.100.254}.
  \item il sistema operativo del \textit{Docker host} consulta la propria routing table
  ed instrada il pacchetto verso \texttt{192.168.200.20}
  \item il sistema operativo del VPN server riceve il pacchetto, consulta la
  routing table e trova l'entry \texttt{192.168.100.0/24 via 10.7.0.1}, quindi inoltra
  alla scheda di rete virtuale di OpenVPN
  \item OpenVPN riceve il pacchetto, consulta la propria routing table interna e
  trova l'entry che specifica che il client ``\texttt{client100}'' è responsabile per
  la rete \texttt{192.168.100.0/24}, provvede quindi ad inviarlo a tale client
  \item il VPN client riceve il pacchetto dal socket, e lo scrive sulla scheda di rete
  virtuale
  \item il pacchetto destinato a \texttt{192.168.100.254} viene ricevuto dal sistema
  operativo, dopo aver compiuto le dovute operazioni di routing decide di inviarlo
  sulla scheda di rete fisica dell'host
  \item il pacchetto passa per l'hook di \texttt{POSTROUTING}, e si modifica l'indirizzo IP
  sorgente in \texttt{192.168.100.50}
  \item \texttt{192.168.100.254} riceve il pacchetto, lo elabora, produce una risposta,
  la destinazione di tale risposta è \texttt{192.168.100.50}
  \item il sistema operativo di \texttt{192.168.100.254} invia la risposta a
  \texttt{192.168.100.50}
  \item il sistema operativo di \texttt{192.168.100.50} riceve il pacchetto ed
  esegue l'inverso del NAT, l'indirizzo IP destinazione appare ora essere \texttt{192.168.200.2}.
  Poiché il client ha una rotta che dice: \textit{rete 192.168.200.2 via virtual NIC's IP}
  tale pacchetto viene inoltrato alla scheda di rete virtuale,
  \item il pacchetto viene ricevuto da OpenVPN ed inviato al server
  \item il server OpenVPN riceve il pacchetto e lo scrive sulla scheda di rete virtuale
  \item il sistema operativo del server riceve il pacchetto e lo inoltra quindi a
  \texttt{192.168.200.2}.
\end{enumerate}


\subsection{Dinamicità della creazione dei client}
\subsection{Il problema}
Si è visto precedentemente che è necessario introdurre nel file di configurazione del
server due direttive fondamentali per ogni client, esse sono:
\begin{itemize}
  \item \texttt{route <rete-client>}
  \item \texttt{iroute <rete-client>} (nel file di configurazione specifico)
\end{itemize}
Il file di configurazione viene letto da OpenVPN all'avvio\ldots Il problema ora
è il seguente: \textit{come poter aggiungere nuovi client dinamicamente al server
senza bisogno di riavviarlo?}
\subsection{La soluzione}
La soluzione di questo problema sfrutta gli hook \texttt{client-connect} e
\texttt{client-disconnect}. L'idea è
quella di eseguire direttamente un comando per l'aggiunta di una rotta al
kernel del sistema operativo quando un client si connette, e di rimuoverla quando
si disconnette. L'effetto di ``\texttt{route}'' è proprio questo, quindi ciò che si fa
è di eseguire i comandi che sarebbero eseguiti da OpenVPN in maniera diretta.\\
Per quanto riguarda la direttiva ``\texttt{iroute}'' invece non ci sono problemi,
una volta che ``\texttt{client-config-dir}'' è stata configurata correttamente, è
possibile aggiungere in tale cartella il file specifico per il nuovo client con la
direttiva ``\texttt{iroute}''.\\
Il file di configurazione del server quindi non presenta più le direttive
di tipo ``\texttt{route}'' %e ``\texttt{push "route \ldots"}
, vi sono però le seguenti aggiunte:
\begin{minted}{squidconf}
client-up /etc/openvpn/server/cs/client-up.sh
client-down /etc/openvpn/server/cs/client-down.sh
\end{minted}
Quando OpenVPN esegue i due script passa ad essi alcune variabili, in particolare
quella di interesse è \texttt{common\_name}, che indica il nome
presente nel certificato fornito dal client.\\
Si supponga ora che vi sia un secondo
client, chiamato ``\texttt{client50}'' responsabile per la rete \texttt{192.168.50.0/24};
OpenVPN rilegge i due script ad ogni connessione/disconnessione, pertanto è possibile
aggiornarli mentre esso è in esecuzione.
Ecco come si presentano i due file:
\begin{minted}[breaklines]{bash}
 # /etc/openvpn/server/cs/client-up.sh

#!/bin/bash


if [ "$common_name" == "client100" ]; then
  ip route add 192.168.100.0/24 via 10.7.0.1
fi
if [ "$common_name" == "client50" ]; then
  ip route add 192.168.50.0/24 via 10.7.0.1
fi
\end{minted}
\begin{minted}[breaklines]{bash}
# /etc/openvpn/server/cs/client-down.sh

#!/bin/bash


if [ "$common_name" == "client100" ]; then
  ip route del 192.168.100.0/24 via 10.7.0.1
fi
if [ "$common_name" == "client50" ]; then
  ip route del 192.168.50.0/24 via 10.7.0.1
fi
\end{minted}
Nel momento in cui un client si connette, si aggiunge una nuova rotta nel kernel
del sistema operativo, e quando si disconnette la si rimuove.\\
La ragione per cui si usa ``\texttt{if-fi, if-fi}'' e non ``\texttt{if-elif-fi}''
è che questo file viene generato automaticamente, ed utilizzare ``\texttt{if-fi, if-fi}''
è più semplice: quando si aggiunge un nuovo client è sufficiente aggiungere un
nuovo ``\texttt{if-fi}'' al file, mentre se si fosse seguita la seconda strada
sarebbe stato necessario modificare l'ultimo ``\texttt{fi}'' in un ``\texttt{elif}''.\\
Quindi, sfruttando le direttive ``\texttt{client-connect}'' e ``\texttt{client-disconnect}''
si è trovato un modo per bypassare la staticità del file di configurazione del server.\\



\subsection{Dinamicità delle rotte server-side}
\subsubsection{Il problema}
Questo problema è simile al precedente. La direttiva ``\texttt{push "route <server-net>"}''
viene utilizzata nel server per indicare al client quale sia la/e rete/i che si trovano
``dietro'' il server OpenVPN e che si vuole far sì che il client possa raggiungere.
Questo presuppone di sapere in partenza quali siano queste reti, ma in un ambiente
dinamico quale MoonCloud ciò non è possibile. Può capitare che i \textit{Docker host}
che vengono aggiunti on-demand appartengano alla stessa rete del server, come
può anche capitare che appartengano ad un'altra rete che il server a priori non conosce.
Il quesito da
risolvere in questo caso è il seguente: \textit{come gestire la dinamicità delle rotte
lato server senza bisogno di riavviare il server?}
Prima di vedere la soluzione si esamina il flusso di un pacchetto nel caso in cui
tale direttiva non venga più inclusa, sempre assumendo le configurazioni precedenti.
\begin{enumerate}
  \item Il \textit{probe} genera il pacchetto, la sorgente di tale pacchetto è
  \texttt{192.168.200.2}, la destinazione \texttt{192.168.100.254}
  \item il sistema operativo del \textit{Docker host} consulta la propria routing table
  ed instrada il pacchetto verso \texttt{192.168.200.20}
  \item il sistema operativo del VPN server riceve il pacchetto, consulta la
  routing table e trova l'entry \texttt{192.168.100.0/24 via 10.7.0.1}, quindi inoltra
  alla scheda di rete virtuale di OpenVPN
  \item OpenVPN riceve il pacchetto, consulta la propria routing table interna e
  trova l'entry che specifica che il client ``\texttt{client100}'' è responsabile per
  la rete \texttt{192.168.100.0/24}, provvede quindi ad inviarlo a tale client
  \item il VPN client riceve il pacchetto dal socket, e lo scrive sulla scheda di rete
  virtuale
  \item il pacchetto destinato a \texttt{192.168.100.254} viene ricevuto dal sistema
  operativo del VPN client, dopo aver compiuto le dovute operazioni di routing decide
  di inviarlo
  sulla scheda di rete fisica dell'host
  \item il pacchetto passa per l'hook di \texttt{POSTROUTING} e si modifica l'indirizzo IP
  sorgente in \texttt{192.168.100.50}
  \item \texttt{192.168.100.254} riceve il pacchetto, lo elabora, produce una risposta,
  la destinazione di tale risposta è \texttt{192.168.100.50}
  \item il sistema operativo di \texttt{192.168.100.254} invia la risposta a
  \texttt{192.168.100.50}
  \item il sistema operativo di \texttt{192.168.100.50} riceve il pacchetto ed
  esegue l'inverso del NAT, l'indirizzo IP destinazione appare ora essere \texttt{192.168.200.2}
  \item poiché non vi sono rotte configurate per \texttt{192.168.200.0/24}, il sistema
  operativo decide di inoltrare il pacchetto al default gateway
  \item il default gateway della rete \texttt{192.168.100.0/24} riceve il pacchetto
  destinato a \texttt{192.168.200.2}, poiché è un indirizzo IP privato ed esso
  lo instraderebbe su Internet, se è configurato correttamente lo droppa, altrimenti
  lo invia al suo default gateway ed il dropping del pacchetto viene solo posticipato.
\end{enumerate}

\subsubsection{La soluzione}
La soluzione è quella di far sì che tutti i pacchetti che provengono dalle reti interne
``dietro'' il server OpenVPN
verso le reti target arrivino ai client \textit{con l'indirizzo IP sorgente uguale
all'indirizzo IP della scheda di rete virtuale OpenVPN del server}.
Questo è vantaggioso poiché quando si assegna un indirizzo IP ad una scheda di rete, virtuale
o meno, il sistema operativo aggiunge automaticamente una rotta verso il NET ID dell'indirizzo
assegnato passando per tale scheda di rete. Ad esempio, se si assegna \texttt{10.7.0.2/24}
a \texttt{ovpn-client}, il kernel crea una rotta così fatta: \texttt{10.7.0.0/24 dev ovpn-client1}.

In questo modo le risposte ai pacchetti provenienti dalla VPN hanno come indirizzo IP di destinazione
quello della scheda di rete virtuale del server, il quale si trova nella stessa subnet VPN
della NIC del client, quindi il kernel del VPN client sa dove instradarlo senza che OpenVPN server
pubblicizzi delle rotte.


Lato server ciò viene fatto mediante
NAT usando iptables, ed i comandi sono ancora una volta presenti
nei file specificati nelle direttive ``\texttt{client-disconnect}'' e ``\texttt{client-disconnect}''.
\begin{minted}[breaklines]{bash}
# /etc/openvpn/server/cs/client-up.sh

#!/bin/bash


if [ "$common_name" == "client100" ]; then
  ip route add 192.168.100.0/24 via 10.7.0.1
  iptables -t nat -A POSTROUTING -d 192.168.100.0/24 -j MASQUERADE
fi
if [ "$common_name" == "client50" ]; then
  ip route add 192.168.50.0/24 via 10.7.0.1
  iptables -t nat -A POSTROUTING -d 192.168.50.0/24 -j MASQUERADE
fi
\end{minted}
\begin{minted}[breaklines]{bash}
# /etc/openvpn/server/cs/client-down.sh

#!/bin/bash


if [ "$common_name" == "client100" ]; then
  ip route del 192.168.100.0/24 via 10.7.0.1
  iptables -t nat -D POSTROUTING -d 192.168.100.0/24 -j MASQUERADE
fi
if [ "$common_name" == "client50" ]; then
  ip route del 192.168.50.0/24 via 10.7.0.1
  iptables -t nat -D POSTROUTING -d 192.168.50.0/24 -j MASQUERADE
fi
\end{minted}
Si descrive quindi il flusso
seguito da un pacchetto dall'invio di un pacchetto da parte di un \textit{probe} con questa nuova
soluzione.
\begin{enumerate}
  \item Il \textit{probe} genera il pacchetto, la sorgente di tale pacchetto è
  \texttt{192.168.200.2}, la destinazione \texttt{192.168.100.254}.
  \item il sistema operativo del \textit{Docker host} consulta la propria routing table
  ed instrada il pacchetto verso \texttt{192.168.200.20}
  \item il sistema operativo del VPN server riceve il pacchetto, consulta la
  routing table e trova l'entry \texttt{192.168.100.0/24 via 10.7.0.1}
  \item il pacchetto transita per l'hook di \texttt{POSTROUTING}, si modifica l'indirizzo IP
  sorgente in \texttt{10.7.0.1}, quindi viene inoltrato
  alla scheda di rete virtuale di OpenVPN
  \item OpenVPN riceve il pacchetto, consulta la propria routing table interna e
  trova l'entry che specifica che il client ``\texttt{client100}'' è responsabile per
  la rete \texttt{192.168.100.0/24}, provvede quindi ad inviarlo a tale client
  \item il VPN client riceve il pacchetto dal socket, e lo scrive sulla scheda di rete
  virtuale
  \item il pacchetto destinato a \texttt{192.168.100.254} viene ricevuto dal sistema
  operativo, dopo aver compiuto le dovute operazioni di routing decide di inviarlo
  sulla scheda di rete fisica dell'host
  \item il pacchetto passa per l'hook di \texttt{POSTROUTING} e si modifica l'indirizzo IP
  sorgente in \texttt{192.168.100.50}
  \item \texttt{192.168.100.254} riceve il pacchetto, lo elabora, produce una risposta,
  la destinazione di tale risposta è \texttt{192.168.100.50}
  \item il sistema operativo di \texttt{192.168.100.254} invia la risposta a
  \texttt{192.168.100.50}
  \item il sistema operativo di \texttt{192.168.100.50} riceve il pacchetto ed
  esegue l'inverso del NAT, l'indirizzo IP destinazione appare ora essere \texttt{10.7.0.1}
  \item il sistema operativo inoltra il pacchetto alla scheda di rete virtuale, (poiché
  vi è una rotta aggiunta nel kernel che dice di instradare per \texttt{10.7.0.0/24}
  verso \texttt{10.7.0.2})
  viene quindi ricevuto da OpenVPN ed inviato al server
  \item il server OpenVPN riceve il pacchetto e lo scrive sulla scheda di rete virtuale
  \item il sistema operativo del server riceve il pacchetto ed applica l'inverso del NAT,
  ora l'indirizzo IP destinazione è \texttt{192.168.200.2}
  \item il pacchetto viene quindi inviato  a \texttt{192.168.200.2}.
\end{enumerate}
Questa soluzione viene detta ``\textit{NAT lato server}''.
