\section{Criticità}
Ogni volta che si invoca un comando in userspace per modificare il set
di regole, il programma in userspace ottiene dal kernel il \textit{blob}
di \textit{tutte} le regole, vi applica le modifiche, invia la versione
modificata al kernel.
Questo porta a numerosi problemi, tra cui il fatto che più il numero di regole
aumenta, più la modifica delle regole è lenta. In sostanza, ciò che
si vorrebbe è un aggiunta \textit{atomica} delle regole, che sia
più veloce ad affidabile.

Ci sono vari altri problemi con iptables che negli anni sono emersi, e la
maggior parte di essi non sono attualmente risolvibili se non con un grosso
cambio a livello progettuale, cambio che non è possibile perché richiederebbe
dei \textit{breaking changes}, e vista la vasta diffusione di iptables, questo
non è possibile.
Alcuni dei problemi sono:
\begin{description}
  \item[Duplicazione del codice] Per ogni protocollo gestito da iptables (o sue
  controparti), occorre che tale programma sappia gestire nel dettaglio
  tale protocollo, in modo che possa creare le strutture dati necessarie per
  il file da inserire nel kernel.\\
  Il kernel stesso deve infine poter gestire tale protocollo perché deve
  compiere le operazioni richieste.\\
  Ciò significa che \textit{per ogni nuovo protocollo}, esso deve essere
  supportato sia in userspace sia in kernel-space.
  \item[Target multipli]Si supponga che si voglia droppare un pacchetto
  \textit{e} dropparlo, in una sola regola non è possibile, è necessario, ad
  esempio, creare una nuova tabella.
  \item[Matching avanzato]iptables supprta matching di fatto controllando l'header
  del protocollo di trasporto e l'header IP, tuttavia non è possibile scrivere
  in una \textit{regola} il fatto che il traffico TCP sulla porta 8 e \textit{e}
  443 sono consentite: occore scrivere almeno due regole. Questo diventa
  particolarmente tedioso quando il numero di regole aumenta molto (esistono
  estensioni come \texttt{multiport} per il matching di porte multiple, ma ne
  supporta fino a 15).
  \item[Multipli programmi]Un altra criticità è il fatto che vi sono 4 programmi
  userspace diversi, particolarmente problematica è la gestione separata di IPv4
  e IPv6.
  \item[Scalabilità]Come si comporta iptables/netfilter quando vi sono
  davvero molte regole (centinaia o più) regole da gestire ed il troughput
  è elevatissimo? La risposta è purtroppo semplice, iptables non scala, a causa
  del suo design che prevede l'attraversamento delle varie regole.
\end{description}

Questi problemi hanno portato gli sviluppatori di netfilter ad introdurre
una nuova soluzione, una soluzione molto più avanzata.
