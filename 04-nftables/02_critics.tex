\section{Criticità}
Ogni volta che si invoca un comando in userspace per modificare il set
di regole, il programma in userspace ottiene dal kernel il \textit{blob}
di \textit{tutte} le regole, vi applica le modifiche, invia la versione
modificata al kernel.
Questo porta a numerosi problemi, tra cui il fatto che più il numero di regole
aumenta, più la modifica delle regole è lenta.
Si vorrebbe un'aggiunta delle regole in batch ed \textit{atomica}, che
sia al contempo più veloce ed affidabile.

Ci sono vari altri problemi con iptables che negli anni sono emersi, e la
maggior parte di essi non sono attualmente risolvibili se non con un grosso
cambio a livello progettuale, cambio che non è possibile perché richiederebbe
dei \textit{breaking changes}, e vista la vasta diffusione di iptables questo
non è possibile.
Alcuni dei problemi sono:
\begin{description}
  \item[Duplicazione del codice]Per ogni protocollo gestito da iptables (o sue
  controparti), occorre che vi sia comprensione di esso sia a livello
  utente (perché il programma in userspace deve creare il \textit{blob} per
  il kernel), sia a livello di kernel-space, per operare correttamente sul pacchetto
  secondo quanto specificato dall'utente.\\
  Ciò significa che \textit{per ogni nuovo protocollo}, esso deve essere
  supportato sia in userspace sia in kernel-space.
  \item[Target multipli]Si supponga che si voglia loggare un pacchetto
  \textit{e} dropparlo, in una sola regola non è possibile, è necessario, ad
  esempio, creare una nuova tabella.
  \item[Matching avanzato]Se non si usano estensioni per ogni match si può specificare
  un solo valore per un certo campo. Ad esempio si può definire un match che controlli
  sia la porta TCP mittente sia quella del destinatario, ma comunque si
  deve specificare \textit{un solo} valore con cui fare match, \textit{solo} una porta mittente
  e \textit{solo} una destinatario.
  Quindi, se si vuole scrivere una regola che abiliti il traffico in entrata verso
  le porte 80 e 443 bisogna scrivere due regole, poiché vi sono due valori per uno stesso campo.
  % \item[Matching avanzato]iptables supprta matching controllando l'header
  % del protocollo di trasporto e l'header IP, tuttavia non è possibile scrivere
  % in una \textit{regola} il fatto che il traffico TCP sulla porta 8 \textit{e}
  % 443 sono consentite: occore scrivere almeno due regole. Questo diventa
  % particolarmente tedioso quando il numero di regole aumenta molto (esistono
  % estensioni come \texttt{multiport} per il matching di porte multiple, ma ne
  % supporta solo fino a 15).
  \item[Multipli programmi]Un'altra criticità è il fatto che vi sono 4 programmi
  userspace diversi, particolarmente problematica è la gestione separata di IPv4
  e IPv6.
  \item[Scalabilità]Come si comporta iptables/netfilter quando vi sono
  davvero molte regole (centinaia o più) regole da gestire ed il troughput
  è elevatissimo? La risposta è purtroppo semplice, iptables non scala, a causa
  del suo design che prevede l'attraversamento delle varie regole.
\end{description}

Questi problemi hanno portato gli sviluppatori di netfilter ad introdurre
una nuova soluzione, molto più avanzata \cite{nftables-pro}.
