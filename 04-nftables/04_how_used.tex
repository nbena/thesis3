\section{Come è stato usato}
Si supponga di voler gestire un VPN client che gestisce una singola rete classe C:
con il vecchio approccio di iptables, come guà detto, sono necessarie esattamente
19 regole:
\begin{itemize}
  \item 254 per trasformare gli indirizzi IP mappati negli indirizzi IP originali
  \item 254 regole per trasformare gli IP originali negli IP mappati
  \item 1 per gestire il \textit{NAT al contrario}
\end{itemize}
nftables è stato adottato prima che si conducessero veri test sul campo
in cui si potessero misurare le performance della soluzione basata su iptables,
quindi
non è stato possibile determinare quale fosse la scalabilità effettiva della vecchia
proposta, certo è che con con nftables \textit{si riduce il numero di regole
da 19 a 3}! Non solo, le nuove regole sono create utilizzando strutture
dati performanti, e sono e rimangono sempre 3, a prescindere che si il client VPN
rimappi una o $n$ reti.

In particolare si è sfruttata la struttura dati \textit{map} proprio come è stata
descritta nella sezione precedente. Si definiscono due \textit{named maps},
una per la gestione
del \texttt{PREROUTING} chiamata \texttt{remapping}, ed una per il \texttt{POSTROUTING}
chiamata \texttt{mapping}. La ragione per cui si sono scelte delle \textit{named maps}
è che consentono di essere modificate in seguito, per cui per qualsiasi eventualità
un operatore può agire direttamente su esse lasciando intatte le regole.
Di seguito un esempio, supponendo la seguente configurazione:
\begin{itemize}
  \item $OTN$: \texttt{192.168.100.0/24}
  \item $MTN$: \texttt{192.168.1.0/24}
  \item subnet VPN: \texttt{10.7.0.0/24}
\end{itemize}
Il file di script per nftables viene creato usando la sintassi \textit{strutturata},
di seguito se ne riporta, per brevità, solo un estratto.
\begin{minted}{squidconf}
#!/usr/sbin/nft -f
table ip ovpn_nat {

  map mapping {
    type ipv4_addr: ipv4_addr
    elements = {
      192.168.1.1: 192.168.100.1,
      192.168.2.1: 192.168.100.2,
      192.168.1.3: 192.168.100.3,
      192.168.1.4: 192.168.100.4,
      192.168.1.5: 192.168.100.5,
      192.168.1.6: 192.168.100.6,
      192.168.1.8: 192.168.100.8,
      192.168.1.9: 192.168.100.9,
    }
  }

  map remapping {
    type ipv4_addr: ipv4_addr
    elements = {
      192.168.100.1: 192.168.1.1,
      192.168.100.2: 192.168.1.2,
      192.168.100.3: 192.168.1.3,
      192.168.100.4: 192.168.1.4,
      192.168.100.5: 192.168.1.5,
      192.168.100.6: 192.168.1.6,
      192.168.100.7: 192.168.1.7,
      192.168.100.8: 192.168.1.8,
      192.168.100.9: 192.168.1.9      
    }
  }

  chain prerouting {
    type nat hook prerouting priority 0; policy accept;
    ip saddr 10.7.0.0/24 dnat to ip addr map @remapping
  }

  chain postrouting {
    type nat hook postrouting priority 0; policy accept;
    ip saddr 10.7.0.0/24 ip daddr 192.168.100.0/24 masquerade
    ip daddr 10.7.0.0/24 snat to ip saddr map @mapping
  }
}
\end{minted}
