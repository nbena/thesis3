\section{nftables}

%TODO BPF REF
La soluzione a questi problemi c'è e si chiama \texttt{nftables}.
Esso utilizza un approccio completamente diverso rispetto ai predecessori: una
\textit{in-kernel virtual machine} per il packet matching, concettualmente
simile a \texttt{BPF -- Berkeley Packet Filter}.
Vi è un unico programma in userspace chiamato
\texttt{nft}, il quale traduce in bytecode i comandi inseriti dall'utente e li
\textit{inietta} nel kernel (mediante \texttt{Netlink}\footnote{\texttt{Netlink}
è un meccanismo IPC -- Inter-Process Comunication del kernel Linux che consente
di comandare e ottenre informazioni riguardo lo stack di rete, è una interfaccia
che offre API basate su socket.}). Ecco un elenco
delle principali novità:
\begin{itemize}
  \item Il kernel non ha più conoscenza specifica
  dei protocolli di rete: fare match con una porta TCP o un flag dell'header Ethernet
  si traduce in semplici istruzioni come: \textit{carica x byte del pacchetto nel
  registro z a partire dall'offset y}. Questo significa che è molto facile estendere
  nftables con nuovi protocolli, e che il codice del kernel è molto più semplice.
  \item Non ci sono tabelle e chain di default.
  \item Ogni regola consente di pià target in una sola volta.
  \item Il programma \texttt{nft} diventa l'unico programma per la gestione delle regole,
  rimpiazzando iptables, ip6tables, arptables, ebtables.
  \item L'aggiunta, l'aggiornamento, la cancellazione ed ogni altra operazione
  sulle regole viene ora fatta in maniera atomica. Ciò significa che ogni volta che
  si fa un'operazione, anche su più regole, essa va a buon fine (cioè i suoi effetti
  sono applicati) solo se ogni singola operazione va a buon fine.
  \item Possibilità di gestire contemporanemente IPv4 ed IPv6 in una singola tabella/chain.
  \item Supporto a strutture dati avanzate basate su \texttt{set} che consentono un matching/targeting
  molto veloce.
  \item La sintassi delle regole cambia e diventa più simile ad una frase.
\end{itemize}

La maggior parte dell'\textit{intelligenza} è spostata su \texttt{nft}, il quale compila
le regole ricevute in input e le passa al kernel, esso è anche responsabile di
decompilare il bytecode presente nel kernel per presentarlo all'utente in un
formato human-readable.

Vi sono due tipi di sintassi supportate: il primo è una sintassi assimilabile a quella
di iptables dove si scrive una regola per volta. Il secondo tipo è invece
una sintassi \textit{strutturata} più simile a sintassi usati in specifici
file di configurazione. Saranno mostrati esempi di entrambe.


\paragraph{Tabelle e chain}
Si è detto che non esistono tabelle di default, quindi è spetta all'utente
il compito di crearle. In nftables, le tabelle sono di una certa \textit{famiglia}:
\begin{itemize}
  \item \texttt{ip} è la famiglia per operare su pacchetti dal livello 3 in su e
  che usano IPv4 come protocollo di rete; è la famiglia di default se non è specificata.
  \item \texttt{ip6} è la famiglia per operare con pacchetti di protocolli che utilizzano
  il protocollo IPv6.
  \item \texttt{arp} viene utilizzata per processare pacchetti ARP.
  \item \texttt{inet} è la famiglia che consente di lavorare con pacchetti sia IPv4
  sia IPv6.
  \item \texttt{bridge} è usata per lavorare con frame Ethernet.
  \item \texttt{netdev} consente di \textit{attaccarsi} direttamente ad una interfaccia
  di rete.
\end{itemize}
Le tabelle sono solo dei contenitori vuoti, il cui nome non ha alcuna semantica (rispetto
ad iptables). Comunque, la scelta della famiglia influenza quali hook possono
essere utilizzati (es: l'hook \texttt{ingress} è disponibile solo per \texttt{netdev}).

Le tabelle a loro volta racchiudono delle chain, ed anche in questo caso non vi
sono delle chain predefinite ma occorre crearle. Quando si crea una chain occorre
specificare
almeno tre cose:
\begin{itemize}
  \item tipo; le scelte possibili sono:
  \begin{itemize}
    \item \texttt{filter} da utilizzare quando si vuole fare del filtraggio
    \item \texttt{nat} per utilizzare le funzionalità NAT
    \item \texttt{route} per fare generico \textit{packet mangling} che non sia NAT
  \end{itemize}
  \item hook, in questo caso le scelte possibili sono gli hook già visti, cambia solo
  il fatto che vengono scritti in minuscolo.
  \item priorità: espressa con un numero, si tratta di una priorità \textit{inversa},
  più è bassa più il suo valore è alto; indica l'ordine con cui alcune operazioni
  sono svolte. Un \textit{buon} valore è 0.
\end{itemize}

Quando si crea una nuova chain è possibile specificare la policy di default, cioè
l'azione da compiere se non si matcha nessun'altra regola; il valore di default è
\texttt{accept} (accettazione il pacchetto), altri valori possibili sono: \texttt{drop},
\texttt{queue}, \texttt{continue}, \texttt{return}.


\subsection{Set e altre strutture dati}
nftables introduce tre strutture dati pensate indirizzando i problemi
presenti in iptables (e sue controparti):
\begin{itemize}
  \item matching limitato, per cui è problematico specificare più di una porta
  o più di un singolo indirizzo IP in una regola; il che porta a:
  \item elevato numero di regole a causa della suddetta limitazione.
\end{itemize}
E' possibile dichiarare queste strutture dati in due modi:
\begin{itemize}
  \item \textit{anonymous}: vengono create ed istanziate all'interno di una regola
  e sono immutabili, non è possibile modificarle dopo la loro creazione.
  \item \textit{Named}: sono vere e proprie variabili che vivono nello \textit{scope}
  di una tabella in cui sono definite. Una volta create è possibile modificarle
  in seguito. Per riferirsi ad una variabile, occorre preporre \texttt{@} al nome
  della variabile.
\end{itemize}
Esse consentono di ridurre il numero di regole necessarie rispetto
ai predecessori, e di compattarle usando delle strutture dati molto efficienti.

\subsubsection{Set}
La struttura dati avanzata di nftables si chiama \textit{set}, ed è implementata
utilizzando alberi rosso-neri, tabelle di hash, bitmap.
Un set è, come dice il nome, un insieme di elementi di un certo \textit{tipo},
esso può essere un tipo \textit{base} predefinito come \texttt{ipv4\_addr} o
\texttt{inet\_service} (porta protocollo di trasporto), oppure un tipo complesso
\textit{composto} da altri tipi, ad esempio un nuovo tipo \texttt{ipv4\_port}.
Quando si crea una nuova struttura dati è possibile specificare vari parametri,
tra cui una \texttt{policy}: per cosa deve essere ottimizzata la struttura?
I valori possibili sono \texttt{performance} o \texttt{memory}
Un esempio è il seguente.

I set sono utilizzati per:
\begin{itemize}
  \item fare match su più elementi in un unico regola, ad esempio, se si vogliono
  bloccare connessioni da 10 indirizzi IP diversi, è possibile definire un set
  che contenga questi 10 IP, anziché scrivere 10 regole diverse. Si consideri che
  i set sono molto performanti.
  \item Essere la struttura dati base su cui si creano ulteriori strutture dati.
\end{itemize}


\subsubsection{Map}
Le map sono costruite sui set, infatti possono essere visti come dei set in cui
ciascun elemento è composto a sua volta da due elementi: il primo di essi
è la \textit{chiave}, il secondo è il \textit{valore} associato alla chiave. Essi
si comportano effettivamente come una mappa poiché definiscono un \textit{mappaggio}
tra la chiave ed il valore, e sono utilizzati in contesti in cui si \textit{trasforma}
un input in un certo output secondo il mappaggio definito in questa struttura dati.
Si può forse intuire l'utilizzo delle map nel mappaggio delle reti\ldots


In particolare, le map sono molto utili con target di tipo \texttt{dnat} ed \texttt{snat}.
Si supponga che si voglia modificare l'indirizzo IP destinazione da \texttt{192.168.1.0/24}
nel corrispondente indirizzo in \texttt{192.168.100.0/24} (mantenendo lo stesso
offset) in \texttt{prerouting}, ed il viceversa in \texttt{postrouting}. Si
è già visto che farlo in iptables richiede 508 regole, 254 per ogni indirizzo in
\texttt{prerouting} e 254 per ogni indirizzo in \texttt{postrouting}. Con nftables
lo si può fare in due regole e due mappe!
L'esempio qui riportato mostra come farlo, ed è esattamente ciò che è stato fatto
nella mia soluzione finale di mappaggio delle reti (si veda la sezione successiva
per maggiori dettagli).


\subsubsection{Dictionary}
La struttura dati \textit{dictionary}, a vole nota anche come \textit{vmap -- verdict map}
, è concettualmente
molto simile alle map, tuttavia gli elementi di questo insieme sono nelle forma:
\texttt{key: verdict}, dove \texttt{Verdict} indica l'azione da applicare nel
momento in cui si la chiave viene
matchata.

\paragraph{Esempi}
Di seguito si riportano degli esempi che mostrano il funzionamento di nftables.
Si noti in particolare come, nelle ultime quattro regole l'utilizzo dei set
consente di specificare match multipli in un'unica regola. Questo non significa
solo una comodità per chi scrive le regole, infatti i set sono molto performanti
e sono le strutture dati base su cui nftables costruisce altre strutture
più avanzate, che verranno descritte in seguito.

Supponendo che ciò che si vede listato qui sotto sia il contenuto di un file,
si può notare lo shebang all'inizio, in modo che sia eseguito da nft nella
modalità nativa di shell che offre: tutte queste regole sono inserite atomicamente
in una transazione. Invocando questi comandi in una shell normale invece si
perderebbe l'atomicità perché ciascun comando è una invocazione separata al programma nft
(si noti che qui tutte le regole non sono appunto prefisse da ``\textit{nft}'').
\begin{minted}{squidconf}
#!/usr/sbin/nft -f

# creazione della tabella, sia IPv4 sia IPv6
# chiamata 'filter'
add table inet filter

# aggiunta chain 'input'
add chain inet filter input { type filter hook input priority 0; \
policy drop;}

# chain 'output'
add chain inet filter output { type filter hook output priotrity 0; \
policy drop;}

# consentire il traffico in entrata sulla porta 22 in modo statefull
add rule inet filter input tcp dport 22 ct state\
{new, related, established} accept

# consentire le risposte
add rule inet filter output tcp sport 22 ct state\
{related, established} accept

# consentire tutto il traffico in uscita sulle
# porte 80 e 443 usando un set
add rule inet output tcp dport { 80, 443 } ct state \
{new, related, established} accept

# consentire le risposte
add rule input tcp sport { 80, 443 } ct state \
{related, established} accept

# creazione di un set chiamato 'allowed'

add set filter allowed { type: ipv4_addr }
add element filter allowed {
  192.168.1.20,
  192.168.2.20,
  192.168.100.1,
  192.168.200.254,
  10.7.0.20
}

# consentire tutte le connessioni dagli indirizzi IP
# elencati sulla porta 8080
# nota: per accedere ad una named set si utilizza '@'
add rule input ip saddr @allowed tcp dport 8080 \
ct state {new, related, established } accept
# consentire le risposte
add rule output ip daddr @allowed tcp sport 8080 \
ct state {related, established }accept
\end{minted}
E' possibile ottenere lo stesso risultato usando una sintassi \textit{strutturata}:
\begin{minted}{squidconf}
#!/usr/sbin/nft -f
table inet filter {
  set allowed {
    type ipv4_addr
    elements = {
      192.168.1.20,
      192.168.2.20,
      192.168.100.1,
      192.168.200.254,
      10.7.0.20
    }
  }

  chain input {
    type inet hook input priority 0; policy drop;
    tcp dport 22 ct state {new, related, established} accept
    tcp sport {80, 443} ct state {related, established} accept
    ip saddr @allowed tcp dport 8080 ct state {new, related,
      established} accept
  }

  chain output {
    type inet hook output priority 0; policy drop;
    tcp sport 22 ct state {related, established} accept
    tcp dport {80, 443}  ct state {new, related, established} accept
    ip addr @allowed tcp sport 8080 ct state {related,
      established} accept
  }
}
\end{minted}
Per ottenere la lista delle regole attive, è disponibile il comando
\texttt{nft list ruleset}, che produce un output nello stesso formato \textit{strutturato}
apppena mostrato.

L'esempio seguente si concentra invece sull'utilizzo dei dizionari.
Si supponga di dover gestire una serie di regole relative alla porta 22:
\begin{itemize}
  \item consentito dalle reti \texttt{192.168.100.0/24}, \texttt{192.168.101.0/24},
  \texttt{192.168.200.0/24}
  \item proveniente dalla rete \texttt{192.168.1.0/24} occorre \texttt{REJECT}
  \item tutto il resto non è consentito.
\end{itemize}
\begin{minted}[breaklines]{bash}
iptables -t filter -A INPUT -p tcp --dport 22 -s 192.168.100.0/24 -j ACCEPT
iptables -t filter -A INPUT -p tcp --dport 22 -s 192.168.101.0/24 -j ACCEPT
iptables -t filter -A INPUT -p tcp --dport 22 -s 192.168.200.0/24 -j ACCEPT

iptables -t filter -A INPUT -p tcp --dport 22 -s 192.168.1.0/24 -j REJECT
iptables -t filter -A INPUT -p tcp --dport 22 -j DROP

# per brevità le regole in OUPUT sono omesse
\end{minted}
Mentre con nftables:
\begin{minted}{squidconf}
#!/usr/sbin/nft -f

# supponendo tabelle e chain già create
# aggiunta di 'dict' alla tabella filter
add map filter dict { type ipv4_addr: verdict; }

add element filter dict { 192.168.100.0/24: accept, \
192.168.101.0/24: accept, \
192.168.200.0/24: accept, \
192.168.1.0/24: reject }

add rule filter input tcp dport 22 ip saddr vmap @dict
add rule filter input tcp dport 22 drop
\end{minted}
