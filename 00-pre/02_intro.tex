\chapter*{Introduzione}

Il successo del cloud computing ha imposto un grande cambiamento nell'ambiente dell'IT,
portando indubbi benefici sotto numerosi punti di vista, anche economici. Dall'altro
lato si pongono problemi di sicurezza derivanti proprio da questo outsourcing, per i
quali sono richiesti nuovi approcci. Tra questi, uno molto efficace è quello basato
sulla raccolta continua di evidenze presso il sistema cloud, al fine di valutare
tramite esse quale sia l'effettivo stato della sicurezza.


MoonCloud è un prodotto per \textit{Security Assessment}  e \textit{Security Assurance}
basato su tale approccio, 
i cui target sono proprio i sistemi cloud. L'obiettivo di questa tesi è stato
estendere MoonCloud per permettere di effettuare attività di analisi di sicurezza anche
verso sistemi IT tradizionali, compresi anche cloud privati. Poiché tali sistemi
sono confinati entro un firewall e non accessibili dall'esterno, vi è stata la necessità
di disporre di hook \textit{dall'interno} di queste reti target. In particolare,
si è studiato come collegare la rete MoonCloud ai sistemi dei clienti mediante
una VPN.\\
Questo collegamento deve essere il più possibile trasparente per i clienti, nei limiti
del possibile non si vuole effettuare alcun tipo di configurazione presso essi; inoltre,
deve esserci un alto grado di automatizzazione nella gestione della VPN.
Questo ha posto di fronte a sfide che in una tradizionale VPN non si hanno, ed ha portato
a diverse soluzioni innovative, tra cui l'\textit{IP mapping} ed il \textit{NAT al contrario}.


L'elaborato è organizzato come segue:
\begin{description}
    \item[Capitolo 1 -- MoonCloud] In questo capitolo si presenta MoonCloud, se ne
    descrive il funzionamento e l'architettura, unitamente
    alle motivazioni che hanno portato all'utilizzo di una VPN.
    \item[Capitolo 2 -- VPN] Il secondo capitolo fornisce una panoramica sulle
    principali tecnologie VPN disponibile. Per ciascuna si analizzano
    i pro ed i contro, unitamente alle possibili integrazioni tra esse e MoonCloud.
    \item[Capitolo 3 -- OpenVPN] \textit{OpenVPN} è la tecnologia scelta: nella
    prima parte del capitolo la si descrive, mentre nella seconda si dettagliano
    i problemi che sono stati incontrati unitamente alle loro soluzioni innovative.
    Di particolare rilevanza vi è l'\textit{IP mapping} per gestire potenziali conflitti degli
    indirizzi IP.
    \item[Capitolo 4 -- nftables] \textit{nftables} è il successore di \textit{iptables},
    e come tale porta numerose migliorie rispetto al precedente. Questo capitolo analizza
    prima \textit{iptables}, identificandone le principali criticità, per mostrare
    infine \textit{nftables} e come si integri nella soluzione proposta.
    \item[Capitolo 5 -- Sicurezza della VPN] Il collegamento VPN \textit{espone}
    parte della rete di MoonCloud che normalmente non sarebbe accessibile, per questo
    sono state adotatte una serie di contromisure volte alla sua protezione: sono descritte
    in questo capitolo.
    \item[Capitolo 6 -- MoonCloud\_VPN] \textit{MoonCloud\_VPN} è il nome del
    microservizio che è stato realizzato come contorno della soluzione VPN, e si
    occupa della sua gestione. I suoi requisiti, la sua architettura, il suo
    funzionamento sono dettagliati nel sesto capitolo.
\end{description}