\documentclass[10pt,a4paper]{article}
    \usepackage[italian]{babel}
    \usepackage{lmodern}
    \usepackage[T1]{fontenc}
    \usepackage[utf8]{inputenc}
    \usepackage{pdfpages}
    \usepackage{palatino}
    \usepackage[left=2cm, right=2cm]{geometry}

    %\usepackage[a-1b]{pdfx}
    \usepackage[pdftex,
                pdfauthor={Nicola Bena},
                pdftitle={Studio ed implementazione di una architettura avanzata
                basata su VPN per Security Assessment}]{hyperref}


    \begin{document}

        \title{\textbf{Studio ed implementazione di una architettura avanzata
        basata su VPN per Security Assessment}}
        \author{Nicola Bena (matricola 870103)}
        \date{}

        \maketitle

        \vspace{0.5cm}

        %\begin{minipage}{.99\linewidth}
        \begin{minipage}{\linewidth}
            \begin{tabular}{l r}
                \begin{minipage}[t]{.4\linewidth}
                    \begin{flushleft}
                        {%\large
                            RELATORE\\[.15cm]
                            Prof. Marco Anisetti
                        }
                    \end{flushleft}
                \end{minipage}
                &
                \begin{minipage}[t]{.53\linewidth}
                    \begin{flushright}
                        {%\large
                            CORRELATORI\\[.15cm]
                            Prof. Claudio A. Ardagna\\[.1cm]
                            Dott. Filippo Gaudenzi
                        }
                    \end{flushright}
                \end{minipage}
            \end{tabular}
        \end{minipage}

        \vspace{2cm}

        Con l'avvento del cloud computing risulta necessario rivoluzionare il modo di svolgere pratiche quali il
        \textit{Security Assessment} e la \textit{Security Assurance}. I più efficaci approcci
        proposti nel panorama scientifico ed industriale risultano essere quelli basati su raccolta continua
        di evidenze per valutare l'evoluzione dello stato della sicurezza del sistema target.
        Tra essi vi è MoonCloud, spin-off dell'Università di Milano, che fornisce una soluzione di
        security governance per la cloud \textit{as a Service}, basata su raccolta di evidenze.
        La tesi è stata sviluppata in collaborazione con tale azienda, al fine di poter eseguire
        le verifiche di sicurezza anche in ambiti di cloud private o reti aziendali tradizionali,
        senza sconvolgere il paradigma \textit{as a Service}. 

        
        Il lavoro svolto si può articolare come segue:
        \begin{enumerate}
            % \item Studio delle possibili soluzioni architetturali che permettano di istaurare
            % un ponte tra MoonCloud e la rete privata da ispezionare, mantenendo il paradigma \textit{as a Service}. %impalcatura as a Service
            \item Studio delle possibili soluzioni architetturali basate su appliance VPN che permettano di istaurare
            un ponte tra MoonCloud e la rete privata da ispezionare, mantenendo il paradigma \textit{as a Service}.
            Si è effettuato uno studio approfondito sulle tecnologie VPN esistenti e sulle topologie realizzabili
            con esse, al fine di valutare quali fossero le più adatte.
            \item Design e realizzazione di una architettura VPN non standard basata su \textit{OpenVPN} tra
            VPN appliance e MoonCloud. 
            Vengono individuate e descritte diverse soluzioni innovative per la gestione delle reti target %il mapping della rete target all'interno di MoonCloud
            (\textit{NAT al contrario} e \textit{IP Mapping}). Il
            \textit{NAT al contrario} serve per evitare di impostare rotte presso il cliente, mentre
            l'\textit{IP Mapping} consiste nel
            \textit{mappare} gli indirizzi IP delle reti target in nuovi
            indirizzi garantiti univoci e quindi utilizzabili dalle sonde MoonCloud per
            identificare i target dell'ispezione. Ciò consente di gestire tutti i possibili
            conflitti di indirizzi IP tra le reti
            dei clienti. Il sistema sviluppato ha la caratteristica di essere completamente trasparente
            al cliente finale oltre ad essere estremamente automatizzato.
            \item La creazione di microservizi a supporto dell'automazione
            della gestione della VPN, per svolgere le seguenti principali operazioni:
            \textit{i)} creazione di tutti i file di configurazione per \textit{OpenVPN},
            \textit{ii)} gestione dei certificati mediante i quali avviene l'autenticazione
            nella VPN,
            \textit{iii)} gestione dell'\textit{IP Mapping},
            \textit{iv)} trasferimento coerente dei file di configurazione per e verso i server VPN in MoonCloud. % occorre specificare in MoonCloud
            % perché non è stato mai detto prima dove sono i server.
            % \end{itemize}
        \end{enumerate}


        Come prospettiva futura si sta studiando un'architettura distribuita, basata sempre
        su VPN, che sposti parte della computazione richiesta per l'ispezione direttamente nelle reti target. In questo
        modo le componenti di MoonCloud che effettivamente raccolgono evidenze agiscono direttamente
        dal cliente, e sulla VPN transitano solo richieste e risposte, diminuendo quindi
        il traffico generato.

    \end{document}