\documentclass[10pt,a4paper]{article}
    \usepackage[italian]{babel}
    \usepackage{lmodern}
    \usepackage[T1]{fontenc}
    \usepackage[utf8]{inputenc}
    \usepackage{pdfpages}
    \usepackage{palatino}
    \usepackage[left=2cm, right=2cm]{geometry}


    \begin{document}

        \title{\textbf{Studio ed implementazione di una architettura avanzata
        basata su VPN per Security Assessment}}
        \author{Nicola Bena (matricola 870103)}
        \date{}

        \maketitle

        \begin{minipage}{.99\linewidth}
            \begin{tabular}{l r}
                \begin{minipage}{.4\linewidth}
                    \begin{flushleft}
                        {\large
                            RELATORE\\[.15cm]
                            Prof. Marco Anisetti
                        }
                    \end{flushleft}
                \end{minipage}
                &
                \begin{minipage}{.6\linewidth}
                    \begin{flushright}
                        {\large
                            CORRELATORI\\[.15cm]
                            Prof. Claudio A. Ardagna\\
                            Dott. Filippo Gaudenzi
                        }
                    \end{flushright}
                \end{minipage}
            \end{tabular}
        \end{minipage}

        Nell'ambito della mia tesi ho collaborato a MoonCloud, spinoff di questa università,
        il quale è un prodotto in grado di fornire valutazione e monitoraggio continuo
        di servizi cloud. In particolare, MoonCloud offre \textit{Security Assessment} e
        \textit{Security Assurance}, cioè la capacità di valutare, mediante
        la raccolta continua di evidenze, lo stato \textit{effettivo} della sicurezza
        di un sistema.

        Il target iniziale di MoonCloud sono i sistemi cloud, tutttavia si è voluto espandere
        tale target anche ai sistemi IT tradizionali (\textit{reti target}), ovvero sistemi
        confinati all'interno di una classica rete aziendale. Se, per quanto riguarda target nella
        cloud, ci si appoggia a degli \textit{hook} forniti dal cloud provider, per analizzare
        sistemi IT tradizionali è stato necessario ripensare a come MoonCloud possa dialogare
        con tali sistemi.

        MoonCloud viene attualmenete erogato come un servizio, ovvero l'utente finale
        non deve installare niente per analizzare i propri assert nella cloud, e nella transizione
        verso sistemi IT tradizionali si è cercato di mantenere questa caratteristica.


        La soluzione che si è trovata, e che è stata il punto di partenza della mia tesi,
        è stata quella di utilizzare una VPN per collegare MoonCloud alla rete
        da analizzare; il collegamento sarebbe avvenuto mediante un device VPN portato
        fisicamente nella rete, il quale si sarebbe collegato alla VPN di MoonCloud dando
        accesso a quest'ultima all'intera rete aziendale, il tutto senza che
        l'utente debba operare alcuna configurazione sulla propria rete (es: aprire porte
        del firewall, ecc\ldots), se non sul VPN client. Quest'ultimo aspetto, ovvero
        una soluzione \textit{configuration-free} è molto importante.

        Il mio lavoro che ha portato alla tesi si può articolare in tre parti principali:
        \begin{enumerate}
            \item uno studio sulle tecnologie VPN esistenti al fine di valutare quali
            fossero le più adatte per il problema
            \item una volta scelta la tecnologia VPN, si è trattato di capire come
            configurarla al meglio, risolvendo numerosi problemi che in una
            installazione VPN classica non si hanno. In particolare, in questa seconda
            fase si sono introdotti alcune soluzioni particolarmente innovative.
            \item la creazione di un microservizio da integrare in MoonCloud per
            supportare la nuova architettura.
        \end{enumerate}

        
        Durante la prima fase ho studiato le maggiori tecnologie VPN disponibili,
        tenendo presente che si richiedeva una VPN molto flessibile, anche a costo
        di non essere la più performante in assoluto. Naturalmente doveva essere
        molto sicura. La scelta è infine ricaduta su OpenVPN, una soluzione open-source
        ampiamente diffusa, sicura, e, soprattutto, flessibile, grazie alle sua numerose
        opzioni di configurazione.


        A questo punto ho affrontato il problema di configurare OpenVPN per
        MoonCloud. Posto che i server VPN verranno installati in MoonCloud, è stato
        fondamentale capire quali problemi tale soluzione
        presenta. In particolare, se ne sono individuati tre.
        \begin{description}
            \item[Staticità della configurazione]OpenVPN si configura mediante
            file di testo, che vengono letti solo all'avvio, tuttavia vi è la necessità
            di aggiungere dinamicamente dei nuovi client e configurare
            rotte, sia interne ad OpenVPN, sia nel kernel.
            Per risolvere questo problema
            si sono sfruttati degli \textit{hook} messi a disposizione da OpenVPN, che
            consentono di eseguire degli script quando si compiono certe azioni
            (es: quando un client si connette, aggiungi una rotta al sistema operativo).
            \item[\textit{Configuration-free}]I pacchetti inviati da MoonCloud alla rete
            target, una volta ricevuti nel target, hanno come indirizzo IP sorgente un IP appartenente
            alla rete interna MoonCloud, e le risposte ad essi devono quindi passare
            per il VPN client, anziché essere inviate al default gateway della rete target.
            Tuttavia non è pensabile di dover chiedere al cliente di effettuare tali configurazioni,
            pertanto si utilizza il cosidetto \textit{NAT al contrario}: il VPN
            client invia i pacchetti sulla rete target facendo del NAT specificando come IP sorgente
            il proprio. Poiché esso appartiene alla rete target, per definizione si trova
            nello stesso spazio di indirizzi, e quindi le risposte tornano ad esso anziché essere
            inviate al default gateway.
            \item[Conflitti di IP]Si vuole far sì che un VPN possa gestire il maggior numero
            di VPN client, tuttavia bisogna rispettare un vincolo fondamentale: ogni rete che
            partecipa alla VPN deve avere un NET ID diverso. E' ragionevole presumere che
            le reti target utilizzino degli indirizzi IP privati, che sono limitati, e che prima
            o poi vi sarà un conflitto. Per risolvere questo problema alla fonte, si è introdotto
            il concetto dell'\textit{IP mapping}. Quando si registra un nuovo VPN client, ad ogni
            rete target che esso raggiunge si assegna una nuova rete garantita univoca per
            il server a cui il client è connesso, e tutta MoonCloud conoscerà solo queste
            reti mappate, anziché le originali.
            Il VPN client è responsabile di fare l'operazione inversa, quando riceve pacchetti
            che hanno per destinazione la rete mappata, deve modificarli specificano come
            destinazione l'IP reale, viceversa per le risposte.
            Capire come fare ciò è stata la parte più difficile dell'intera tesi, alla fine
            si è deciso di utilizzare \texttt{nftables}, successore di iptables.
        \end{description}


        Come ultima parte, ho realizzato un microservizio in Python, linguaggio che non conoscevo,
        a supporto di questo nuova soluzione.
        In particolare, il microservizio mette a disposizione delle API REST che
        assolvono ai seguenti compiti:
        \begin{itemize}
            \item creazione di tutti i file di configurazione per la VPN
            \item gestione dei certificati mediante i quali avviene l'autenticazione
            nella VPN
            \item gestione dell'\textit{IP mapping}: assegnare correttamente le reti
            ai client e, dato un indirizzo IP originale, ritornare la sua versione mappata
            \item trasferimento dei file di configurazione per i server verso i server stessi.
        \end{itemize}


        Oltre all'architettura per MoonCloud basata su VPN, che consente di mantenerla
        \textit{as-a Service}, vi sono ulteriori evoluzioni in fase di studio, tra cui
        la possibilità di portare parte dei componenti MoonCloud presso i clienti.

    \end{document}