\documentclass[10pt,a4paper]{article}
    \usepackage[italian]{babel}
    \usepackage{lmodern}
    \usepackage[T1]{fontenc}
    \usepackage[utf8]{inputenc}
    \usepackage{pdfpages}
    \usepackage{palatino}
    \usepackage[left=2cm, right=2cm]{geometry}


    \begin{document}

        \title{\textbf{Studio ed implementazione di una architettura avanzata
        basata su VPN per Security Assessment}}
        \author{Nicola Bena (matricola 870103)}
        \date{}

        \maketitle

        \vspace{0.5cm}

        %\begin{minipage}{.99\linewidth}
        \begin{minipage}{\linewidth}
            \begin{tabular}{l r}
                \begin{minipage}[t]{.4\linewidth}
                    \begin{flushleft}
                        {%\large
                            RELATORE\\[.15cm]
                            Prof. Marco Anisetti
                        }
                    \end{flushleft}
                \end{minipage}
                &
                \begin{minipage}[t]{.53\linewidth}
                    \begin{flushright}
                        {%\large
                            CORRELATORI\\[.15cm]
                            Prof. Claudio A. Ardagna\\[.1cm]
                            Dott. Filippo Gaudenzi
                        }
                    \end{flushright}
                \end{minipage}
            \end{tabular}
        \end{minipage}

        \vspace{2cm}

        % MoonCloud è un prodotto in grado di fornire valutazione e monitoraggio continuo
        % di servizi cloud. In particolare, MoonCloud offre \textit{Security Assessment} e
        % \textit{Security Assurance}, cioè la capacità di valutare, mediante
        % la raccolta continua di evidenze, lo stato \textit{effettivo} della sicurezza
        % di un sistema.

        Il successo del cloud computing ha richiesto un nuovo modo di svolgere
        \textit{Security Assessment} e \textit{Security Assurance}: modelli
        basati sulla raccolta continua di evidenze per valutare lo stato effettivo della
        sicurezza si stanno dimostrando sempre più utili ed efficaci.

        Tra essi vi è MoonCloud, spin-off dell'Università di Milano, i cui target iniziali
        sono stati proprio i sistemi cloud. L'ambito di questa tesi è stata una collaborazione
        con tale azienda, al fine di espanderne i target anche a sistemi IT tradizionali
        o cloud privati, comunque confinati nell'ambito di una o più reti aziendali
        (dette \textit{reti target}).
        Per fare ciò, è necessario disporre di \textit{hook dall'interno} di tali
        reti, mentre per sistemi cloud si sfruttano hook messi a disposizione dal
        cloud provider.
        % Il target iniziale di MoonCloud sono i sistemi cloud, tutttavia si è voluto espandere
        % tale target anche ai sistemi IT tradizionali, ovvero sistemi
        % confinati all'interno di una classica rete aziendale (\textit{reti target}).
        % Se, per quanto riguarda target nella
        % cloud, ci si appoggia a degli \textit{hook} forniti dal cloud provider, per analizzare
        % sistemi IT tradizionali è stato necessario ripensare a come MoonCloud possa dialogare
        % con tali sistemi.
        Inoltre, MoonCloud viene erogato come un servizio, ovvero l'utente finale
        non deve installare niente per analizzare i propri asset nella cloud, e nella transizione
        verso sistemi IT tradizionali si è cercato di mantenere questa caratteristica.


        Il cosìdetto \textit{hook dall'interno} è una VPN tra
        MoonCloud e le reti da analizzare, mediante un device che viene portato presso i clienti
        e responsabile di instaurare il collegamento. E' molto importante che questa soluzione
        di VPN
        sia trasparente per il cliente finale (oltre che sicura), ovvero richieda di
        effettuare il minor numero di configurazioni possibili, possibilmente nessuna.\\
        Il lavoro svolto si può articolare in tre parti principali:
        \begin{enumerate}
            \item Uno studio sulle tecnologie VPN esistenti e delle topologie realizzabili
            con esse, al fine di valutare quali
            fossero le più adatte. La scelta è stata \textit{OpenVPN},
            un software consolidato e molto flessibile: in MoonCloud si posizionano
            i VPN server, mentre il device che si porta nella rete target funge da
            VPN client.
            \item Un'analisi delle problematiche di configurazione, derivanti anche
            dall'uso
            altamente automatizzato della VPN.
            In particolare,
            poiché si vuole far sì che un server possa servire il maggior numero
            di client possibili, è stato necessario dover gestire la possibilità
            che diversi client abbiano gli stessi indirizzi IP, causando quindi
            dei conflitti. L'innovativa soluzione è stata chiamata \textit{IP
            mapping}, e consiste nel \textit{mappare} le reti dei clienti su nuove
            reti garantite univoche, il tutto in maniera trasparente.\\
            Sempre ai fini della trasparenza di configurazioni per il cliente,
            è stato introdotto il concetto di \textit{NAT al contrario}.
            % una volta scelta la tecnologia VPN, si è trattato di capire come
            % configurarla al meglio, risolvendo numerosi problemi che in una
            % installazione VPN classica non si hanno. In particolare, in questa seconda
            % fase si sono introdotti alcune soluzioni particolarmente innovative.
            \item La creazione di un microservizio a supporto dell'automazione
            della gestione della VPN. Esso è scritto in Python ed espone API REST che
            assolvono ai seguenti compiti:
            \begin{itemize}
                \item creazione di tutti i file di configurazione per la VPN
                \item gestione dei certificati mediante i quali avviene l'autenticazione
                nella VPN
                \item gestione dell'\textit{IP mapping}: assegnare correttamente le reti
                ai client e, dato un indirizzo IP originale, ritornare la sua versione mappata
                \item trasferimento dei file di configurazione per i server verso i server stessi,
                mantenendone la coerenza.
            \end{itemize}
        \end{enumerate}

        
        % Durante la prima fase si sono studiante le maggiori tecnologie VPN disponibili,
        % tenendo presente che si richiedeva una VPN molto flessibile, anche a costo
        % di non essere la più performante in assoluto. Naturalmente doveva essere
        % molto sicura. La scelta è infine ricaduta su \textit{OpenVPN}, una soluzione open-source
        % ampiamente diffusa, sicura, e, soprattutto, flessibile, grazie alle sua numerose
        % opzioni di configurazione.


        % A questo punto si è affrontato il problema di configurare OpenVPN per
        % MoonCloud. Posto che i server VPN verranno installati in MoonCloud, è stato
        % fondamentale capire quali problemi tale soluzione
        % presenta. In particolare, se ne sono individuati tre.
        % \begin{description}
        %     \item[Staticità della configurazione]OpenVPN si configura mediante
        %     file di testo, che vengono letti solo all'avvio, tuttavia vi è la necessità
        %     di aggiungere dinamicamente dei nuovi client e configurare
        %     rotte, sia interne ad OpenVPN, sia nel kernel.
        %     Per risolvere questo problema
        %     si sono sfruttati degli \textit{hook} messi a disposizione da OpenVPN, che
        %     consentono di eseguire degli script quando si compiono certe azioni
        %     (es: quando un client si connette, aggiungi una rotta al sistema operativo).
        %     \item[\textit{Configuration-free}]I pacchetti inviati da MoonCloud alla rete
        %     target, una volta ricevuti nel target, hanno come indirizzo IP sorgente un IP appartenente
        %     alla rete interna MoonCloud, e le risposte ad essi devono quindi passare
        %     per il VPN client, anziché essere inviate al default gateway della rete target.
        %     Tuttavia non è pensabile di dover chiedere al cliente di effettuare tali configurazioni,
        %     pertanto si utilizza il cosidetto \textit{NAT al contrario}: il VPN
        %     client invia i pacchetti sulla rete target facendo del NAT specificando come IP sorgente
        %     il proprio. Poiché esso appartiene alla rete target, per definizione si trova
        %     nello stesso spazio di indirizzi, e quindi le risposte tornano ad esso anziché essere
        %     inviate al default gateway.
        %     \item[Conflitti di IP]Si vuole far sì che un VPN server possa gestire il maggior numero
        %     di VPN client, tuttavia bisogna rispettare un vincolo fondamentale: ogni rete che
        %     partecipa alla VPN deve avere un NET ID diverso. E' ragionevole presumere che
        %     le reti target utilizzino degli indirizzi IP privati, che sono limitati, e che prima
        %     o poi vi sarà un conflitto. Per risolvere questo problema alla fonte, si è introdotto
        %     il concetto dell'\textit{IP mapping}. Quando si registra un nuovo VPN client, ad ogni
        %     rete target che esso raggiunge si assegna una nuova rete garantita univoca per
        %     il server a cui il client è connesso, e tutta MoonCloud conoscerà solo queste
        %     reti mappate, anziché le originali.
        %     Il VPN client è responsabile di fare l'operazione inversa, quando riceve pacchetti
        %     che hanno per destinazione la rete mappata, deve modificarli specificano come
        %     destinazione l'IP reale, viceversa per le risposte.
        %     Per realizzare questa funzionalità si utilizza \textit{nftables}, successore
        %     di \textit{iptables}.
        % \end{description}
        % Come ultima parte, è stato realizzato un microservizio in Python 
        % a supporto di questo nuova soluzione.
        % In particolare, il microservizio mette a disposizione delle API REST che
        % assolvono ai seguenti compiti:

        Come prospettiva futura si sta studiando un'architettura distribuita, basata sempre
        su VPN, che sposti parte della computazione direttamente nelle reti target. In questo
        modo le componenti di MoonCloud che effettivamente raccolgono evidenze agiscono direttamente
        dal cliente, e sulla VPN transitano solo richieste e risposte, diminuendo quindi
        il traffico generato.

    \end{document}