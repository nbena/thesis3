\section{Non solo cloud}\label{sec:mooncloud-not-only-cloud}
MoonCloud è stata progettata per poter essere eseguita sulla cloud, e più di tutto,
la si vuole \textit{as-a-service}
%\footnote{L'ultimo capitolo è dedicato ad ulteriori evoluzioni dell'architettura MoonCloud}  who knows
, ciò significa che i clienti che vogliono
certificare parte della propria infrastruttura cloud, non devono installare alcun
software, semplicemente usano la \texttt{Dashboard} per regolare i controlli
che vogliono fare.

La verifica del fatto che, ad esempio, si utilizzi uno storage cifrato, può essere
effettuata mediante hook messi a disposizione dal cloud provider.

Tuttavia, si è voluto ampliare la platea di possibili target per MoonCloud:
si supponga che anziché voler analizzare la configurazione di OpenStack di
un deployment pubblico si vogliano verificare certe policy di \texttt{Active
Directory} all'interno di una rete aziendale\footnote{\url{https://docs.microsoft.com/en-us/windows-server/identity/ad-ds/get-started/virtual-dc/active-directory-domain-services-overview}}.
Anziché utilizzare gli hook messi a disposizione da un cloud provider, occorrerebbe
utilizzare le API di \texttt{Active Directory}, e come veicolo di accesso ad esse
non più (ad esempio) HTTP, ma bisognerebbe aver accesso alla rete interna
in cui Windows Server è in esecuzione.
Infatti, l'obiettivo di MoonCloud per espandere i propri clienti è quello di non
solo essere in grado di analizzare dei target nelle cloud pubbliche, ma anche in sistemi IT tradizionali
o cloud private.
Ciò pone un grande problema: l'accesso alla rete fisica, infatti non si potrebbero
più utilizzare gli hook forniti dal cloud provider.

A maggior ragione, il problema si fa più complesso se si vogliono verificare
proprietà relative a servizi \textit{solo interni} di una rete, che non sono
esposti all'esterno.

Occorre quindi cambiare metodo di accesso, in particolare occorre avere accesso
\textit{dall'interno}, visto che tali reti (definite \textit{reti target})
sono ragionevolmente e auspicabilmente protette da un firewall.

Nella fattispecie, si è investigato l'utilizzo di una soluzione basata su VPN,
per la quale si porta fisicamente nella rete target un VPN client (\textit{VPN appliance}),
responsabile
di connettersi alla rete MoonCloud e di fornire un canale di comunicazione
altamente sicuro.
Le Docker machine restano deployate nella cloud in cui MoonCloud stessa è, e le
loro richieste di analisi passano tramite la VPN.

Sono stati definiti i seguenti requisiti:
\begin{itemize}
	\item soluzione altamente sicura
	\item se possibile, basata su TLS/HTTPS. Questo perché si ritiene che
	      almeno il traffico HTTP ed HTTPS sia consentito anche dai firewall
	      più stringenti. Una VPN basata su TLS quindi è ragionevolmente consentita
	      senza necessità di riconfigurare il firewall aziendale.
	      Questo si è anche tradotto in una preferenza di TCP rispetto ad UDP.
	\item Preferire VPN configurabili e flessibili rispetto a
	      VPN performanti ma poco adattibili ai requisiti.
	\item La soluzione VPN deve essere il più possibile lightweight
	      per la rete target, ovvero deve poter, nei limiti del
	      possibile, funzionare senza dover configurare nulla nella rete
	      target (ad eccezione del VPN client stesso).
\end{itemize}

\begin{figure}
	\includegraphics[scale=0.6]{img/mooncloud_archi_extended}
	\label{fig:mooncloud-archi-extended}
	\caption[L'architettura di MoonCloud estesa con VPN]
	{L'architettura con l'utilizzo di una VPN per
	consentire l'analisi di sistemi IT tradizionali}
\end{figure}