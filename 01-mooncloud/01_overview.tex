\section{Overview}\label{sec:mooncloud-overview}
MoonCloud è un framework per la valutazione ed il monitoraggio continuo
di servizi cloud.
Molto sinteticamente, si definiscono delle proprietà (es: \textit{cifratura
dei dati salvati su storage}) e si verifica in maniera continua che tale
proprietà sia sempre rispettata. Il rispetto della proprietà porta al
rilascio di un certificato che ne attesta il rispetto. Nel momento in cui
tale rispetto viene meno il certificato diventa revocato, e, si presume
che vengano al più presto applicate azioni correttive per ripristinare
la proprietà.


Il framework prevede il coinvolgimento di diversi soggetti:
\begin{itemize}
    \item \textit{CA -- Certification Authority} responsabile del
    rilascio dei certificato
    \item \textit{accreditated lab} che si occupa dell'analisi e della
    verifica delle proprietà (può essere il framework MoonCloud stesso)
    \item \textit{service provider} ovvero lo sviluppatore di di applicazioni
    basate sul cloud
    \item \textit{cloud provider}, il fornitore della piattaforma/infrastruttura
    cloud. Può essere esso stesso a volersi certificare, oppure semplicemente
    supportare le applicazioni del \textit{service provider}.
\end{itemize}

Si definisce il concetto di \textit{ToC -- Target of Certification} come il
contesto di applicazione ed il perimetro del cloud entro il quale
si vuole effettuare una valutazione.
Il processo di certificazione si svolge in due fasi Nella prima fase
avviene la verifica del rispetto della proprietà in un ambiente di
laboratorio controllato (\textit{pre-deployment evaluation}), allo scopo
di collezionare evidenze.
Nel caso in cui le evidenze diano esito positivo, la CA rilascia
un certificato per il ToC.

Nella seconda fase, detta \textit{production evaluation}, l'accreditated
lab, mediante test continui, verifica che il certificato sia o meno
valido.

\subsection{Proprietà, meccanismi, certificati}
Fondamentale è la separazione tra \textit{proprietà} e \textit{meccanismi}.

Una \textit{proprietà non funzionali} è definita
come ``una proprietà astratta presa da vocabolari comuni o domain-specific, ed un set
di attributi che la raffinano'' \cite{mooncloud-semi-automatic-and-trustworthy}.

Un \textit{meccanismo non funzionale} viene definito come ``un tipo di meccanismo (es: cifratura)
ed un set di attributi che specificano la configurazione del meccanismo (es: lunghezza chiave),
unitamente a configurazioni cloud-specific che interessano il comportamento del
meccanismo''\cite{mooncloud-semi-automatic-and-trustworthy}.

Questi due concetti si riflettono su due \textit{modelli} necessari per l'attività di
certificazione.
\begin{description}
    \item[\textit{CM Template -- Certificate Model Template $\mathcal{T}$}]E' un modello dichiarativo che
    descrive ad alto livello che cosa deve essere fatto per verificare delle proprietà.
    Viene definito dalla CA.
    \item[\textit{CM Instance -- Certificate Model Instance $\mathcal{I}$}]Si tratta di un modello 
    eseguibile generato istanziando un $\mathcal{T}$ su un vero ToC.
\end{description}
Il certificato viene rilasciato quando le attività indicate da $\mathcal{I}$ danno un
riscontro positivo, esso contiene una descrizione della proprietà certificata,
un riferimento alle evidenze collezionate, ed è legato al ToC.


Affinché le evidenze raccolte ed i risultati derivati da esse siano conformi alla
realtà è fondamentale che tutti i modelli usati siano conformi a quelli reali. Ad esempio
i modelli di evidenze che ci si aspetta siano corretti devono corrispondere alle
evidenze effettive che provino la correttezza. Lo stesso vale per l'intera modellazione
del sistema \cite{mooncloud-modelling-time}.

Oltre al test di proprietà funzionali, recentemente (\cite{mooncloud-modelling-time}),
MoonCloud ha introdotto anche verifichi di vincoli basati sul tempo (es: garantire che
un tempo di risposta di una API inferiore ad $x$ ms), sulla probabilità che
certi eventi si verifichino, su particolari configurazioni che devono essere adottati
all'interno di un modello più ampio, ed anche alla verifica di attacchi noti.